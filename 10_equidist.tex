
\chapter{一様分布論}

\section{一様分布定理}

この章では一様分布定理と呼ばれる今までとは少し趣の異なる極限に関する定理を紹介する。
内容的には高度なものになるので読み物としてみた方がよいかもしれない。

その一様分布定理を述べるために次の用語を定義する。

\begin{definition}[一様分布列]
各$n$に対して$x_n \in [0, 1)$を満たす数列$(x_n)_{n = 1}^\infty$が$[0, 1)$で\emph{一様分布}するとは任意の$0 \le a < b \le 1$に対して
$$
\lim_{N \to \infty}\#\lrset{ n = 1, \cdots, N \mid a \le x_n < b } = b-a
$$
が成り立つことである。
ただし、有限集合$A$に対して、自然数として定まる元の個数を$\# A$と書く。
\end{definition}

この条件は区間$[a, b)$に入る$x_n$の割合が$b-a$であるとみなせる。
以降では$n = 1, \cdots, N$をしばしば$n \le N$と表現する。

$[0, 1)$という区間は実数$x$の床(整数部分)$\lrfloor{x}$を引いた小数部分$\lrfrac{x}$の属する集合として現れた。
そこで一般の実数列$(x_n)$に対しても小数部分の列$(\lrfrac{x})$が一様分布するかどうかで一様分布列の概念を導入する。
なお、$\lrfrac{x}$という記号は集合の記号と紛らわしいがよく使われるので本テキストでも採用し文脈によってどちらの意味か判断される。

\begin{remark}
小数部分の演算について任意の実数$x, y$に対して、$\lrfrac{x} \ge \lrfrac{y}$である場合は$\lrfrac{x-y} = \lrfrac{x}-\lrfrac{y}$であり、そうでない場合は$\lrfrac{x-y} = \lrfrac{x}-\lrfrac{y}+1$であることに注意する。
\end{remark}

以上の準備の下で一様分布定理は次のように表現される。

\begin{theorem}[一様分布定理]
$\theta$を無理数とするとき、数列$(\lrfrac{n\theta})_{n = 1}^\infty$は一様分布する。
\end{theorem}

$[0, 1)$の$0$と$1$と引っ付けてわっかのように見なすと、$(\lrfrac{n\theta})_n$というのはわっかをぐるぐるまわる感じになっており回転と呼ばれ$\theta$は角度のような役割を果たす。

この定理において$\theta$が無理数であることが非常に重要で、例えば$\theta$が有理数$\frac{1}{p}$の場合は$n$が$p$増えるともとの位置に戻るため、$\lrfrac{n\theta}$の取りうる値は$\lrset{ \frac{0}{p}, \cdots \frac{p-1}{p} }$となり一様分布にはならない。
無理数の場合は表現が正確ではないが分母$p$が無限大のような状況になっていて、$(\lrfrac{n\theta})_n$は$[0, 1)$上均等にまんべんなく分布するということを定理は主張している。
