%!texpl_options -x platex
\documentclass{jsbook}
\usepackage{mathtools}
\usepackage{amsmath}
\usepackage{amssymb}
\usepackage{amsthm}
\usepackage{physics}
\usepackage{makeidx}
\usepackage{mdframed}

% \usepackage{comment}
% \usepackage{lineno}
% \pagewiselinenumbers
\usepackage{showkeys}
\newcommand{\todo}[1]{\marginpar{TODO: #1}}
\usepackage[dvipdfmx]{hyperref}
\usepackage{pxjahyper}

% \graphicspath{{fig/}}

\setlength{\textwidth}{\fullwidth}
\setlength{\evensidemargin}{\oddsidemargin}

% \newtheorem{theorem}{定理}[section]
\newmdtheoremenv[linecolor=blue]{theorem}{定理}[section]
\newmdtheoremenv[linecolor=blue]{lemma}[theorem]{補題}
\newmdtheoremenv[linecolor=blue]{proposition}[theorem]{命題}
\newmdtheoremenv[linecolor=blue]{corollary}[theorem]{系}

\theoremstyle{definition}
% \newtheorem{definition}[theorem]{定義}
\newmdtheoremenv[linecolor=blue]{definition}[theorem]{定義}
\newtheorem{example}[theorem]{例}
\newtheorem{xca}[theorem]{練習問題}

\theoremstyle{remark}
\newtheorem{remark}[theorem]{注意}

\renewcommand{\proofname}{証明}
\renewcommand{\qedsymbol}{【証明終わり】}
% \makeatletter
% \AtBeginEnvironment{proof}{\let\@addpunct\@gobble}
% \makeatother
\makeatletter
\renewenvironment{proof}[1][\proofname]{\par
  \pushQED{\qed}%
  \normalfont \topsep6\p@\@plus6\p@\relax
  \trivlist
  \item\relax
  {\itshape
  【#1】}\hspace\labelsep\ignorespaces
}{%
  \popQED\endtrivlist\@endpefalse
}
\makeatother

\numberwithin{equation}{section}

\DeclareMathOperator{\opabs}{abs}
\DeclareMathOperator{\sgn}{sgn}
\DeclareMathOperator{\floor}{floor}
\DeclareMathOperator{\ceil}{ceil}
\DeclareMathOperator{\arsinh}{arsinh}
\DeclareMathOperator{\arcosh}{arcosh}
\DeclareMathOperator{\artanh}{artanh}
\DeclareMathOperator{\sinc}{sinc}

\DeclarePairedDelimiter{\lrparen}{(}{)}
\DeclarePairedDelimiter{\lrset}{\{}{\}}
\DeclarePairedDelimiter{\lrabs}{|}{|}
\DeclarePairedDelimiter{\lreval}{[}{]}
\DeclarePairedDelimiter{\lrfloor}{\lfloor}{\rfloor}
\DeclarePairedDelimiter{\lrceil}{\lceil}{\rceil}

\makeindex

\begin{document}

\title{微分積分学1}
\author{中安淳}
\date{\today}

\maketitle


\chapter*{序文}

これは1変数の微分積分学に関するテキストです。


\tableofcontents


\chapter{集合と論理}

\section{集合、命題、写像}

\emph{集合}とはいくつかの「もの」の集まりであり、その集合に所属するかしないか判定できるものをいう(ことにする)。
ここでいう集合に所属する「もの」のことを\emph{元}または\emph{要素}という。
「もの」としては整数や実数のような「数」を想定しているが、関数や図形さらには集合自体まで「もの」として認めることがある。
つまり、集合の集合などもあり得て、実際後に述べる実数の実体は特殊な集合でありその集合が実数の集合としてこのテキストで頻繁に出てくる(が、集合としての実数は最初に基礎的な性質を示す時のみに使われてあとは集合であることを意識せずに数として扱うことができる)。
集合$A$と「もの」$a$に対して$a$が$A$の元であるとき、$a$は$A$に\emph{属する}や含まれるあるいは入っているなどと表現し、$a \in A$と表す。
そうでない時は、$a \notin A$と書く。

集合を記述する方法にはいくつかありこのテキストでは四種類使う。
そのうち一つが外延記法といって元を列挙して記述する方法であり、例えば$1, 2, 3$からなる集合は
$$
\lrset{ 1, 2, 3 }
$$
と書かれる。
なお、この集合に対しては$1 \in \lrset{ 1, 2, 3 }$で、$5 \notin \lrset{ 1, 2, 3 }$である。
集合で重要なことは元が属するかどうかであり、属する順番は関係がない。
また、上記の記法で重複したものがあっても重複分は無視される。
つまり、
$$
\lrset{ 1, 2, 3 } = \lrset{ 3, 2, 1 } = \lrset{ 1, 1, 2, 2, 3, 3 }
$$
である。

このテキストでは元の列挙の際に誤解の余地がない場合にいくつかの元を省略する。
その際には記号$\cdots$を用いる。
例えば、
$$
\lrset{ 0, 1, 2, 3, 4, 5, 6, 7, 8, 9 } = \lrset{ 0, 1, 2, 3, \cdots, 9 }
$$
や、$N$を自然数として
$$
\lrset{ 0, 1, 2, 3, \cdots, N }
$$
などである。

この省略に関係して、集合の二つ目の記法があり、列挙が無限に続く場合にも省略を用いた外延記法を認める。
これにより自然数全体の集合は
$$
\lrset{ 0, 1, 2, 3, \cdots }
$$
と表される。
さらにこのテキストでは前方にも省略を使うことを認めて、整数全体の集合を
$$
\lrset{ \cdots, -3, -2, -1, 0, 1, 2, 3, \cdots }
$$
や偶数全体の集合を
$$
\lrset{ \cdots, -6, -4, -2, 0, 2, 4, 6, \cdots }
$$
などと記述する。

上記二つの外延記法を使っている場合に限り、例えば「任意の$n \in \lrset{ 0, 1, 2, 3, \cdots, N }$に対して」という記述を「任意の$n = 0, 1, 2, 3, \cdots, N$に対して」などと書くことがある。

三つ目の記法に関連する概念が命題である。
\emph{命題}とは真か偽か判定可能な条件を記述した文字列である。
例えば$1 \in \lrset{ 1, 2, 3 }$は真の命題で、$5 \in \lrset{ 1, 2, 3 }$は偽の命題である。
命題はしばしば$P$や$Q$の文字が割り当てられ、命題中に変数$x$や$y$がある場合は$P(x)$や$Q(x, y)$などと書かれる。
集合$X$が与えられてその元$x \in X$のうち命題$P(x)$が真になるようなものを集めて得られる集合を
$$
\lrset{ x \in X \mid P(x) }
$$
と記述する。
命題については次の節で詳しく述べる。

四つ目の集合の記法を説明するために写像の概念を導入する。
集合$X$から集合$Y$への\emph{写像}とは$X$の各元$x$に対して$Y$のただ一つの元$y$を対応させる規則である。
この時の集合$X$を写像の定義域、集合$Y$を写像の値域といい、値域$Y$が数の集合のとき写像は特に関数と呼ばれる。
写像を
$$
F: X \to Y
$$
や$x$を$y$に対応させていることを明示する際は
$$
F: x \in X \mapsto y \in Y
$$
のように記述する。
また、$x$に対して対応する$y$のことを$F(x)$と書き、写像$F$のことを$y = F(x)$などと書くこともある。

四つ目の集合の記法はこの写像$F$を使ったものであり、$X$の各元$x$を集合$Y$の元$F(x)$に対応させ集めた集合を
$$
\lrset{ F(x) \mid x \in X}
$$
と書く。
三つ目と四つ目の記法を同時に用いて
$$
\lrset{ F(x) \mid x \in X, P(x) }
$$
と記述することもある。

三つ目と四つ目の記法を合わせて内包記法という。
この内包記法を用いることで様々な集合を記述することができる。

元を一つも持たない集合を\emph{空集合}といい$\emptyset$で表す。

\section{命題と論理}

命題から次のようにして新しい命題を作る。

\begin{definition}[命題の演算]
命題$P$, $Q$に対して以下を定める。
\begin{itemize}
\item
$P$, $Q$の両方が真のときのみ真とする命題を「$P$\emph{かつ}$Q$」といい$P \land Q$と表す。
\item
$P$, $Q$の両方または片方が真のときのみ真とする命題を「$P$\emph{または}$Q$」といい$P \lor Q$と表す。
\item
$P$が真のとき偽、偽のとき真とする命題を「$P$\emph{でない}」といい$\lnot P$と表す。
\item
$Q$が真または$P$が偽のときのみ真とする命題を「$P$\emph{ならば}$Q$」といい$P \implies Q$と表す。
\end{itemize}
\end{definition}

このうち「$P$ならば$Q$」の定義で$P$が偽のときならばの命題は真としていることは不自然に思えるかもしれない。
例えば極端な例では「円周率が$3$ならば$1 = 0$」という命題は真ということになる。
命題$1 = 0$は偽なのでおかしく感じるかもしれないが、以降の数学の議論を行う上ではこのようにした方が都合がよい。
論理と言語の間にはすこし隙間があると考えるとよいかもしれない。

\todo{これらの命題の演算を表にまとめると以下になる。}

命題$P$と$Q$の真偽が一致するとき「$P$と$Q$は\emph{同値}」といい$P \iff Q$あるいは$P = Q$と表す。

前節でも述べたように命題中に変数$x$や$y$がある場合は$P(x)$や$Q(x, y)$などと書かれる。
この変数付き命題から以下の新しい命題を作る。

\begin{definition}[全称と存在]
$X$を集合として各元$x \in X$に対して命題$P(x)$が与えられるとする。
\begin{itemize}
\item
すべての$x \in X$に対して命題$P(x)$が真であるときのみ真である命題を「\emph{任意}の$x \in X$に対して$P(x)$」といい$\forall x \in X, P(x)$と表す。
\item
命題$P(x)$が真である$x \in X$が一つでもあるときのみ真である命題を「ある$x \in X$が\emph{存在}して$P(x)$」といい$\exists x \in X, P(x)$と表す。
\end{itemize}
\end{definition}

\begin{remark}
$X$が空集合のとき、命題「任意の$x \in X$に対して$P(x)$」は真とする。
\end{remark}

命題の演算の性質として以下が挙げられる。
\begin{enumerate}
\item
(結合法則)任意の命題$P$, $Q$, $R$に対して、$(P \land Q) \land R = P \land (Q \land R)$, $(P \lor Q) \lor R = P \lor (Q \lor R)$。
\item
(交換法則)任意の命題$P$, $Q$に対して、$P \land Q = Q \land P$, $P \lor Q = Q \lor P$。
\item
(結合法則)任意の命題$P$, $Q$, $R$に対して、$P \land (Q \lor R) = (P \land Q) \lor (P \land R)$, $P \lor (Q \land R) = (P \lor Q) \land (P \lor R)$。
\item
(二重否定)任意の命題$P$に対して、$\neg (\neg P) = P$。
\item
(ド・モルガンの法則)任意の命題$P$, $Q$に対して、$\neg (P \land Q) = \neg P \lor \neg Q$, $\neg (P \lor Q) = \neg P \land \neg Q$。
\item
(背理法の原理)任意の命題$P$, $Q$に対して、$\neg (P \implies Q) = P \land \neg Q$。
\item
(対偶)任意の命題$P$, $Q$に対して、$P \implies Q = \neg Q \implies \neg P$。
\item
任意の命題$P(x)$に対して、$\neg (\forall x \in X, P(x)) = \exists x \in X, \neg P(x)$。
\item
任意の命題$P(x)$に対して、$\neg (\exists x \in X, P(x)) = \forall x \in X, \neg P(x)$。
\end{enumerate}

背理法の原理にもとづいて命題$P \implies Q$が真であることを示すために$P \land \neg Q$が偽であることつまり$P$と$\neg Q$を仮定して矛盾を導くことを示す証明法を\emph{背理法}という。
また、命題$P \implies Q$が真であることを示すために\emph{対偶}$\neg Q \implies \neg P$が真であることを示す証明法を\emph{対偶論法}という。

\section{集合の演算}

集合と集合の関係で重要なものとして以下の\emph{包含}がある。

\begin{definition}[集合の包含]
$A$, $B$を集合とする。
\begin{itemize}
\item
任意の$x \in A$が$x \in B$となっている時、集合$A$は集合$B$に\emph{含まれる}あるいは集合$A$は集合$B$の\emph{部分集合}であるといい、$A \subset B$あるいは$B \supset A$と表す。
\item
$A \subset B$かつ$A \supset B$のとき、集合$A$と集合$B$は\emph{等しい}といい、$A = B$と表す。
\end{itemize}
\end{definition}

集合と集合から以下の集合を定義する。

\begin{definition}[集合の演算]
$X$を集合として、$A$, $B$をその部分集合とする。
\begin{itemize}
\item
集合$\lrset{ x \in X \mid x \in A \land x \in B }$を$A$と$B$の\emph{共通部分}といい$A\cap B$と表す。
\item
集合$\lrset{ x \in X \mid x \in A \lor x \in B }$を$A$と$B$の\emph{和集合}といい$A\cup B$と表す。
\item
集合$\lrset{ x \in X \mid x \in A \land x \notin B }$を$A$と$B$の\emph{差集合}といい$A\setminus B$と表す。
\item
特に集合$\lrset{ x \in X \mid x \notin A } = X\setminus A$を$A$の\emph{補集合}といい$A^c$と表す。
また、この時の$X$を\emph{全体集合}と表現することがある。
\end{itemize}
\end{definition}

命題の演算の性質に対応して集合の演算には次の性質がある。
\begin{enumerate}
\item
(結合法則)任意の集合$A$, $B$, $C$に対して、$(A\cap B)\cap C = A\cap (B\cap C)$, $(A\cup B)\cup C = A\cup (B\cup C)$
\item
(交換法則)任意の集合$A$, $B$に対して、$A\cap B = B\cap A$, $A\cup B = B\cup A$。
\item
(結合法則)任意の集合$A$, $B$, $C$に対して、$A\cap(B\cup C) = (A\cap B)\cup(A\cap C)$, $A\cup(B\cap C) = (A\cup B)\cap(A\cup C)$。
\item
任意の集合$A$に対して、$(A^c)^c = A$。
\item
(ド・モルガンの法則)任意の集合$A$, $B$に対して、$(A\cap B)^c = A^c\cup B^c$, $(A\cup B)^c = A^c\cap B^c$。
\end{enumerate}

\section{写像の像と逆像}

写像は集合の元を別の集合の元に対応させるものであるが、そこから発展させて部分集合を対応させることを考えることが今後必要になってくる。

\begin{definition}[写像の像と逆像]
$F$を集合$X$から集合$Y$への写像として、$A$を$X$の部分集合、$B$を$Y$の部分集合とする。
\begin{itemize}
\item
集合$\lrset{ F(x) \mid x \in A }$は$Y$の部分集合であり$A$の$F$による\emph{像}といい$F(A)$と表す。
\item
集合$\lrset{ x \in X \mid F(x) \in B }$は$X$の部分集合であり$B$の$F$による\emph{逆像}といい$F^{-1}(A)$と表す。
\end{itemize}
\end{definition}

\begin{example}
$f(x) = x^2$を考えると、$f(\lrset{0, 1}) = \lrset{0, 1}$, $f(\lrset{0, 1, -1}) = \lrset{0, 1}$。
また、$f^{-1}(\lrset{0, 1}) = \lrset{0, 1, -1}$, $f^{-1}(\lrset{-1}) = \emptyset$。
\end{example}

写像の像と逆像については以下の成立が成り立つ。
$F$は集合$X$から集合$Y$への写像である。
\begin{enumerate}
\item
任意の$X$の部分集合$A_1, A_2$に対して、$A_1 \subset A_2$ならば$F(A_1) \subset F(A_2)$。
\item
任意の$Y$の部分集合$B_1, B_2$に対して、$B_1 \subset B_2$ならば$F^{-1}(B_1) \subset F^{-1}(B_2)$。
\item
任意の$X$の部分集合$A_1, A_2$に対して、$F(A_1\cap A_2) \subset F(A_1)\cap F(A_2)$。
\item
任意の$X$の部分集合$A_1, A_2$に対して、$F(A_1\cup A_2) = F(A_1)\cup F(A_2)$。
\item
任意の$Y$の部分集合$B_1, B_2$に対して、$F^{-1}(B_1\cap B_2) = F(B_1)\cap F(B_2)$。
\item
任意の$Y$の部分集合$B_1, B_2$に対して、$F^{-1}(B_1\cup B_2) = F(B_1)\cup F(B_2)$。
\item
任意の$X$の部分集合$A$に対して、$F^{-1}(F(A)) \supset A$。
\item
任意の$Y$の部分集合$B$に対して、$F(F^{-1}(B)) \subset B$。
\end{enumerate}

\section{逆写像}

写像には可逆と呼ばれる特殊な扱いのできるものがある。

\begin{definition}[逆写像]
$F$を集合$X$から集合$Y$への写像とする。
集合$Y$から集合$X$への写像$G$であって、任意の$x \in X$に対して$G(F(x)) = x$が、任意の$y \in Y$に対して$F(G(y)) = y$が成り立つとき、$F$は\emph{可逆}であるといい$G$は$F$の\emph{逆写像}という。
また、逆写像$G$を$F^{-1}$と表す。
\end{definition}

つまり、$F^{-1}(F(x)) = x$, $F(F^{-1}(y)) = y$である。
逆写像は存在すれば一つしかない。
逆像も逆写像も同じ記号$F^{-1}$を用いるがそれぞれ集合の元と部分集合を当てはめて対象が違う概念である。
とはいえ可逆つまり逆写像がある写像に対しては$F^{-1}(\lrset{y}) = \lrset{F^{-1}(y)}$という関係式が成り立つ。
ここで、左辺の$F^{-1}$は逆像であり右辺の$F^{-1}$は逆写像として用いられている。

可逆は全単射とも呼ばれ、次の特徴づけがある。

\begin{proposition}
集合$X$から集合$Y$への写像$F$が可逆であることは次の二つの条件が成り立つことと同値である。
\begin{enumerate}
\item
(全射性)$F(X) = Y$である。
\item
(単射性)任意の$y \in Y$に対して$F(x) = y$となる$x \in X$は一意である。
すなわち$x_1, x_2 \in X$が$F(x_1) = F(x_2)$を満たすならば$x_1 = x_2$が成り立つ。
\end{enumerate}
\end{proposition}

\begin{proof}
$F$が可逆の時、逆写像$F^{-1}$が存在する。
全射性を示すために$y \in Y$を取ると$x = F^{-1}(y)$とすることで$F(x) = F(F^{-1}(y)) = y$。
よって$F(X) \supset Y$なので、$F(X) = Y$である。
単射性を示すために$F(x_1) = F(x_2)$を満たす$x_1, x_2 \in X$を取ると$x_1 = F^{-1}(F(x_1)) = F^{-1}(F(x_2)) = x_2$である。
以上より全射性と単射性が示された。

逆に$F$が全射性と単射性を満たす時、$y \in Y$に対して$F(x) = y$となる$x$が一意に存在するのでそれを$G(y)$とおいて$Y$から$X$への写像$G$を定める。
この時、$F(G(y)) = y$であり$G(F(x)) = G(y) = x$なので、$G$は$F$の逆写像となっている。
\end{proof}


\chapter{数}

\section{自然数}

\emph{自然数}とはものの個数や順番を表す数のことで$1, 2, 3$といった数である。
$-1$や$1.5$、$\sqrt{2}$などは自然数ではない。
$0$を自然数とするかどうかは状況によってまちまちであるがこのテキストでは$0$も自然数として考える。
\emph{自然数全体の集合}を$\mathbb{N}$で表す。
つまり、
$$
\mathbb{N} = \lrset{ 0, 1, 2, 3, \cdots }
$$
である。

自然数は一つ大きくしたものも自然数つまり$n \in \mathbb{N}$ならば$n+1 \in \mathbb{N}$が成り立つ。
また、何回大きくしてももとの自然数には戻らず、ずっと増えていくものである。
このことから自然数の集合は無限集合の典型例であり、また数列の添え字として利用される。
自然数(の集合)において重要なのはこのことであり、$0$を含むかどうかは些末な問題である。
このテキストでは抽象的な数列の添え字として利用する場合のみに$\mathbb{N}$を用いて、最初の数を明示したい場合は$0, 1, 2, 3, \cdots$や$1, 2, 3, \cdots$などと書く。

\emph{数学的帰納法}とは自然数$n$に関する命題$P(n)$の証明法である。
つまり、命題「任意の$n = 0, 1, 2, 3, \cdots$に対して$P(n)$」を証明するためには以下の二つの命題を示せばよい。
\begin{itemize}
\item
$P(0)$。
\item
任意の$n = 0, 1, 2, 3, \cdots$に対して$P(n) \implies P(n+1)$。
\end{itemize}
つまり最初の自然数で示して、その後一つずつ増やして証明していく論法であり、直接任意の自然数に対して証明できない場合でも数学的帰納法で証明できることが多い。
最初の自然数(上の場合は$0$)を取りかえることで証明できる自然数の範囲($n = 0, 1, 2, 3, \cdots$)を調節できるほか、
二つ目の条件の中の``$P(n)$''という部分をより示しやすく「任意の自然数$k = 0, 1, 2, 3, \cdots, n$に対して$P(k)$」に取りかえても同じ結論が得られる(累積帰納法)など様々な数学的帰納法の変種がある。

自然数には\emph{加法}(足し算)と\emph{乗法}(掛け算)が定義できる。
自然数$a$, $b$に対して、$a$を$b$回一つ大きくすることが加法であり、その結果を$a$と$b$の\emph{和}と言い$a+b$と表す。
また自然数$a$, $b$に対して、$0$に$b$回$a$を足すことが乗法であり、その結果を$a$と$b$の\emph{積}と言い$a\cdot b$と表す。
積$a\cdot b$は$a\times b$とも表されるほか、$b$が文字の場合は$\cdot$や$\times$は省略され$a b$と表される。

加法と乗法の重要な性質として、以下がある。
\begin{enumerate}
\item
(加法の結合法則)任意の自然数$a, b, c$に対して$(a+b)+c = a+(b+c)$。
\item
(加法の交換法則)任意の自然数$a, b$に対して$a+b = b+a$。
\item
(乗法の結合法則)任意の自然数$a, b, c$に対して$(a\cdot b)\cdot c = a\cdot (b\cdot c)$。
\item
(乗法の交換法則)任意の自然数$a, b$に対して$a\cdot b = b\cdot a$。
\item
(分配法則)任意の自然数$a, b, c$に対して$a\cdot (b+c) = a\cdot b+a\cdot c$、$(a+b)\cdot c = a\cdot c+b\cdot c$。
\item
(零元)$0$は零元である。つまり任意の自然数$a$に対して$a+0 = 0+a = a$と$a\cdot 0 = 0\cdot a = 0$が成り立つ。
\item
(単位元)$1$は単位元である。つまり任意の自然数$a$に対して$a\cdot 1 = 1\cdot a = a$が成り立つ。
\end{enumerate}

二つの自然数$a$, $b$の間には\emph{大小関係}が定義できる。
つまり、$a+x = b$を満たす自然数$x = 0, 1, 2, 3, \cdots$が存在する時$a \le b$と書き、そうでない時$a > b$と書く。
また、$a = b+x$を満たす自然数$x = 0, 1, 2, 3, \cdots$が存在する時$a \ge b$と書き、そうでない時$a < b$と書く。

この大小関係の重要な性質として、以下がある。
\begin{itemize}
\item
(全順序)任意の自然数$a, b$に対して$a \le b$または$a \ge b$のどちらかが成立し、両方が成立する場合は$a = b$に他ならない。
\item
(推移性)任意の自然数$a, b, c$に対して$a \le b$かつ$b \le c$ならば$a \le c$。
\item
(加法との両立)任意の自然数$a, b, c$に対して$a \le b$ならば$a+c \le b+c$と$a+c \le b+c$ならば$a \le b$。
\item
(乗法との両立)任意の自然数$a, b, c$に対して$a \le b$ならば$a c \le b c$。
\end{itemize}

また、次の基礎的な大小関係が成立する。
$$
0 < 1 < 2 < 3 < \cdots.
$$

\section{整数}

自然数には加法と乗法が定義されるが減法(引き算)が定義されるとは言えない。
これは$1$から$2$を引く場合に答えが自然数の範囲から出てしまうためである。

そのような場合のために負の符号$-$と自然数$a$の負数$-a$を導入する。
また、対応して正の符号$+$と自然数$a$の正数$+a$を導入する。
正の符号$+$を導入したがこれはしばしば省略される。
$+a$と$-a$をまとめて$\pm a$と表す。
\emph{整数}はこれらの数$\pm 0, \pm 1, \pm 2, \pm 3, \cdots$をまとめてさらに$\pm 0$を$0$として同一視したものである。
整数のうち$+0, +1, +2, +3, \cdots$の部分は自然数と同一視する。
\emph{整数全体の集合}を$\mathbb{Z}$で表す。
つまり、
$$
\mathbb{Z} = \lrset{ 0, \pm 1, \pm 2, \pm 3, \cdots } = \lrset{ \cdots, -3, -2, -1, 0, 1, 2, 3, \cdots }
$$
である。

整数の演算を定義するために、まず自然数の減法を答えが整数の範囲で定義する。
つまり自然数$a, b$に対して、$a \le b$の場合は$a+x = b$を満たす自然数$x$が存在するので$a-b = -x$とし、$a \ge b$の場合は$a = b+x$を満たす自然数$x$が存在するので$a-b = +x$とする。
$a \le b$か$a \ge b$の片方は成立するのと、両方が成立する場合は$a = b$で$a+x = a+x = a = b$なので$x = 0$で整数の定義で$\pm 0$を同一視しているのでこの減法は答えが整数の範囲で定義できる。

ここから整数の加法と乗法を次で定義できる。
$$
(+a)+(+b) = +(a+b),
\quad (+a)+(-b) = a-b,
\quad (-a)+(+b) = b-a,
\quad (-a)+(-b) = -(a+b).
$$
$$
(+a)\cdot (+b) = +(a\cdot b),
\quad (+a)\cdot (-b) = -(a\cdot b),
\quad (-a)\cdot (+b) = -(b\cdot a),
\quad (-a)\cdot (-b) = +(a\cdot b).
$$

整数$a$の\emph{反数}$-a$は$a$が正の符号を持つ場合は負の符号に、負の符号を持つ場合は正の符号に変えたものである。
整数の\emph{減法}は二つの整数$a$, $b$に対して
$$
a-b = a+(-b)
$$
として定義し、その結果を$a$と$b$の\emph{差}と言う。

以上の整数の加法、乗法、減法は自然数に制限すると(減法は答えが整数の範囲で)自然数の加法、乗法、減法と一致することに注意する。
また、以下の性質が成り立つ。
\begin{enumerate}
\item
(加法の結合法則)任意の整数$a, b, c$に対して$(a+b)+c = a+(b+c)$。
\item
(加法の交換法則)任意の整数$a, b$に対して$a+b = b+a$。
\item
(乗法の結合法則)任意の整数$a, b, c$に対して$(a\cdot b)\cdot c = a\cdot (b\cdot c)$。
\item
(乗法の交換法則)任意の整数$a, b$に対して$a\cdot b = b\cdot a$。
\item
(分配法則)任意の整数$a, b, c$に対して$a\cdot (b+c) = a\cdot b+a\cdot c$、$(a+b)\cdot c = a\cdot c+b\cdot c$。
\item
(零元)$0$は零元である。つまり任意の整数$a$に対して$a+0 = 0+a = a$と$a\cdot 0 = 0\cdot a = 0$が成り立つ。
\item
(反数)任意の整数$a$に対して$a+x = x+a = 0$が成り立つような整数$x = -a$がただ一つ存在する。
\item
(単位元)$1$は単位元である。つまり任意の整数$a$に対して$a\cdot 1 = 1\cdot a = a$が成り立つ。
\end{enumerate}

正の符号を持つ整数を正の整数、負の符号を持つ整数を負の整数と言う。
ただし、$0 = \pm 0$は例外でどちらでもない整数として扱う。

二つの整数$a$, $b$の間の大小関係$a \le b$, $a > b$, $a \ge b$, $a < b$を差$a-b$の符号によって定義する。
以下の性質が成り立つ。
\begin{itemize}
\item
(全順序)任意の整数$a, b$に対して$a \le b$または$a \ge b$のどちらかが成立し、両方が成立する場合は$a = b$に他ならない。
\item
(推移性)任意の整数$a, b, c$に対して$a \le b$かつ$b \le c$ならば$a \le c$。
\item
(加法との両立)任意の整数$a, b, c$に対して$a \le b$ならば$a+c \le b+c$。
\item
(乗法との両立)任意の整数$a, b, c$に対して$a \le b$, $c \ge 0$ならば$a c \le b c$。
\end{itemize}
基礎的な大小関係として次が成立する。
$$
\cdots < -3 < -2 < -1 < 0 < 1 < 2 < 3 < \cdots.
$$

\section{有理数}

減法を定義するために自然数を整数に拡張したように、除法(割り算)を定義するために整数を拡張する。
\emph{有理数}は二つの整数$n$, $m$ ($m \ne 0$)を使って分数$n/m$や$\frac{n}{m}$の形で表される数のことであり、
二つの有理数$n/m$, $l/k$が$n k = l m$を満たすとき同一視する(通分すると同じ数であるため)。
ここで分母$m$が$0$でない場合しか考えないのは$0$の場合を入れると後述の四則演算の性質をすべて成り立たせることができなくなるためである。
また、$(-n)/(-m) = n/m$であるため、$m$として正の整数のみ考えれば十分である。
有理数のうち$n/1$は整数$n$と同一視する。
\emph{有理数全体の集合}を$\mathbb{Q}$で表す。

有理数$a = n/m$に対して$n \ne 0$の時、$m/n$も有理数であり$a$の\emph{逆数}といい$a^{-1}$と表す。
有理数のうち$0$と同一視されるものは$0/m$ ($m \ne 0$)であることに注意すると、$0$以外の有理数には逆数が存在する。

有理数の加法、減法、乗法、\emph{除法}は以下で定義する。
$$
(n/m)+(l/k) = (n k+m l)/(m k),
\quad (n/m)-(l/k) = (n k-m l)/(m k),
$$
$$
(n/m)\times(l/k) = (n l)/(m k),
\quad (n/m)\div(l/k) = (n k)/(m l).
$$
ただし、除法は$l \ne 0$の場合に限って定義される。
除法の記号も$/$や分数の記号を使ったりする。
定義からすぐわかる通り、除法は逆数をかけることに相当する。
これら4つの演算をまとめて\emph{四則演算}という。

以上の有理数の加法、乗法、減法は整数に制限すると整数の加法、乗法、減法と一致することに注意する。
また、以下の性質が成り立つ。
\begin{enumerate}
\item
(加法の結合法則)任意の有理数$a, b, c$に対して$(a+b)+c = a+(b+c)$。
\item
(加法の交換法則)任意の有理数$a, b$に対して$a+b = b+a$。
\item
(乗法の結合法則)任意の有理数$a, b, c$に対して$(a\cdot b)\cdot c = a\cdot (b\cdot c)$。
\item
(乗法の交換法則)任意の有理数$a, b$に対して$a\cdot b = b\cdot a$。
\item
(分配法則)任意の有理数$a, b, c$に対して$a\cdot (b+c) = a\cdot b+a\cdot c$、$(a+b)\cdot c = a\cdot c+b\cdot c$。
\item
(零元)$0$は零元である。つまり任意の有理数$a$に対して$a+0 = 0+a = a$と$a\cdot 0 = 0\cdot a = 0$が成り立つ。
\item
(反数)任意の有理数$a$に対して$a+x = x+a = 0$が成り立つような有理数$x = -a$がただ一つ存在する。
\item
(単位元)$1$は単位元である。つまり任意の有理数$a$に対して$a\cdot 1 = 1\cdot a = a$が成り立つ。
\item
(逆数)任意の有理数$a$に対して$a\cdot x = x\cdot a = 1$が成り立つような有理数$x = a^{-1}$がただ一つ存在する。
\end{enumerate}

有理数$a = n/m$($m$は正の整数)の符号は$n$の符号と同じになるように定義する。
この符号は$n$, $m$の取り方によらないことがわかる。
二つの有理数$a$, $b$の間の大小関係$a \le b$, $a > b$, $a \ge b$, $a < b$を差$a-b$の符号によって定義する。
以下の性質が成り立つ。
\begin{itemize}
\item
(全順序)任意の有理数$a, b$に対して$a \le b$または$a \ge b$のどちらかが成立し、両方が成立する場合は$a = b$に他ならない。
\item
(推移性)任意の有理数$a, b, c$に対して$a \le b$かつ$b \le c$ならば$a \le c$。
\item
(加法との両立)任意の有理数$a, b, c$に対して$a \le b$ならば$a+c \le b+c$。
\item
(乗法との両立)任意の有理数$a, b, c$に対して$a \le b$, $c \ge 0$ならば$a c \le b c$。
\end{itemize}

\section{実数}

これまでの流れで四則演算が定義できるように数の範囲を拡張して有理数を得たが、
さらに極限と呼ばれる操作をしたときにその結果が有理数から出ても大丈夫なようにさらに数の範囲を拡張する。

実数はそのような拡張の一つで、ここでは\emph{デデキント切断}と呼ばれる方法で定義する。
つまり、次の条件を満たす有理数の集合$A \subset \mathbb{Q}$を\emph{有理数の切断}という。
\begin{itemize}
\item
任意の$a \in A$と$b \in A^c$に対して、$a \le b$が成り立つ。
\item
$A$は空集合$\emptyset$でない。
\item
$A$は全体集合$\mathbb{Q}$でない。
\end{itemize}
\emph{実数}はこのような有理数の切断であり、$1$つの有理数を取り除いたら有理数の切断ではなくなるものである。
有理数の切断の本質的な条件は最初のものであり、これにより有理数を数直線上で左($A$)と右($A^c$)に分けて(切断して)いる。
実数は切断する点に相当し切断する点が左には入れないとしたものである。

$a$を有理数とすると、$a$以下の有理数全体$\lrset{ x \in \mathbb{Q} \mid x \le a }$は有理数の切断である。
また、$a$未満の有理数全体$\lrset{ x \in \mathbb{Q} \mid x < a }$も有理数の切断である。
これらの有理数の切断は$a$が元としてあるかないかの違いであり、後者はこれ以上有理数を取り除けないので実数となり、有理数$a$と同一視する。

\begin{example}
実数であるが有理数ではない(有理数とはみなせない)数として$2$の平方根$\sqrt{2}$がある。
この有理数の切断としての表現は$\lrset{ x \in \mathbb{Q} \mid x < \sqrt{2} }$ということになるが$\sqrt{2}$は定義されていないので、今までの記号で書くならば$\lrset{ x \in \mathbb{Q} \mid x < 0 \lor x\cdot x < 2 }$となり確かにこれは有理数でない有理数の切断(実数)になっている。
\end{example}

実数のうち有理数でないものを\emph{無理数}という。
\emph{実数全体の集合}を$\mathbb{R}$で表す。

実数$a = A$, $b = B$の大小関係を$A \subset B$を満たすとき$a \le b$、そうでない時$a > b$として定義する。
また、$A \supset B$を満たすとき$a \ge b$、そうでない時$a < b$と定義する。
有理数の切断としての実数は切断する点が取り除かれていることに注意する。
以下の性質が成り立つ。
\begin{itemize}
\item
(全順序)任意の実数$a, b$に対して$a \le b$または$a \ge b$のどちらかが成立し、両方が成立する場合は$a = b$に他ならない。
\item
(推移性)任意の実数$a, b, c$に対して$a \le b$かつ$b \le c$ならば$a \le c$。
\item
(加法との両立)任意の実数$a, b, c$に対して$a \le b$ならば$a+c \le b+c$。
\item
(乗法との両立)任意の実数$a, b, c$に対して$a \le b$, $c \ge 0$ならば$a c \le b c$。
\end{itemize}

実数は有理数の隙間を埋めて得られると考えられるが、有理数は2つの実数の間にぎっしり詰まっていて次が成立する。

\begin{proposition}[有理数の稠密性]
\label{t_dense_rational}
実数$a$, $b$が$a < b$を満たす時、$a < x < b$を満たす有理数$x$が存在する。
\end{proposition}

\begin{proof}
有理数の切断による。
$a$の有理数の切断としての表現を$A$、$b$の有理数の切断としての表現を$B$とする。
$\overline{A}$を$a$が有理数のとき$A\cup\lrset{ a }$、そうでないとき$A$として定義すると$\overline{A}$も有理数の切断で、仮定より$x \in \overline{A}\setminus B$となる有理数$x$が存在する。
この$x$に対して$a < x < b$を示せばよい(詳細省略)。
\end{proof}

\begin{definition}[区間]
$a$, $b$を$a < b$を満たす実数として、以下の集合を定義し、まとめて\emph{区間}という。
$$
[a, b] = \lrset{ x \in \mathbb{R} \mid a \le x \le b },
\quad [a, b) = \lrset{ x \in \mathbb{R} \mid a \le x < b },
\quad [a, +\infty) = \lrset{ x \in \mathbb{R} \mid a \le x },
$$
$$
(a, b] = \lrset{ x \in \mathbb{R} \mid a < x \le b },
\quad (a, b) = \lrset{ x \in \mathbb{R} \mid a < x < b },
\quad (a, +\infty) = \lrset{ x \in \mathbb{R} \mid a < x },
$$
$$
(-\infty, b] = \lrset{ x \in \mathbb{R} \mid x \le b },
\quad (-\infty, b) = \lrset{ x \in \mathbb{R} \mid x < b },
\quad (-\infty, +\infty) = \mathbb{R}.
$$
このうち$[a, b], [a, +\infty), (-\infty, b], (-\infty, +\infty)$を\emph{閉区間}という。
また、$(a, b), (a, +\infty), (-\infty, b), (-\infty, +\infty)$を\emph{開区間}という。
\end{definition}

実数にも有理数同様、四則演算つまり加法、減法、乗法、除法と反数、逆数が定義できるがその説明をするには上限の概念が必要になる。
ここではそれらが定義できたとしてその性質を紹介する。
\begin{enumerate}
\item
(加法の結合法則)任意の実数$a, b, c$に対して$(a+b)+c = a+(b+c)$。
\item
(加法の交換法則)任意の実数$a, b$に対して$a+b = b+a$。
\item
(乗法の結合法則)任意の実数$a, b, c$に対して$(a\cdot b)\cdot c = a\cdot (b\cdot c)$。
\item
(乗法の交換法則)任意の実数$a, b$に対して$a\cdot b = b\cdot a$。
\item
(分配法則)任意の実数$a, b, c$に対して$a\cdot (b+c) = a\cdot b+a\cdot c$、$(a+b)\cdot c = a\cdot c+b\cdot c$。
\item
(零元)$0$は零元である。つまり任意の実数$a$に対して$a+0 = 0+a = a$と$a\cdot 0 = 0\cdot a = 0$が成り立つ。
\item
(反数)任意の実数$a$に対して$a+x = x+a = 0$が成り立つような実数$x = -a$がただ一つ存在する。
\item
(単位元)$1$は単位元である。つまり任意の実数$a$に対して$a\cdot 1 = 1\cdot a = a$が成り立つ。
\item
(逆数)任意の実数$a$に対して$a\cdot x = x\cdot a = 1$が成り立つような実数$x = a^{-1}$がただ一つ存在する。
\end{enumerate}

\section{複素数}

\section{床・天井}

通常の除法とは別に\emph{剰余付き除法}は実数と整数を結びつける意味で重要である。
なお、この部分の内容は後の内容を使うが自然数の大小関係が定義された段階で最大元・最小元(定義\ref{d_max})、有界性(定義\ref{d_bdd})、命題\ref{t_maxdisc}が得られるので問題ない。

まず、自然数$n$と$0$でない自然数$m$を考え、自然数の集合$A = \lrset{ x \in \mathbb{N} \mid x m \le n }$を定義する。
$m \ge 1$なので、この時任意の$x \in A$に対して$x \le x m \le n$より$A$は上に有界で命題\ref{t_maxdisc}から$A$の最大元$q$が存在する。
ここで$q m \le n$であるため、$q m+r = n$となる自然数$r$が存在することがいえる。
さらに$r \ge m$だとすると、$r = m+x$となる自然数$x$が存在するため$(q+1) m+x = n$つまり$q+1 \in A$となり$q$が$A$の最大元であることに矛盾する。
したがって$r < m$である。
この時の$q$を剰余付き除法の\emph{商}、$r$を\emph{剰余}という。
剰余付き除法の重要な点は$q$, $r$は$n = q m+r$, $r < m$を満たす自然数であるということで、一意に存在する。

自然数$n$と$0$でない自然数$m$に対して、剰余付き除法の商を$q$、剰余を$r$ ($0 \le r < m$)とする。
この時、整数$+n$の正の整数$m$で割ったときの剰余付き除法の商を$+q$、剰余を$r$で定義する。
また、整数$-n$の正の整数$m$で割ったとき、$r = 0$のとき剰余付き除法の商を$-q$、剰余を$r = 0$で、$r > 0$のとき商を$-q-1$、剰余を$m-r$で定義する。
以上により整数$a$を正の整数$b$で割ったときの剰余付き除法の商$q$と剰余$r$が定義され、どの場合でも$q$, $r$は$a = q b+r$ ($0 \le r < b$)を満たす整数である。

有理数での剰余付き除法を定義するために次の命題を準備する。

\begin{proposition}
\label{t_int_rational}
有理数$a$に対して、$x \le a \le y$を満たす整数$x$, $y$が存在する。
\end{proposition}

\begin{proof}
有理数$a$は整数$n$と正の整数$m$を使って$a = n/m$と表現できる。
ここで$n$を$m$で割った剰余付き除法の商を$q$、剰余を$r$とすると$q$, $r$は$n = q m+r$, $0 \le r < m$を満たす整数なので、$q m \le n < q m+m$つまり$q \le a < q+1$である。
よって命題が示された。
\end{proof}

有理数$a$を正の有理数$b$で割ったときの剰余付き除法について、上の命題から有理数$\frac{a}{b}$以下の整数の集合は上に有界で空でないので最大元$q$が存在しそれを商とする。
また、$r = a-q b$を剰余とする。
これにより、有理数$a$を正の有理数$b$で割ったときの剰余付き除法の商$q$と剰余$r$が定義され、$a = q b+r$ ($0 \le r < b$)を満たす整数$q$と有理数$r$である。

実数での剰余付き除法を定義するために次の命題を準備する。

\begin{proposition}
\label{t_int_real}
実数$a$に対して、$x \le a \le y$を満たす整数$x$, $y$が存在する。
\end{proposition}

\begin{proof}
有理数の稠密性(命題\ref{t_dense_rational})より、$a-1 < x' < a < y' < a+1$を満たす有理数$x'$, $y'$が存在する。
さらに命題\ref{t_int_rational}より$x \le x' < a < y' \le y$を満たす整数$x$, $y$が存在する。
よって命題が示された。
\end{proof}

実数$a$を正の実数$b$で割ったときの剰余付き除法について、上の命題から実数$\frac{a}{b}$以下の整数の集合は上に有界で空でないので最大元$q$が存在しそれを商とする。
また、$r = a-q b$を剰余とする。
これにより、実数$a$を正の実数$b$で割ったときの剰余付き除法の商$q$と剰余$r$が定義され、$a = q b+r$ ($0 \le r < b$)を満たす整数$q$と実数$r$である。

実数$a$に対して命題\ref{t_int_real}から$a$以下の整数の集合に最大元が存在しそれを$a$の\emph{床}といい$\lrfloor{a}$と表す。
つまり、
$$
\lrfloor{a} = \max\lrset{ n \in \mathbb{Z} \mid n \le a }
$$
である。
$a$の床$\lrfloor{a}$は$a$を$1$で剰余付き除法をした時の商に他ならない。
$a$の床$\lrfloor{a} \in \mathbb{Z}$を$a$の\emph{整数部分}とも言い、残りの部分$a-\lrfloor{a} \in [0, 1)$を$a$の\emph{小数部分}という。

同様にして、実数$a$に対して命題\ref{t_int_real}から$a$以上の整数の集合に最小元が存在しそれを$a$の\emph{天井}といい$\lrceil{a}$と表す。
つまり、
$$
\lrceil{a} = \min\lrset{ n \in \mathbb{Z} \mid n \ge a }
$$
である。

\section{累乗}

複素数$a$と自然数$n = 0, 1, 2, 3, \cdots$に対して、
$1$に$a$を$n$回かけて得られる複素数を\emph{$a$の$n$乗}といい、$a^n$と表す。
$0^n$は常に$1$である。
$a^b$という表現は今後様々な$a$, $b$の場合に拡張されるが、定義される数の範囲が変わることがあることに注意する。
$a$が自然数の時は$a^n$も自然数で、$a$が整数の時は$a^n$も整数、$a$が有理数の時は$a^n$も有理数、$a$が実数の時は$a^n$も実数である。

\section{総和と総乗}

$M$, $N$を$M \le N$を満たす自然数として$N-M+1$個($N-M+1$は$1$以上の自然数になる)の数$a_M, \cdots, a_N$の和について考える。
$0$に$a_M, \cdots, a_N$を順番に足して得られる数を$a_M, \cdots, a_N$の\emph{総和}といい、$\sum_{n = M}^N a_n$と表す。
つまり、
$$
\sum_{n = M}^N a_n = a_M+\cdots+a_N
$$
である。
実際には加法の結合法則と交換法則より足す順番は関係ない。

\begin{proposition}
$a_M, \cdots a_N$が$b_M, \cdots, b_{N+1}$を使って$a_n = b_{n+1}-b_n$と表されたとすると
$$
\sum_{n = M}^N a_n = \sum_{n = M}^N(b_{n+1}-b_n) = b_{N+1}-b_M
$$
である。
\end{proposition}

いくつかの特別な数の和が計算できることが知られている。

\begin{proposition}[和の公式]
\begin{itemize}
\item
$\sum_{n = 1}^N 1 = N$.
\item
$\sum_{n = 1}^N n = \frac{1}{2}N(N+1)$.
\item
$\sum_{n = 1}^N n^2 = \frac{1}{6}N(N+1)(2 N+1)$.
\item
$\sum_{n = 1}^N n^3 = \frac{1}{4}N^2(N+1)^2$.
\item
$c \ne 1$の時、$\sum_{n = M}^N c^n = \frac{c^{N+1}-c^M}{c-1}$。
\end{itemize}
\end{proposition}

また、後で使う次の式もここで紹介する。

\begin{proposition}[部分和分]
\label{t_sum_part}
数$a_M, \cdots a_{N+1}$が$b_M, \cdots, b_{N+1}$に対して
$$
\sum_{n = M}^N a_n(b_{n+1}-b_n) = a_{N+1}b_{N+1}-a_M b_M-\sum_{n = M}^N (a_{n+1}-a_n)b_{n+1}
$$
が成り立つ。
\end{proposition}

この式は定積分でいうところの部分積分に相当するものである。

\begin{proof}
和の中身を計算すると
$$
a_n(b_{n+1}-b_n)+(a_{n+1}-a_n)b_{n+1} = a_{n+1}b_{n+1}-a_n b_n
$$
なので、
$$
\sum_{n = M}^N a_n(b_{n+1}-b_n)+\sum_{n = M}^N (a_{n+1}-a_n)b_{n+1} = \sum_{n = M}^N (a_{n+1}b_{n+1}-a_n b_n) = a_{N+1}b_{N+1}-a_M b_M.
$$
よって、ほしい式が得られる。
\end{proof}

$M$, $N$を$M \le N$を満たす自然数として$N-M+1$個の数$a_M, \cdots, a_N$の積について考える。
$1$に$a_M, \cdots, a_N$を順番にかけて得られる数を$a_M, \cdots, a_N$の\emph{総乗}といい、$\prod_{n = M}^N a_n$と表す。
つまり、
$$
\prod_{n = M}^N a_n = a_M\times\cdots\times a_N
$$
である。
実際には乗法の結合法則と交換法則よりかける順番は関係ない。

$n$を$1$以上の自然数として$1, \cdots, n$の総乗を$n$の\emph{階乗}といい、$n!$と表す。
$n = 0$の時は$n! = 1$と約束する。
この時、任意の自然数$n$に対して$(n+1)! = (n+1)\times n!$が成立する。

\section{二項定理}

自然数$n$, $k$を$k \le n$をみたすつまり$n-k$が自然数となるとする。
この時、次の数$\binom{n}{k}$を\emph{二項係数}という。
$$
\binom{n}{k} = \frac{n!}{k!(n-k)!}.
$$
定義に除法が用いられているが二項係数は常に自然数であることが知られている。

\begin{example}
$\binom{n}{0} = 1$, $\binom{n}{1} = n$, $\binom{n}{2} = \frac{1}{2}n(n-1)$.
\end{example}

二項係数は二項定理を表現するのに用いられる。

\begin{proposition}[二項定理]
複素数$a$, $b$と自然数$n$に対して、以下が成立する。
$$
(a+b)^n = \sum_{k = 0}^n \binom{n}{k}a^k b^{n-k} = \binom{n}{0}b^n+\binom{n}{1}a b^{n-1}+\cdots+\binom{n}{n-1}a^{n-1}b+\binom{n}{n}a^n.
$$
\end{proposition}

\section{不等式}

不等式の基礎として次の$2$乗に関する性質がある。

\begin{proposition}
任意の実数$a$に対して$a^2 \ge 0$が成立する。
等号成立条件は$a = 0$である、つまり$a^2 = 0$ならば$a = 0$である。
\end{proposition}

ここから次を得る。

\begin{proposition}
任意の実数$a$, $b$に対して次が成立する。
$$
\frac{a^2+b^2}{2} \ge a b.
$$
等号成立条件は$a = b$である。
\end{proposition}

\begin{remark}
平方根はまだ定義されていないが、$a$, $b$が平方根の場合を考えると、任意の$a \ge 0$, $b \ge 0$に対して、
$$
\frac{a+b}{2} \ge \sqrt{a b}
$$
が成立する。
ここで、$\frac{a+b}{2}$を$a$, $b$の\emph{算術平均}または\emph{相加平均}、$\sqrt{a b}$を$a$, $b$の\emph{幾何平均}または\emph{相乗平均}という。
そのため今回の不等式は算術平均は幾何平均以上であることを保証し、\emph{算術平均・幾何平均の不等式}という。
\end{remark}

\begin{proof}
$$
\frac{a^2+b^2}{2}-a b
= \frac{1}{2}(a^2+b^2-2 a b)
= \frac{1}{2}(a-b)^2 \ge 0.
$$
\end{proof}


\chapter{極値}

\section{最大・最小}

$A$を実数の集合つまり$\mathbb{R}$の部分集合とする。
$A$の最大元とは$A$の元の中で最も大きいもので、$A$の最小元とは$A$の元の中で最も小さいものであり正確には以下で定義される。

\begin{definition}[集合の最大元と最小元]
\label{d_max}
$A$を実数の集合とする。
\begin{itemize}
\item
次の二つの条件を満たす実数$a$が存在したら$a$は$A$の\emph{最大元}といい、$\max A$と表す。
\begin{enumerate}
\item
任意の$x \in A$に対して$x \le a$が成り立つ。
\item
$a \in A$。
\end{enumerate}
\item
次の二つの条件を満たす実数$a$が存在したら$a$は$A$の\emph{最小元}といい、$\min A$と表す。
\begin{enumerate}
\item
任意の$x \in A$に対して$x \ge a$が成り立つ。
\item
$a \in A$。
\end{enumerate}
\end{itemize}
\end{definition}

重要な事実として最大元、最小元は一意である。

\begin{proposition}
$A$を実数の集合とする。
\begin{itemize}
\item
$A$の最大元は存在したら一つしかない、つまり$a$, $b$を$A$の最大元とすると$a = b$である。
\item
$A$の最小元は存在したら一つしかない、つまり$a$, $b$を$A$の最小元とすると$a = b$である。
\end{itemize}
\end{proposition}

\begin{proof}
最大元について証明する。
$a, b$は$A$の最大元より最大元の定義の二つ目の条件から$a, b \in A$である。
ここで$a$は$A$の最大元より最大元の定義の一つ目の条件で$x = b$とすることで、$b \le a$である。
同様に$b$は$A$の最大元より最大元の定義の一つ目の条件で$x = a$とすることで、$a \le b$である。
これら二つの大小関係から$a = b$が従う。

最小元についても同様なので証明は省略する。
\end{proof}

最大元、最小元は一意ではあるが存在するとは限らない。
特に空集合には(最大元、最小元の定義の二つ目の条件が問題になって)最大元、最小元が存在しない。

一方で存在が保証される場合がいくつかある。
最大・最小の基礎は有限集合である。

\begin{proposition}
\label{t_maxfin}
$A$を実数の空でない有限集合とする。
このとき$A$には最大元と最小元が存在する。
\end{proposition}

\begin{example}
$A$が2つの数$a$, $b$からなる$2$点集合$A = \lrset{ a, b }$の時を考える。
このとき$a \le b$の場合と$b \le a$の場合がある。
\begin{itemize}
\item
$a \le b$の場合は$b$が$A$の最大元で$a$が$A$の最小元である。
\item
$b \le a$の場合は$a$が$A$の最大元で$b$が$A$の最小元である。
\end{itemize}
よって、どちらの場合でも最大元と最小元が存在する。
なお、両方の場合は$a = b$なのでそもそも$A$は$1$点集合であるが、最大元と最小元がともに$a = b$となっているともいえる。
\end{example}

\begin{proof}[命題\ref{t_maxfin}の証明]
証明は元の個数に関する数学的帰納法による。
\end{proof}

いくつかの区間(特に閉区間)はその定義から最大・最小がすぐ求まる。

\begin{example}
区間$[a, b], (a, b], (-\infty, b]$の最大元は$b$であり、区間$[a, b], [a, b), [a, +\infty)$の最小元は$a$である。
これら以外の区間には最大元、最小元は存在しない。
\end{example}

\section{絶対値}

実数$a$に対して$a$と$-a$のうち大きい方の値を$a$の\emph{絶対値}といい$|a|$で表す。
絶対値は$a$と$-a$の大小関係つまり$a$の符号によって特徴づけられ常に非負の値をとる。
つまり、
$$
|a| = \max\lrset{ a, -a } =
\begin{cases}
a & \text{($a \ge 0$),} \\
-a & \text{($a \le 0$)}
\end{cases}
$$
である。
絶対値は符号を取り除いた数値といえるし、$0$からの距離ともみなせる。

絶対値において重要な事項は次の三角不等式である。
$|-a| = |a|$であることに注意する。

\begin{proposition}[三角不等式]
実数$a$, $b$に対して次の不等式が成り立つ。
$$
||a|-|b|| \le |a+b| \le |a|+|b|,
\quad ||a|-|b|| \le |a-b| \le |a|+|b|.
$$
\end{proposition}

4つの不等式があるが実質的には2つ目の$|a|+|b| \le |a+b|$が重要で、他の3つはこの不等式からすぐに得られる。

\begin{proof}
絶対値の定義より$|a| \ge a, -a$なので$-|a| \le a \le |a|$であり、同様に$-|b| \le b \le |b|$である。
よって、辺々足せば
$$
-|a|-|b| \le a+b \le |a|+|b|.
$$
特に$a+b \le |a|+|b|$かつ$-a-b \le |a|+|b|$なので、
$$
|a+b| = \max\lrset{ a+b, -a-b } \le |a|+|b|
$$
を得る。

あとはこの式の$a$を$a+b$、$b$を$-b$とすれば$|a| = |(a+b)+(-b)| \le |a+b|+|-b| = |a+b|+|b|$より$|a|-|b| \le |a+b|$。
$a$を$a+b$、$b$を$-a$とすれば$|b| = |(a+b)+(-a)| \le |a+b|+|-a| = |a+b|+|a|$より$|b|-|a| \le |a+b|$。
よって、$||a|-|b|| = \max\lrset{ |a|-|b|, -|a|+|b| } \le |a+b|$である。

さらに$b$を$-b$とすれば、$||a|+|-b|| \le |a-b| \le |a|+|-b|$でこれは$||a|-|b|| \le |a-b| \le |a|+|b|$に他ならない。
\end{proof}

\section{上限・下限}

実数の集合$A$に対して最大元や最小元は存在しないがそれに準じる数がある場合がある。
例えば開区間$(0, 1)$を考えると$1$は最大元の$0$は最小元の一つ目の条件を満たしているが開区間の定義からそれぞれ$(0, 1)$の元ではないため最大元や最小元ではない。
このようなぎりぎりまで近い数を取れるがそれ自身は元でないため最大元・最小元にならないような数のために上限・下限の概念がある。

\begin{definition}[集合の上限と下限]
$A$を実数の集合とする。
\begin{itemize}
\item
任意の$x \in A$に対して$x \le m$が成り立つような実数$m$は$A$の\emph{上界}であるという。
$A$の上界からなる集合に最小元がある時、その最小元を$A$の\emph{上限}といい$\sup A$と表す。
\item
任意の$x \in A$に対して$x \ge m$が成り立つような実数$m$は$A$の\emph{下界}であるという。
$A$の下界からなる集合に最大元がある時、その最大元を$A$の\emph{下限}といい$\inf A$と表す。
\end{itemize}
\end{definition}

上限・下限が何らかの集合の最小元・最大元で定義されているので、上限・下限も一意である。

上限・下限の同値な言い換えを一つ紹介する。

\begin{proposition}
$A$を実数の集合、$a$を実数とする。
\begin{itemize}
\item
$a$が$A$の上限であることは次の二つの条件が成り立つことと同値である。
\begin{enumerate}
\item
任意の$x \in A$に対して$x \le a$が成り立つ。
\item
任意の実数$b < a$に対してある$x \in A$が存在して$b < x$が成り立つ。
\end{enumerate}
\item
$a$が$A$の下限であることは次の二つの条件が成り立つことと同値である。
\begin{enumerate}
\item
任意の$x \in A$に対して$x \ge a$が成り立つ。
\item
任意の実数$b > a$に対してある$x \in A$が存在して$b > x$が成り立つ。
\end{enumerate}
\end{itemize}
\end{proposition}

\begin{proof}
上限について証明する。
$a$が$A$の上限である時、$a$は$A$の上界なので一つ目の条件が満たされる。
二つ目の条件について背理法でもし成り立たないとするとある実数$b < a$が存在して任意の$x \in A$に対して$b \ge x$である。
したがって$b$も$A$の上界だが、$b < a$より$a$が$A$の上界からなる集合の最小元だったことに矛盾する。
よって、二つ目の条件が成立する。
逆に、二つの条件が満たされるとき、まず一つ目から$a$は$A$の上界である。
$b$を$A$の上界とするとき$b \ge a$を示すが背理法で$b < a$だったとする。
二つ目の条件から$b < x$となる$x \in A$が存在するが、これは$b$が$A$の上界でなく矛盾である。
したがって$A$の上界$b$は$b \ge a$を満たすので、$a$は$A$の上限である。
以上より同値性が成り立つ。
下限についても同様に証明できる。
\end{proof}

上限・下限が存在する条件について考えるために上限・下限が存在しない場合を二つ紹介する。
一つは実数の集合$A$に対して上界・下界が存在しない場合で$A$の上界・下界からなる集合が空になるので上限・下限が存在しない。
この場合を排除するために次の有界性の概念を導入する。

\begin{definition}[集合の有界性]
\label{d_bdd}
$A$を実数の集合とする。
\begin{itemize}
\item
ある実数$M$が存在して任意の$x \in A$に対して$x \le M$が成り立つような時$A$は\emph{上に有界}であるという。
\item
ある実数$M$が存在して任意の$x \in A$に対して$x \ge M$が成り立つような時$A$は\emph{下に有界}であるという。
\item
上に有界かつ下に有界の時$A$は\emph{有界}であるという。
\end{itemize}
\end{definition}

\begin{remark}
実数の集合$A$が有界であることとある実数$M$が存在して任意の$x \in A$に対して$|x| \le M$が成り立つことは同値である。
\end{remark}

上限・下限が存在しないもう一つの場合は$A$自身が空集合の場合で、この時$A$の上界からなる集合と下界からなる集合は実数全体になってしまい最大元・最小元は存在しない。
しかしながらこれら二つの場合を除けば常に実数の集合には上限・下限が存在する。
より正確には以下が成立する。

\begin{proposition}[実数の連続性]
$A$を実数の集合とする。
\begin{itemize}
\item
もし$A$が上に有界かつ空でないならば$A$の上限が存在する。
\item
もし$A$が下に有界かつ空でないならば$A$の下限が存在する。
\item
もし$A$が有界かつ空でないならば$A$の上限と下限が存在する。
\end{itemize}
\end{proposition}

また、$A$が上に有界でない場合$\sup A = +\infty$、下に有界でない場合$\inf A = -\infty$、$A$が空集合の場合$\inf A = +\infty$, $\sup A = -\infty$と書くことにする。
この時、任意の実数の集合$A$に対して$\sup A$, $\inf A$が定義できることになる。

\begin{example}
区間の上限について
$\sup [a, b] = \sup [a, b) = \sup (a, b] = \sup (a, b) = \sup (-\infty, b] = \sup (-\infty, b) = b$, $\sup [a, +\infty) = \sup (a, +\infty) = \sup (-\infty, +\infty) = +\infty$である。
下限については
$\inf [a, b] = \inf [a, b) = \inf [a, +\infty) = \inf (a, b] = \inf (a, b) = \inf (a, +\infty) = a$, $\inf (-\infty, b] = \inf (-\infty, b) = \inf (-\infty, +\infty) = -\infty$である。
\end{example}

冒頭に述べたように最大元・最小元と上限・下限は密接な関係がある。

\begin{theorem}
$A$を実数の集合とする。
\begin{itemize}
\item
もし$A$が最大元を持つならばそれは$A$の上限でもある。
\item
もし$A$が最小元を持つならばそれは$A$の下限でもある。
\end{itemize}
\end{theorem}

\begin{proof}
\end{proof}

最後に実数の連続性に関連して整数や自然数の集合では上限・下限どころか最大元・最小元が存在することを示す。

\begin{proposition}
\label{t_maxdisc}
次が成立する。
\begin{itemize}
\item
もし整数の集合$A$が上に有界かつ空でないならば$A$の最大元が存在する。
\item
もし整数の集合$A$が下に有界かつ空でないならば$A$の最小元が存在する。
\item
もし自然数の集合$A$が上に有界かつ空でないならば$A$の最大元が存在する。
\item
もし自然数の集合$A$が空でないならば$A$の最小元が存在する。
\end{itemize}
\end{proposition}

\begin{proof}
まず、最初の内容を示す。
$A$は空でないので整数$N \in A$が存在する。
また、$A$は上に有界なので任意の$x \in A$に対して$x \le M$となるような整数$M$が存在する。
ここで$A$のうち$N$以上$M$以下のものに制限して得られる集合$A\cap\lrset{ N, \cdots, M }$は高々$M-N+1$個の元からなる有限集合なので命題\ref{t_maxfin}より最大元$a$が整数として存在する。
この時、任意の$x \in A\cap\lrset{ N, \cdots, M }$に対して$x \le a$かつ$a \in A$かつ$N \le a \le M$である。
ここで$x \in A$に対して$x \le N$の時は$x \le N \le a$で、$N \le x \le M$の時は上より$x \le a$で、$M$の取り方から$x > M$は起こりえないので、どの場合でも$x \le a$である。
加えて$a \in A$だったので整数$a$が$A$の最大元である。

二つ目の内容も同様にして証明できる。
三つ目の内容は自然数の集合は整数の集合とみなすことができるので、一つ目から直ちに従う。
四つ目の内容は自然数の集合は整数の集合とみなすことができることと、自然数の集合は常にどの元も$0$以上より下に有界なので、二つ目から直ちに従う。
\end{proof}

\section{関数の最大・最小}

関数の最大・最小はその像の最大・最小として定義される。

\begin{definition}[関数の最大・最小・上限・下限]
$f$を集合$X$上の実数値関数とする。
\begin{itemize}
\item
像$f(X)$の最大を関数$f$の\emph{最大値}といい、
$$
\max f, \quad \max_X f, \quad \max_x f(x), \quad \max_{x \in X}f(x)
$$
などと表す。
\item
像$f(X)$の上限を関数$f$の\emph{上限}といい、
$$
\sup f, \quad \sup_X f, \quad \sup_x f(x), \quad \sup_{x \in X}f(x)
$$
などと表す。
\item
像$f(X)$の最小を関数$f$の\emph{最小値}といい、
$$
\min f, \quad \min_X f, \quad \min_x f(x), \quad \min_{x \in X}f(x)
$$
などと表す。
\item
像$f(X)$の下限を関数$f$の\emph{下限}といい、
$$
\inf f, \quad \inf_X f, \quad \inf_x f(x), \quad \inf_{x \in X}f(x)
$$
などと表す。
\item
$f(a) = \max f$となる$a \in X$のことを$f$の\emph{最大点}といい、$f$は点$a$で最大という。
つまり、$a$を$f$の最大点とすると任意の$x \in X$に対して$f(x) \le f(a)$が成り立つ。
\item
$f(a) = \min f$となる$a \in X$のことを$f$の\emph{最小点}といい、$f$は点$a$で最小という。
つまり、$a$を$f$の最小点とすると任意の$x \in X$に対して$f(x) \ge f(a)$が成り立つ。
\end{itemize}
\end{definition}


\chapter{極限}

\section{$\epsilon$論法}

ある実数が正でないつまり$0$以下であることを直接示すのが難しい時でも、条件を緩和していくらでも小さくできる正の数$\varepsilon$よりも小さいことが言えたらその数は正でないと結論付けられる。
次の定理はそのような論法の正当性を保証する。

\begin{theorem}
\label{t_ep_real}
$a$, $b$を実数とするとき次が成立する。
\begin{itemize}
\item
任意の正の数$\varepsilon$に対して$a \le b+\varepsilon$ならば$a \le b$である。
\item
任意の正の数$\varepsilon$に対して$a \ge b-\varepsilon$ならば$a \ge b$である。
\end{itemize}
\end{theorem}

\begin{remark}
この定理は$a < b+\varepsilon$の時でも使えるが結論は$a \le b$であって、その場合でも$a < b$は得られない。
例えば$a = b = 0$の時、$0 < 0+\varepsilon$だが$0 < 0$ではない。
あくまでも等号付き不等号が示される。
\end{remark}

\begin{proof}
最初の内容のみ証明する。
背理法でもし$a-b > 0$とすると$\varepsilon = \frac{a-b}{2} > 0$と取ることができ、
$$
a-b \le \varepsilon = \frac{a-b}{2}
$$
よって、$a-b > 0$で両辺を割ることで$2 \le 1$を得て、これは$1 < 2$であることに矛盾する。
したがって$a-b \le 0$つまり$a \le b$である。

二つ目は$a$, $b$を入れ替えて同じ議論をすればよい。
最後は仮定が$b-\varepsilon \le a \le b+\varepsilon$なので、上二つの内容が使えて$a \le b$かつ$a \ge b$つまり$a = b$である。
\end{proof}

特に次のバリエーションがある。

\begin{theorem}
\label{t_ep_number}
$a$, $b$を数とするとき次が成立する。
任意の正の数$\varepsilon$に対して$|a-b| \le \varepsilon$ならば$a = b$である。
\end{theorem}

\begin{proof}
$|a-b|$は非負の実数なので定理\ref{t_ep_real}より$|a-b| \le 0$で、$|a-b| = 0$がわかりこれは$a = b$である。
\end{proof}

\section{数列の極限}

数列とは数の並びのこと全般をいうが、ここでは順序を持ち無限に続くものをいう。
そのため順序を表現するため添え字として自然数を用いる。
つまり\emph{\index{すうれつ@数列}}とは各自然数$n \in \mathbb{N}$に対して一つの数$a_n$を対応させたものであり$(a_n)_{n \in \mathbb{N}}$や一部省略して$(a_n)_n$あるいは$(a_n)$と表す。
また、最初の自然数(または整数)を明示したり変更したい場合は$(a_n)_{n = 1}^\infty$などと表す。
数列の中でも数が入るクラスに応じて、各$a_n$が実数ならば$(a_n)$は実数列、有理数ならば有理数列、整数ならば整数列、自然数ならば自然数列、複素数ならば複素数列という。
このテキストでは特に指定がなければ数列といったら実数列である。

\begin{example}
$(2 n)_{n = 0}^\infty$, $(5^n)_{n = 0}^\infty$, $(3)_{n = 0}^\infty$, $(\frac{1}{n})_{n = 1}^\infty$などは(実)数列である(最初の3つは自然数列でもある)。
\end{example}

数列の極限はいわゆる$\varepsilon$論法によって次のように定義される。

\begin{definition}[数列の極限]
数列$(a_n)_{n \in \mathbb{N}}$が$n \to \infty$で数$a$に\emph{収束する}あるいは$a$が$(a_n)_{n \in \mathbb{N}}$の$n \to \infty$での\emph{極限である}とは
\begin{quote}
任意の$\varepsilon > 0$に対して、ある自然数$N$が存在して、$N$以上のすべての自然数$n$に対して$|a_n-a| < \varepsilon$が成り立つ
\end{quote}
ことを言う。
なお、この条件を論理記号で書くと以下になる。
$$
\forall \varepsilon > 0, \exists N \in \mathbb{N}, \forall n \in \mathbb{N}, n \ge N \implies |a_n-a| < \varepsilon.
$$
また、この時$a_n \to a$ ($n \to \infty$)や$\lim_{n \to \infty}a_n = a$と書くが、``$\to \infty$''の部分や``$n \to \infty$''はしばしば省略される。

数列$(a_n)_{n \in \mathbb{N}}$が何らかの数に収束する時この数列は\emph{収束する}といい、そうでない時\emph{発散する}という。
\end{definition}

\begin{example}
$a$を定数として定数列$(a)_{n \in \mathbb{N}}$の極限は$a$である。
\end{example}

この定義の中で「ある自然数$N$が存在して、$N$以上のすべての自然数$n$に対して」の部分はしばしば「\emph{十分大きな}すべての自然数$n$に対して」と略し、論理記号では$\forall n \gg 1$と表す。
つまり、$a_n \to a$は
$$
\forall \varepsilon > 0, \forall n \gg 1, |a_n-a| < \varepsilon.
$$
とも表される。
このテキストでは記号$\gg$は極限に関係する上記の意味でのみ使い、$\forall$を伴う。
$a \gg b$のようには用いないので注意する。
また、$\forall \varepsilon$と$\forall n \gg 1$は一般には交換できないので注意する。

なお、$|a_n-a| = ||a_n-a|-0|$であることから数列$(a_n)$が$a$に収束することは実数列$(|a_n-a|)_n$が$0$に収束することと同値である。

それはさておき、極限は存在したら一意である。

\begin{proposition}
数列$(a_n)$が収束したらその極限は一つしかない、つまり$(a_n)$が$a$に収束し$b$にも収束したとすると$a = b$である。
\end{proposition}

\begin{proof}
$\varepsilon > 0$を任意の正の数として、$|a-b| < \varepsilon$を示す。
ここで、$(a_n)$が$a$に収束することからある自然数$N_a$が存在して$N_a$以上の任意の自然数$n$に対して$|a_n-a| < \frac{\varepsilon}{2}$であり、
$(a_n)$が$b$に収束することからある自然数$N_b$が存在して$N_b$以上の任意の自然数$n$に対して$|a_n-b| < \frac{\varepsilon}{2}$である。
よって$N = \max\lrset{ N_a, N_b }$とおくと、三角不等式より
$$
|a-b| = |a-a_N+a_N-b| \le |a-a_N|+|a_N-b| < \varepsilon.
$$
ここで定理\ref{t_ep_number}より$a = b$と結論付けられる。
\end{proof}

発散の中でも次の二つは後でわかるように収束に近い性質を持っている。

\begin{definition}[数列の発散]
$(a_n)_{n \in \mathbb{N}}$を実数列とする。
\begin{itemize}
\item
$(a_n)_{n \in \mathbb{N}}$が$n \to \infty$で\emph{$+\infty$に発散する}とは
\begin{quote}
任意の$M > 0$に対して、十分大きいすべての自然数$n$に対して$a_n > +M$が成り立つ
\end{quote}
ことを言う。
この条件を論理記号で書くと以下になる。
$$
\forall M > 0, \forall n \gg 1, a_n > +M.
$$
また、この時$a_n \to +\infty$ ($n \to \infty$)や$\lim_{n \to \infty}a_n = +\infty$と書く。
\item
$(a_n)_{n \in \mathbb{N}}$が$n \to \infty$で\emph{$-\infty$に発散する}とは
\begin{quote}
任意の$M > 0$に対して、十分大きいすべての自然数$n$に対して$a_n < -M$が成り立つ
\end{quote}
ことを言う。
この条件を論理記号で書くと以下になる。
$$
\forall M > 0, \forall n \gg 1, a_n < -M.
$$
また、この時$a_n \to -\infty$ ($n \to \infty$)や$\lim_{n \to \infty}a_n = -\infty$と書く。
\item
$(a_n)_{n \in \mathbb{N}}$が$n \to \infty$で収束せず$+\infty$にも$-\infty$にも発散しない時、$(a_n)$は\emph{振動する}という。

\end{itemize}
\end{definition}

極限の具体的な計算において基礎となるのは次の事実である。

\begin{proposition}
\label{t_limid}
数列$(n)_{n \in \mathbb{N}}$は$n \to \infty$で$+\infty$に発散する。
\end{proposition}

このことは自明のように思えるが、数列の添え字の$n$は自然数なのに対して数列の項は実数であり、自然数と実数の差を埋める必要がある。

\begin{proof}[命題\ref{t_limid}の証明]
$M > 0$とする。
天井を使って整数$N = \lceil M \rceil+1$とおくことで、$N$は$N > M > 0$を満たす自然数である。
この時$N$以上のすべての自然数$n$に対して$n \ge N > M$より、数列$(n)_{n \in \mathbb{N}}$は$n \to \infty$で$+\infty$に発散することがわかる。
\end{proof}

以下に述べる収束する数列の性質は今後の議論をするうえで非常に重要である。

\begin{proposition}
\label{t_conv_bdd}
収束する数列は有界である。
\end{proposition}

\begin{proof}
数列$(a_n)_{n \in \mathbb{N}}$が実数$a$に収束するとする。
この時、収束することの定義で$\varepsilon = 1$とすることで、ある自然数$N$が存在して$N$以上のすべての自然数$n$に対して$|a_n-a| < 1$であるので、三角不等式から
$$
|a_n| = |a_n-a+a| \le |a_n-a|+|a| \le |a|+1
$$
が成り立つ。
$N$未満の自然数は$n = 0, 1, 2, 3, \cdots, N-1$の有限個なので、
$$
M = \max\{ |a_0|, |a_1|, |a_2|, |a_3|, \cdots, |a_{N-1}|, |a|+1 \}
$$
とおくと$M$は実数で、任意の自然数$n$に対して$|a_n| \le M$である。
つまり$(a_n)$は有界であることが示された。
\end{proof}

\begin{proposition}[極限と大小関係]
\label{t_limord}
$(a_n)$, $(b_n)$を実数列とし、それぞれ実数$a$, $b$に収束するとする。
ここでもし各$n$に対して$a_n \le b_n$が成り立つならば$a \le b$である。
\end{proposition}

\begin{remark}
この命題は$a_n < b_n$の時でも適用できるが結論は$a \le b$であって、その場合でも$a < b$は得られない。
\end{remark}

\begin{proof}
$\varepsilon > 0$を任意の正の数として、$a < b+\varepsilon$を示す。
ここで、$(a_n)$が$a$に収束することからある自然数$N_a$が存在して$N_a$以上の任意の自然数$n$に対して$|a_n-a| < \frac{\varepsilon}{2}$であり、
$(a_n)$が$b$に収束することからある自然数$N_b$が存在して$N_b$以上の任意の自然数$n$に対して$|a_n-b| < \frac{\varepsilon}{2}$である。
ここで$N = \max\{ N_a, N_b \}$とおくと、仮定$a_n \le b_n$より
$$
a \le a-a_N+a_N < a_N+\frac{\varepsilon}{2} \le b_N+\frac{\varepsilon}{2} \le b_N-b+b+\frac{\varepsilon}{2} < b+\varepsilon.
$$
よって定理\ref{t_ep_real}より$a \le b$と結論付けられる。
\end{proof}

\begin{proposition}[はさみうちの原理]
\label{t_squeeze}
$(a_n)$, $(b_n)$を実数列とし、それぞれ実数$a$, $b$に収束するとする。
さらに$(c_n)$を実数列として各$n$に対して$a_n \le c_n \le b_n$が成り立ち$a = b$の時、$(c_n)$は$a = b$に収束する。
\end{proposition}

この命題をもちいれば極限値の決定に加えて収束すること自体いえて便利である。

\begin{proof}
$\varepsilon > 0$を任意の正の数とする。
ここで、$(a_n)$が$a$に収束することから十分大きなすべての自然数$n$に対して$|a_n-a| < \varepsilon$であり、
$(b_n)$が$b$に収束することから十分大きなすべての自然数$n$に対して$|b_n-b| < \varepsilon$である。
ここで十分大きなすべての自然数$n$に対して、仮定から
$$
c_n-a \ge a_n-a > -\varepsilon, \quad c_n-b \le b_n-b < \varepsilon
$$
でまとめると
$$
|c_n-a| = |c_n-b| < \varepsilon
$$
なので、$c_n$は$a = b$に収束する。
\end{proof}

\begin{corollary}
\label{t_squeeze2}
$(a_n)$を数列で$a$を数、$(b_n)$を$0$に収束する実数列とする。
ここでもし各$n$に対して$|a_n-a| \le b_n$が成り立つならば、$(a_n)$は$a$に収束する。
\end{corollary}

\begin{proof}
$0 \le |a_n-a| \le b_n$であり、$(0)$, $(b_n)$は$0$に収束することから、はさみうちの原理(命題\ref{t_squeeze})より、実数列$(|a_n-a|)_n$は$0$に収束する。
したがって$(a_n)$は$a$に収束することが示された。
\end{proof}

\begin{proposition}[極限と四則演算]
$(a_n)$, $(b_n)$をそれぞれ数$a$, $b$に収束する数列とする。
$c$を定数とする。
\begin{enumerate}
\item
$a_n+b_n \to a+b$.
\item
$c a_n \to c a$.
\item
$a_n-b_n \to a-b$.
\item
$a_n b_n \to a b$.
\item
$b \ne 0$の時、$\frac{a_n}{b_n} \to \frac{a}{b}$.
\item
$|a_n| \to |a|$.
\item
$\max\lrset{ a_n, b_n } \to \max\lrset{ a, b }$.
\item
$\min\lrset{ a_n, b_n } \to \min\lrset{ a, b }$.
\end{enumerate}
\end{proposition}

\begin{remark}
証明でも触れるが、最後の商の極限について$b_n \to b \ne 0$だと十分大きいすべての$n$について$b_n \ne 0$であり、そのような$n$に対して$\frac{a_n}{b_n}$を考えることに注意する。
\end{remark}

これらのうち極限の定義に立ち返って証明されるのは最初の二つで、他ははさみうちの原理を使うことで見通しよく証明できる。
また、二つ目は四つ目の特殊な場合($b_n = c$)であるが、先に二つ目を示すことでその内容を使って四つ目を示す。

\begin{proof}
\begin{enumerate}
\item
$\varepsilon > 0$を固定する。
$a_n \to a$より極限の定義で$\varepsilon$として$\frac{\varepsilon}{2} > 0$として取れば、十分大きな自然数$n$に対して$|a_n-a| < \frac{\varepsilon}{2}$である。
$b_n \to b$より同様にして、十分大きな自然数$n$に対して$|b_n-b| < \frac{\varepsilon}{2}$である。
したがって、三角不等式より十分大きな自然数$n$に対して、
$$
|(a_n+b_n)-(a+b)| = |a_n-a+b_n-b| \le |a_n-a|+|b_n-b| < \frac{\varepsilon}{2}+\frac{\varepsilon}{2} = \varepsilon.
$$
よって$a_n+b_n \to a+b$である。
\item
$c = 0$の時、$c a_n = 0$, $c a = 0$なので$c a_n \to c a$であるため、$c \ne 0$の場合を考えればよい。
$a_n \to a$より極限の定義で$\varepsilon$として$\frac{\varepsilon}{|c|} > 0$として取れば、十分大きな自然数$n$に対して$|a_n-a| < \frac{\varepsilon}{|c|}$である。
よって、十分大きな自然数$n$に対して、
$$
|c a_n-c a| = |c(a_n-a)| \le |c||a_n-a| < |c|\frac{\varepsilon}{|c|} = \varepsilon.
$$
よって、$c a_n \to c a$である。
\item
二つ目で$a_n$として$b_n$、$c$として$-1$として適用すれば$-b_n \to -b$である。
よって、一つ目より
$$
a_n-b_n = a_n+(-b_n) \to a+(-b) = a-b
$$
である。
\item
計算すると
$$
|a_n b_n-a b| = |a_n(b_n-b)+(a_n-a)b| \le |a_n||b_n-b|+|b||a_n-a|.
$$
ここで、$(a_n)$は収束するので命題\ref{t_conv_bdd}より有界であるからある実数$M$が存在してすべての自然数$n$に対して$|a_n-a| \le M$とできることから、
$$
|a_n b_n-a b| \le M|b_n-b|+|b||a_n-a|.
$$
したがって一つ目と二つ目を使うことで右辺は$0$に収束することがわかる。
よって、系\ref{t_squeeze2}より$a_n b_n \to a b$が結論付けられる。
\item
まず、$b_n \to b \ne 0$より極限の定義で$\varepsilon$として$\frac{|b|}{2} > 0$として取れば、十分大きな自然数$n$に対して$|b_n-b| < \frac{|b|}{2}$である。
したがって三角不等式より、
$$
|b_n| = |(b_n-b)+b| \ge ||b|-|b_n-b|| \ge |b|-|b_n-b| > |b|-\frac{|b|}{2} = \frac{|b|}{2} > 0.
$$
よって特に$b_n \ne 0$である。
また、計算すると
$$
\lrabs{\frac{a_n}{b_n}-\frac{a}{b}} = \frac{|a_n b-a b_n|}{|b||b_n|} = \frac{|(a_n-a)b-a(b_n-b)|}{|b||b_n|} \le \frac{|b||a_n-a|+|a||b_n-b|}{|b||b_n|} \le 2\frac{|b||a_n-a|+|a||b_n-b|}{|b|^2}.
$$
ここで一つ目と二つ目を使うことで右辺は$0$に収束することがわかる。
よって、系\ref{t_squeeze2}より$\frac{a_n}{b_n} \to \frac{a}{b}$が結論付けられる。
\item
三角不等式より、
$$
||a_n|-|a|| \le |a_n-a|
$$
なので、系\ref{t_squeeze2}より$|a_n| \to |a|$がわかる。
\item
一般に実数$x$, $y$に対して
$$
\max\lrset{ x, y } = \frac{x+y}{2}+\frac{|y-x|}{2}
$$
が成り立つことに注意する。
よって
$$
\max\lrset{ a_n, b_n } = \frac{a_n+b_n}{2}+\frac{|b_n-a_n|}{2} \to \frac{a+b}{2}+\frac{|b-a|}{2} = \max\lrset{ a, b }
$$
である。
\item
一般に実数$x$, $y$に対して
$$
\min\lrset{ x, y } = \frac{x+y}{2}-\frac{|y-x|}{2}
$$
が成り立つことに注意する。
よって
$$
\min\lrset{ a_n, b_n } = \frac{a_n+b_n}{2}-\frac{|b_n-a_n|}{2} \to \frac{a+b}{2}-\frac{|b-a|}{2} = \min\lrset{ a, b }
$$
である。
\end{enumerate}
\end{proof}

これらの計算は$\pm \infty$に発散する場合でもある程度拡張できる。
以下では特に重要な場合を示す。

\begin{proposition}
$(a_n)$を実数列とする。
\begin{enumerate}
\item
$a_n \to +\infty$ならば$a_n^{-1} \to 0$。
\item
$a_n \to -\infty$ならば$a_n^{-1} \to 0$。
\item
十分大きい任意の自然数$n$に対して$a_n > 0$かつ$a_n \to 0$ならば$a_n^{-1} \to +\infty$。
\item
十分大きい任意の自然数$n$に対して$a_n < 0$かつ$a_n \to 0$ならば$a_n^{-1} \to -\infty$。
\end{enumerate}
\end{proposition}

\begin{proof}
一つ目を示す。
まず、$a_n \to +\infty$の定義で$M = 1$とすることで十分大きい自然数$n$に対して$a_n > 1 > 0$であることに注意する。
$\varepsilon > 0$を固定する。
この時、$a_n \to +\infty$の定義で$M = \varepsilon^{-1} > 0$とすることで十分大きい自然数$n$に対して$a_n > M$である。
よって、十分大きい自然数$n$に対して
$$
|a_n^{-1}-0| = a_n^{-1} \le M^{-1} = \varepsilon.
$$
よって、$a_n^{-1} \to 0$である。
二つ目は同様にして証明できる。

三つ目を示す。
$M > 0$を固定する。
この時、$a_n \to 0$の定義で$\varepsilon = M^{-1} > 0$とすることで十分大きい自然数$n$に対して$|a_n| < \varepsilon$である。
ここでもう一つの仮定より十分大きい自然数$n$に対して$|a_n| = a_n < \varepsilon$であるから、
$$
a_n^{-1} > \varepsilon^{-1} = M.
$$
よって、$a_n^{-1} \to +\infty$である。
四つ目も同様にして証明できる。
\end{proof}

$\infty-\infty$, $0\times\infty$, $\frac{\infty}{\infty}$, $\frac{0}{0}$の形を\emph{不定形}といい、不定形以外の極限は$\pm \infty$を含んでいても基本的に正当化される。

以上を用いると様々な数列の極限を求めることができる。

\begin{example}
\begin{itemize}
\item
$\frac{1}{n} \to 0$.
\item
$\frac{n+2}{3 n+4} = \frac{1+2 n^{-1}}{3+4 n^{-1}} \to \frac{1+2\cdot 0}{3+4\cdot 0} = \frac{1}{3}$.
\end{itemize}
\end{example}

同じ$+\infty$に発散する数列でも発散の速度は
$$
n, \quad n^2, \quad n^3, \quad 2^n, \quad 3^n, \quad n!, \quad n^n
$$
の順に早くなる。
つまり、左にあるものをそれより右にあるもので割って得られる数列は$0$に収束する。

\begin{proposition}[等比数列の極限]
\label{t_limgeomseq}
$c$を実数とする。
\begin{enumerate}
\item
$c > 1$の時、$c^n \to +\infty$ ($n \to \infty$)。
\item
$c = 1$の時、$c^n \to 1$ ($n \to \infty$)。
\item
$-1 < c < 1$の時、$c^n \to 0$ ($n \to \infty$)。
\item
$c \le -1$の時、$(c^n)$は$n \to \infty$で発散する($\pm \infty$に発散もしない)。
\end{enumerate}
\end{proposition}

\begin{proof}
\begin{enumerate}
\item
$c > 1$の時、$c = 1+a$とする$a > 0$が取れ、二項定理より$n \ge 1$において
$$
c^n = (1+a)^n = 1+n a+\binom{n}{2}a^2+\cdots+\binom{n}{n}a^n \ge n a.
$$
ここで最右辺は$n \to \infty$で$+\infty$に発散するので、$c^n \to \infty$である。
\end{enumerate}
\end{proof}

\begin{proposition}
数列$(a_n)$に対して、実数列$(|\frac{a_{n+1}}{a_n}|)$がある$1$より小さい実数$c$に収束したとする。
この時、$a_n \to 0$である。
\end{proposition}

\begin{proof}
$0 \le c < 1$より$d = \frac{1+c}{2}$とおくと$0 \le d < 1$で$d$を公比とする等比数列に帰着させたい。
そこで$\varepsilon = d-c = \frac{1-c}{2} > 0$とおくと、十分大きいすべての自然数$n$に対して
$$
\lrabs{\frac{a_{n+1}}{a_n}} \le c+\varepsilon = d
$$
つまり、$|a_{n+1}| \le d|a_n|$である。
したがってこれを繰り返し用いることで、
$$
|a_n| \le d|a_{n-1}| \le d^2|a_{n-2}| \le \cdots \le d^n|a_0|.
$$
よって、$0 \le d < 1$より命題\ref{t_limgeomseq}から最右辺は$0$に収束するので、はさみうちの原理から$a_n \to 0$である。
\end{proof}

\section{単調数列の極限}

一般に数列は収束するとは限らないがそれを保証する単調収束定理がある。
その中で重要になる概念が次の単調性と有界性である。

\begin{definition}[数列の単調性]
$(a_n)_{n \in \mathbb{N}}$を実数列とする。
\begin{itemize}
\item
任意の自然数$n$に対して$a_n \le a_{n+1}$が成り立つ、つまり
$$
a_0 \le a_1 \le a_2 \le a_3 \le \cdots
$$
のとき、$(a_n)$は\emph{単調増加}という。
\item
任意の自然数$n$に対して$a_n \ge a_{n+1}$が成り立つ、つまり
$$
a_0 \ge a_1 \ge a_2 \ge a_3 \ge \cdots
$$
のとき、$(a_n)$は\emph{単調減少}という。
\end{itemize}
\end{definition}

\begin{example}
$(n)_{n = 0}^\infty$は単調増加であり、$(\frac{1}{n})_{n = 1}^\infty$は単調減少である。
\end{example}

\begin{definition}[数列の有界性]
$(a_n)_{n \in \mathbb{N}}$を実数列とする。
\begin{itemize}
\item
集合$\{ a_n \mid n \in \mathbb{N} \}$が上に有界の時、すなわちある実数$M$が存在して任意の$n \in \mathbb{N}$に対して$a_n \le M$の時、$(a_n)$は上に有界という。
\item
集合$\{ a_n \mid n \in \mathbb{N} \}$が下に有界の時、すなわちある実数$M$が存在して任意の$n \in \mathbb{N}$に対して$a_n \ge M$の時、$(a_n)$は下に有界という。
\item
上に有界かつ下に有界の時、$(a_n)$は有界であるという。
\end{itemize}
\end{definition}

\begin{remark}
実数列$(a_n)_{n \in \mathbb{N}}$が有界であることとある実数$M$が存在して任意の$n \in \mathbb{N}$に対して$|a_n| \le M$が成り立つことは同値である。
\end{remark}

\begin{theorem}[単調収束定理]
\label{t_mono_conv}
$(a_n)_{n \in \mathbb{N}}$を実数列とする。
\begin{itemize}
\item
$(a_n)$が単調増加かつ上に有界のとき、$(a_n)$は収束する。
\item
$(a_n)$が単調減少かつ下に有界のとき、$(a_n)$は収束する。
\end{itemize}
\end{theorem}

\begin{remark}
証明を見ればわかるが実は極限は、$(a_n)$が単調増加かつ上に有界のとき上限$\sup_n a_n$で、$(a_n)$が単調減少かつ下に有界のとき下限$\inf_n a_n$である。
また、有界性の仮定は極限を実数の範囲に収めるためで、$(a_n)$が単調増加かつ上に有界でないときは$+\infty$に発散し、$(a_n)$が単調減少かつ下に有界でないときは$-\infty$に発散することがわかる。
そのため$\pm \infty$を考慮に入れれば、$(a_n)$が単調増加のとき$a_n \to \sup_n a_n$で、$(a_n)$が単調減少のとき$a_n \to \inf_n a_n$とまとめて書くことができる。
\end{remark}

\begin{proof}
$(a_n)$が単調増加かつ上に有界の場合のみ証明する。
まず、$(a_n)$が上に有界であることから集合$\{ a_n \mid n \in \mathbb{N} \}$には上限$\sup_n a_n \in \mathbb{R}$が存在する。
$\varepsilon > 0$を固定すると、上限の同値な条件から$a_N > \sup_n a_n-\varepsilon$を満たす自然数$N$が存在する。
ここで、$(a_n)$が単調増加であることから$N$以上の自然数$n$に対して$a_n \ge a_N > \sup_n a_n-\varepsilon$、また$a_n \le \sup_n a_n$であるから、$|a_n-\sup_n a_n| < \varepsilon$である。
したがって、$a_n \to \sup_n a_n$が得られ、特に$(a_n)$は収束する。
\end{proof}

有理数からなる集合でも上限・下限が無理数になりうるように、有理数列でも無理数に収束することがありうる。
そのため有理数は極限操作について閉じていないが、実数だとある程度閉じていることになる。
実は自然数列の極限は存在したら自然数であり、整数列では整数である。
極限操作について閉じることは自然数や整数では成り立っていたものが有理数で崩れてしまい、実数になってまた成り立つようになっている。

このような数列の収束を保証する定理を用いると、次のような手法で数列の極限を求めることができる。

つまり、関数$F$が与えられたときに\emph{漸化式}
\begin{equation}
\label{e_rec}
a_{n+1} = F(a_n)
\end{equation}
を考える。
これは\emph{初期値}$a_0$が与えられたら、$n = 0, 1, 2, 3, \cdots$に対して$a_{n+1} = f(a_n)$で$a_0, a_1, a_2, a_3, \cdots$が(数学的帰納法から)求まり、数列$(a_n)_{n = 0}^\infty$が定まりその極限を考える。
ここでもし$(a_n)$が何らかの数$a$に収束したとする。
後で示す命題から$(a_{n+1})$も$a$に収束することと、さらにもし$(F(a_n))$が$F(a)$に収束したとする(これを保証するのが関数の連続性である)と
\begin{equation}
\label{e_limiteq}
a = F(a)
\end{equation}
が成り立つ。
式\eqref{e_limiteq}は$a$についての方程式とみなすと漸化式\eqref{e_rec}の\emph{極限方程式}と呼ばれ、これに解がただ一つ存在する場合は極限$a$はその解以外にあり得ないので極限が求まる。

つまり、単調収束定理などで収束を保証して極限方程式の解の一意存在をいえば極限が求まる。
これによって、一般項を求めることが難しい問題でも極限を計算することができる。

\begin{example}
$c \ge 0$として、漸化式
$$
a_{n+1} = \frac{1}{2}\lrparen*{a_n+\frac{c}{a_n}}
$$
を考える。
これは$x^2-c$の根つまり$\sqrt{c}$をニュートン法によって求める際に現れる漸化式である。

$a_0 \ge \sqrt{c}$とするとき、算術平均・幾何平均の不等式から$n = 0, 1, 2, 3, \cdots$に対して
$$
a_{n+1} = \frac{1}{2}\lrparen*{a_n+\frac{c}{a_n}} \ge \sqrt{a_n\cdot\frac{c}{a_n}} = \sqrt{c}
$$
で特に$(a_n)$は下に有界である。
さらに
$$
a_{n+1}-a_n = \frac{1}{2}\lrparen*{\frac{c}{a_n}-a_n} = \frac{c-a_n^2}{2 a_n} \le 0
$$
より、$(a_n)$は単調減少である。
よって単調収束定理より$(a_n)$は何らかの実数$a$に収束するが極限方程式
$$
a = \frac{1}{2}\lrparen*{a+\frac{c}{a}}
$$
を満たしこれを解くと$a = \pm\sqrt{c}$である。
ここで$a_n \ge \sqrt{c}$なので命題\ref{t_limord}より$a \ge \sqrt{c}$から$a = \sqrt{c}$のみ適。
よって、$a_n \to \sqrt{c}$ ($n \to \infty$)が結論付けられる。

この問題は極限方程式の解が$a = \sqrt{c}$であることから$a_n \to \sqrt{c}$を示すために、
$$
|a_{n+1}-\sqrt{c}| = \frac{(a_n-\sqrt{c})^2}{2|a_n|} \le \frac{1}{2}|a_n-\sqrt{c}|
$$
と計算することで
$$
|a_n-\sqrt{c}| \le \frac{1}{2^n}|a_0-\sqrt{c}| \to 0
$$
より同じ結論を得ることができる。
後者の解法の方が収束の速さまでわかる一方、前者の方が計算が楽といえる。
\end{example}

\begin{example}
漸化式
$$
a_{n+1} = \sqrt{2+a_n}
$$
を考える。
これは$\sqrt{2+\sqrt{2+\sqrt{2+\sqrt{2}}}}$のような面白い項からなる数列を作る。

$0 \le a_n \le 2$ならば$a_{n+1} = \sqrt{2+a_n} \ge \sqrt{2} > 0$かつ$a_{n+1} = \sqrt{2+a_n} \le \sqrt{2+2} = 2$なので、$0 \le a_0 \le 2$とすると任意の$n = 0, 1, 2, 3, \cdots$に対して$0 \le a_n \le 2$で特に$(a_n)$は上に有界である。
さらに
$$
a_{n+1}^2-a_n^2 = 2+a_n-a_n^2 = -(a_n+1)(a_n-2) \ge 0
$$
より$a_n \ge 0$に注意して、$(a_n)$は単調増加である。
よって単調収束定理より$(a_n)$は何らかの実数$a$に収束するが極限方程式
$$
a = \sqrt{2+a}
$$
を満たしこれを解くと$a = 2$である。
よって、$a_n \to 2$ ($n \to \infty$)が結論付けられる。

この問題も前回の問題同様に、極限方程式の解が$a = 2$であることから$a_n \to 2$を示すために、
$$
|a_{n+1}-2| = |\sqrt{2+a_n}-2| = \frac{|2+a_n-4|}{\sqrt{2+a_n}+2} \le \frac{1}{2}|a_n-2|
$$
と計算することで
$$
|a_n-2| \le \frac{1}{2^n}|a_0-2| \to 0
$$
より同じ結論を得ることができる。
\end{example}

\section{部分列の極限}

$\pm 1$を交互に取る数列$((-1)^n)_{n = 0}^\infty$はそれ自身は極限の定義に照らし合わせるとどのような数にも収束しない(発散する)が、
奇数番目の項や偶数番目の項に制限すれば$\pm 1$に収束するので、それらを極限のようなものとして扱いたい場合がある。
そのような場合に重要になる概念が部分列である。

\begin{definition}[部分列]
$(a_n)_{n \in \mathbb{N}}$を数列とする。
ここで、$(n_k)_{k \in \mathbb{N}}$を自然数列であって、各$k$に対して$n_k < n_{k+1}$が成り立つ、つまり
$$
n_0 < n_1 < n_2 < n_3 < \cdots
$$
のとき、$(a_n)$のうち項を$n = n_k$に制限して得られる数列$(a_{n_k})_{k \in \mathbb{N}}$を$(a_n)$の\emph{部分列}という。

部分列$(a_{n_k})_{k \in \mathbb{N}}$が何らかの数$a$に収束する時、もとの数列$(a_n)$は$a$に\emph{部分列収束}するあるいは$a$が$(a_n)$の\emph{部分列極限}であるという。
\end{definition}

\begin{remark}
\begin{itemize}
\item
$n_k = k$として取ると$(a_{n_k})_k = (a_k)_k$なので、$(a_n)$自身は$(a_n)$の部分列であり、特に部分列は常に存在する。
\item
部分列極限は部分列の取り方によって変わるので一意とは限らない。
また、部分列は収束するとは限らないので存在とも限らない。
\end{itemize}
\end{remark}

\begin{example}
\begin{itemize}
\item
数列$((-1)^n)_{n = 0}^\infty$に対して$n$が偶数に制限すると$((-1)^{2 k})_{k = 0}^\infty = (+1)_{k = 0}^\infty$で$+1$に部分列収束し、$n$が奇数に制限すると$((-1)^{2 k+1})_{k = 0}^\infty = (-1)_{k = 0}^\infty$で$-1$に部分列収束することがわかる。
\item
実は数列$(\sin n)_{n = 0}^\infty$の部分列極限全体からなる集合は閉区間$[-1, 1]$になる(三角関数の周期$2\pi$が無理数であることに注意する)。
\end{itemize}
\end{example}

もとの数列が収束するならば部分列は同じ極限に収束する。

\begin{proposition}
数列$(a_n)_{n \in \mathbb{N}}$が数$a$に収束するとすると、任意の部分列$(a_{n_k})_{k \in \mathbb{N}}$も$a$に収束する。
% 特に部分列極限は一意に存在する。
\end{proposition}

\begin{proof}
定義から$n_k \ge k$がわかることに注意する(特に$n_k \to \infty$ ($k \to \infty$)である)。
$\varepsilon > 0$に対して、$a_n \to a$より任意の$N$以上の自然数$n$に対して$|a_n-a| < \varepsilon$となるような自然数$N$が存在する。
ここで$K = N$とおくと任意の$K$以上の自然数$k$に対して、$n_k \ge k \ge N$なので、$|a_{n_k}-a| < \varepsilon$である。
よって、$a_{n_k} \to \infty$である。
\end{proof}

これを用いれば$a_n \to a$のとき$a_{n+1} \to a$などがいえる。

部分列極限の中でも最大のものと最小のものはそれぞれ上極限、下極限と呼ばれる。

\begin{definition}[上極限・下極限]
$(a_n)_{n \in \mathbb{N}}$を実数列とする。
\begin{itemize}
\item
上限・下限を使って定義される次の数を$(a_n)$の$n \to \infty$での\emph{上極限}という。
$$
\limsup_{n \to \infty}a_n = \inf_{k \in \mathbb{N}}\sup\{ a_n \mid n \in \mathbb{N}, n \ge k \}.
$$
\item
上限・下限を使って定義される次の数を$(a_n)$の$n \to \infty$での\emph{下極限}という。
$$
\liminf_{n \to \infty}a_n = \sup_{k \in \mathbb{N}}\inf\{ a_n \mid n \in \mathbb{N}, n \ge k \}.
$$
\end{itemize}
\end{definition}

上極限・下極限は上限・下限で定義されるので($\pm \infty$を認めれば)必ず存在するという利点がある。

\begin{remark}
自然数$n$に対して$n$以上の自然数の集合は$n$が増えると小さくなるので$\sup\{ a_k \mid k \in \mathbb{N}, k \ge n \}$は単調減少である。
したがって上極限の定義の中の下限は極限にかえることができる。
下極限も同様で、つまり
$$
\limsup_{n \to \infty}a_n = \lim_{k \to \infty}\sup\{ a_n \mid n \in \mathbb{N}, n \ge k \},
\quad \liminf_{n \to \infty}a_n = \lim_{k \to \infty}\inf\{ a_n \mid n \in \mathbb{N}, n \ge k \}
$$
が成立する。
\end{remark}

また、定義から$\liminf a_n \le \limsup a_n$である。
この等号が成立するための条件が極限が存在することになる。

\begin{proposition}
$(a_n)_{n \in \mathbb{N}}$を数列とする。
\begin{itemize}
\item
$(a_n)$が収束するならば、$\liminf a_n = \limsup a_n = \lim a_n$である。
\item
$\liminf a_n = \limsup a_n$ならば$(a_n)$はその値に収束する。
\end{itemize}
\end{proposition}

\begin{remark}
この定理は$a_n \to \pm \infty$の場合も同様のものが成立する。
\end{remark}

\begin{proof}
前半はすぐわかるので証明を省略する。
後半について任意の自然数$k$に対して
$$
\inf\{ a_n \mid n \in \mathbb{N}, n \ge k \} \le a_k \le \sup\{ a_n \mid n \in \mathbb{N}, n \ge k \}
$$
であり最左辺と最右辺がともに$\liminf a_n = \limsup a_n$に収束するので、
はさみうちの原理より$a_n \to \liminf a_n = \limsup a_n$である。
\end{proof}

\begin{proposition}
\label{t_ord_slim}
$(a_n)_{n \in \mathbb{N}}$, $(b_n)_{n \in \mathbb{N}}$を十分大きい任意の自然数$n$に対して$a_n \le b_n$を満たす実数列とする。
\begin{itemize}
\item
$\limsup a_n \le \limsup b_n$.
\item
$\liminf a_n \le \liminf b_n$.
\end{itemize}
\end{proposition}

この命題は命題\ref{t_limord}を一般化したものになっている。

\begin{remark}
この命題は上極限・下極限が$\pm \infty$の場合でも成立する。
\end{remark}

\begin{proof}
上極限・下極限の定義から容易に証明できるので詳細は省略する。
\end{proof}

\begin{proposition}
\label{t_max_ulim}
$(a_n)_{n \in \mathbb{N}}$を実数列とする。
\begin{itemize}
\item
$a_{n_k} \to \limsup a_n$となる部分列$(a_{n_k})$が存在する。
\item
$a_{n_k} \to \liminf a_n$となる部分列$(a_{n_k})$が存在する。
\end{itemize}
\end{proposition}

\begin{remark}
この命題は上極限・下極限が$\pm \infty$の場合でも成立する。
\end{remark}

\begin{proof}
上極限のみ示す。
$\limsup_{n \to \infty}a_n$が$+\infty$でないつまり実数の場合を証明する。
各$k \in \mathbb{N}$に対して実数$b_k = \limsup a_n-\frac{1}{k+1}$を定めると$b_k \to \limsup a_n$であり、$b_k < \sup\{ a_n \mid n \in \mathbb{N}, n \ge k \}$であることに注意する。
よって$b_k < a_n$となる自然数$n \ge k$が存在するので、自然数列$n_k$を
$$
n_{k+1} = \min\{ n \in \mathbb{N} \mid b_{k+1} < a_n, n \ge n_k+1 > n_k \},
\quad n_0 = \min\{ n \in \mathbb{N} \mid b_0 < a_n \}
$$
で定義すると、$(a_{n_k})$は$(a_n)$の部分列であり、
$$
b_k < a_{n_k} \le \sup\{ a_n \mid n \in \mathbb{N}, n \ge k \}
$$
を満たす。
最左辺も再右辺も$\limsup_{n \to \infty}a_n$に収束するので、はさみうちの原理より部分列$(a_{n_k})$も$\limsup_{n \to \infty}a_n$に収束する。

$\limsup_{n \to \infty}a_n = +\infty$の場合は$b_k = k$として同じようなことをすればよい。
下極限については同様に証明できる(詳細省略)。
\end{proof}

特に次が従う。

\begin{theorem}[ボルツァーノ・ワイエルシュトラスの定理]
有界な実数列は収束する部分列を持つ。
\end{theorem}

\begin{proof}
実数列$(a_n)$の上極限$\limsup a_n$はその定義から$(a_n)$が有界だと$\pm \infty$でないことがわかる。
よって、命題\ref{t_max_ulim}を用いれば$(a_n)$は実数$\limsup a_n$に収束する部分列が存在することがわかるので、定理が示された。
\end{proof}

\begin{theorem}
$(a_n)_{n \in \mathbb{N}}$を数列、$a$を数とする。
もし$(a_n)$の任意の部分列$(a_{n_k})$が$a$に部分列収束する、つまり$a$に収束する部分列$(a_{n_{k_l}})$が存在するとき、$(a_n)$は$a$に収束する。
\end{theorem}

\begin{proof}
実数列$(|a_n-a|)_n$を考えると命題\ref{t_max_ulim}より上極限$\limsup_n|a_n-a|$に収束する部分列$(|a_{n_k}-a|)_k$が取れる。
仮定より$(a_{n_k})_k$は$a$に部分列収束するので$|a_{n_{k_l}}-a| \to 0$となる$(|a_{n_k}-a|)_k$の部分列$(|a_{n_{k_l}}-a|)_l$が存在する。
$(|a_{n_{k_l}}-a|)_l$は$\limsup_n|a_n-a|$にも収束すると言え、極限の一意性から$\limsup_n|a_n-a| = 0$である。
したがって、$0 \le \liminf_n|a_n-a| \le \limsup_n|a_n-a| = 0$より、$(|a_n-a|)_n$は$0$に収束し、つまり$(a_n)$は$a$に収束する。
\end{proof}

以上の内容をもとに漸化式で定まる数列の極限について一般論を述べる。

\begin{theorem}
$I$を実数の集合、$F = F(a, n)$, $G = G(a)$を関数とする。
\begin{itemize}
\item
$I$, $F$, $G$が
\begin{enumerate}
\item
任意の$a \in I$と$n \in \mathbb{N}$に対して$F(a, n) \in I$、つまり$F(I, \mathbb{N}) \subset I$。
\item
$I$は上に有界である。
\item
任意の$a \in I$と$n \in \mathbb{N}$に対して$a \ge F(a, n)$。
\item
任意の$(a_n)_{n \in \mathbb{N}}$ ($a_n \in I$)と$a \in I$に対して$a_n \to a \implies F(a_n, n) \to G(a)$。
\item
$a = G(a)$を満たす$a \in I$は一意である。
\end{enumerate}
を満たす時、漸化式$a_{n+1} = F(a_n, n)$によって任意の$a_0 \in I$に対して数列$(a_n)_{n \in \mathbb{N}}$ ($a_n \in I$)が定義でき、$(a_n)$は上に有界かつ単調増加より収束し極限を$a$とおくと、$a$は極限方程式$a = G(a)$の一意解であり$a = \max I$である。
\item
$I$, $F$, $G$が
\begin{enumerate}
\item
任意の$a \in I$と$n \in \mathbb{N}$に対して$F(a, n) \in I$、つまり$F(I, \mathbb{N}) \subset I$。
\item
$I$は下に有界である。
\item
任意の$a \in I$と$n \in \mathbb{N}$に対して$a \le F(a, n)$。
\item
任意の$(a_n)_{n \in \mathbb{N}}$ ($a_n \in I$)と$a \in I$に対して$a_n \to a \implies F(a_n, n) \to G(a)$。
\item
$a = G(a)$を満たす$a \in I$は一意である。
\end{enumerate}
を満たす時、漸化式$a_{n+1} = F(a_n, n)$によって任意の$a_0 \in I$に対して数列$(a_n)_{n \in \mathbb{N}}$ ($a_n \in I$)が定義でき、$(a_n)$は下に有界かつ単調減少より収束し極限を$a$とおくと、$a$は極限方程式$a = G(a)$の一意解であり$a = \min I$である。
\end{itemize}
\end{theorem}

\begin{remark}
三つ目の仮定は一般には外せない。
例えば$I = [-1, 1]$, $F(a, n) = G(a) = -a$とおくと、三番目以外の仮定を満たすが、
$a_n = (-1)^n a_0$となり$a_0 \ne 0$のとき$(a_n)$は発散する。
\end{remark}

数列の極限の最後に次のような極限に関する定理を紹介する。

\begin{theorem}
数列$(a_n)_{n = 1}^\infty$が$a$に収束するとする。
この時、
$$
\frac{a_1+\cdots+a_n}{n} \to a \quad (n \to \infty)
$$
が成り立つ。
\end{theorem}

\begin{remark}
$(a_n)_{n = 1}^\infty$が$\pm \infty$に発散する時でも同様の結果が成り立つ。
\end{remark}

この定理の証明にはある程度極限の定義に立ち戻る必要がある。

\begin{proof}
三角不等式より$n = 1, 2, 3, \cdots$に対して
$$
\lrabs{\frac{a_1+\cdots+a_n}{n}-a} = \frac{|(a_1-a)+\cdots+(a_n-a)|}{n} \le \frac{|a_1-a|+\cdots+|a_n-a|}{n}
$$
であることに注意する。
$\varepsilon > 0$に対して、$a_n \to a$であることから、$N$以上のすべての自然数$n$に対して$|a_n-a| < \varepsilon$となるような$N = 1, 2, 3, \cdots$が存在するので、$N$以上の自然数$n$に対して、
$$
\lrabs{\frac{a_1+\cdots+a_n}{n}-a}
< \frac{|a_1-a|+\cdots+|a_N-a|+(n-N)\varepsilon}{n}
= \varepsilon+\frac{|a_1-a|+\cdots+|a_N-a|-N\varepsilon}{n}
$$
である。
ここで$n \to \infty$とすると最右辺は$\varepsilon$に収束するので、
$$
\limsup_{n \to \infty}\lrabs{\frac{a_1+\cdots+a_n}{n}-a} \le \varepsilon
$$
で、$\varepsilon$は任意の正の数なので、ほしかった極限の式が得られる。
\end{proof}

\section{関数の極限}

この節では実数の集合$X$上で定義された実数値関数$f: X \to \mathbb{R}$を考える。
この時、$\varepsilon$論法により$f$の極限を定義する。
$f$の極限を定義するにあたって、実数$a$は任意の正の数$\delta$に対して、$0 < |x-a| < \delta$を満たす$x \in X$の存在を暗に仮定している。
このような$a$を$X$の\emph{極限点}という。
$X$として閉区間から有限個の点を取り除いて得られる集合を想定していてその場合は$a$はもとの閉区間の点なら何でもよいが、
$\{ \frac{1}{n} \mid n = 1, 2, 3, \cdots \}$のような集合の場合は極限点は$0$のみでありそれ以外の点では極限を考えられないことに注意する。

\begin{definition}[関数の極限]
$f$を実数の集合$X$上で定義された実数値関数、$a$を$X$の極限点、$l$を実数とする。

関数$f(x)$が$x \to a$で実数$l$に\emph{収束する}あるいは$l$が$f(x)$の$x \to a$での\emph{極限である}とは
\begin{quote}
任意の$\varepsilon > 0$に対して、ある$\delta > 0$が存在して、$0 < |x-a| < \delta$を満たす任意の$x \in X$に対して$|f(x)-l| < \varepsilon$が成り立つ
\end{quote}
ことをいう。
なお、この条件を論理記号で書くと以下になる。
$$
\forall \varepsilon > 0, \exists \delta > 0, \forall x \in X, 0 < |x-a| < \delta \implies |f(x)-l| < \varepsilon.
$$
この時、$f(x) \to l$ ($x \to a$)や$\lim_{x \to a}f(x) = l$と書く。

関数$f(x)$が$x \to a$で\emph{正の無限大に発散する}とは
\begin{quote}
任意の$M > 0$に対して、ある$\delta > 0$が存在して、$0 < |x-a| < \delta$を満たす任意の$x \in X$に対して$f(x) > +M$が成り立つ
\end{quote}
ことをいう。
なお、この条件を論理記号で書くと以下になる。
$$
\forall M > 0, \exists \delta > 0, \forall x \in X, 0 < |x-a| < \delta \implies f(x) > +M.
$$
この時、$f(x) \to +\infty$ ($x \to a$)や$\lim_{x \to a}f(x) = +\infty$と書く。

関数$f(x)$が$x \to a$で\emph{負の無限大に発散する}とは
\begin{quote}
任意の$M > 0$に対して、ある$\delta > 0$が存在して、$0 < |x-a| < \delta$を満たす任意の$x \in X$に対して$f(x) < -M$が成り立つ
\end{quote}
ことをいう。
なお、この条件を論理記号で書くと以下になる。
$$
\forall M > 0, \exists \delta > 0, \forall x \in X, 0 < |x-a| < \delta \implies f(x) < -M.
$$
この時、$f(x) \to -\infty$ ($x \to a$)や$\lim_{x \to a}f(x) = -\infty$と書く。
\end{definition}

ここで「ある$\delta > 0$が存在して、$0 < |x-a| < \delta$を満たす任意の$x \in X$に対して」の部分はしばしば「$a$でなく\emph{十分$a$に近い}任意の$x \in X$に対して」と略し、論理記号では$\forall x \in X, 0 < |x-a| \ll 1$と表す。

$f(x) \to l$は$|f(x)-l| \to 0$と同値である。

\begin{definition}[関数の無限大での極限]
$f$を実数の集合$X$上で定義された実数値関数、$l$を実数とする。

関数$f(x)$が$x \to +\infty$で実数$l$に収束するあるいは$l$が$f(x)$の$x \to +\infty$での極限であるとは
\begin{quote}
任意の$\varepsilon > 0$に対して、ある$N > 0$が存在して、$x > +N$を満たす任意の$x \in X$に対して$|f(x)-l| < \varepsilon$が成り立つ
\end{quote}
ことをいう。
なお、この条件を論理記号で書くと以下になる。
$$
\forall \varepsilon > 0, \exists N > 0, \forall x \in X, x > +N \implies |f(x)-l| < \varepsilon.
$$
この時、$f(x) \to l$ ($x \to +\infty$)や$\lim_{x \to +\infty}f(x) = l$と書く。

関数$f(x)$が$x \to +\infty$で正の無限大に発散するとは
\begin{quote}
任意の$M > 0$に対して、ある$N > 0$が存在して、$x > +N$を満たす任意の$x \in X$に対して$f(x) > +M$が成り立つ
\end{quote}
ことをいう。
なお、この条件を論理記号で書くと以下になる。
$$
\forall M > 0, \exists N > 0, \forall x \in X, x > +N \implies f(x) > +M.
$$
この時、$f(x) \to +\infty$ ($x \to +\infty$)や$\lim_{x \to +\infty}f(x) = +\infty$と書く。

関数$f(x)$が$x \to +\infty$で負の無限大に発散するとは
\begin{quote}
任意の$M > 0$に対して、ある$N > 0$が存在して、$x > +N$を満たす任意の$x \in X$に対して$f(x) < -M$が成り立つ
\end{quote}
ことをいう。
なお、この条件を論理記号で書くと以下になる。
$$
\forall M > 0, \exists N > 0, \forall x \in X, x > +N \implies f(x) < -M.
$$
この時、$f(x) \to -\infty$ ($x \to +\infty$)や$\lim_{x \to +\infty}f(x) = -\infty$と書く。

$x \to -\infty$での極限は同様にして定義される(詳細省略)。
\end{definition}

ここで「ある$N > 0$が存在して、$x > +N$を満たす任意の$x \in X$に対して」の部分はしばしば「\emph{十分大きい}任意の$x \in X$に対して」と略し、論理記号では$\forall x \in X, x \gg +0$と表す。
省略してしまったが「ある$N > 0$が存在して、$x < -N$を満たす任意の$x \in X$に対して」の部分はしばしば「\emph{十分小さい}任意の$x \in X$に対して」と略し、論理記号では$\forall x \in X, x \ll -0$と表す。

関数$f(x)$が$x \to a$で($a = \pm \infty$も含む)何らかの実数に収束する時$f(x)$は$x \to a$で\emph{収束する}といい、そうでない時は\emph{発散する}という。
収束せず正の無限大にも負の無限大にも発散しない時、$f(x)$は$x \to a$で\emph{振動する}という。

数列同様、極限は一意であることに注意する。

また、数列の時同様以下が成立する。

\begin{proposition}[はさみうちの原理]
関数$f$, $g$, $h$が、$a$でなく十分$a$に近い任意の$x \in X$に対して
$$
f(x) \le h(x) \le g(x)
$$
と$\lim_{x \to a}f(x) = \lim_{x \to a}g(x) = l$が成り立つとき$h(x) \to l$ ($x \to a$)である。

系として関数$f$に対して、$a$でなく十分$a$に近い任意の$x \in X$に対して
$$
|f(x)-l| \le g(x)
$$
で$x \to a$で$g(x) \to 0$となる実数値関数$g$が存在する時、$f(x) \to l$ ($x \to a$)である。
\end{proposition}

\begin{proposition}[極限と演算]
関数$f$, $g$が$x \to a$で($a = \pm \infty$も含む)$f(x) \to l$, $g(x) \to m$を満たしているとする。
\begin{enumerate}
\item
$f(x)+g(x) \to l+m$.
\item
$f(x)-g(x) \to l-m$.
\item
$f(x)g(x) \to l m$.
\item
$m \ne 0$の時$\frac{f(x)}{g(x)} \to \frac{l}{m}$.
\item
$|f(x)| \to |l|$.
\item
$\max\lrset{ f(x), g(x) } \to \max\lrset{ l, m }$.
\item
$\min\lrset{ f(x), g(x) } \to \min\lrset{ l, m }$.
\end{enumerate}
なお、これらは不定形以外なら$l, m = \pm \infty$でもよい。
\end{proposition}

\begin{definition}[片側極限]
\end{definition}

\begin{proposition}
\end{proposition}

\section{連続関数}

関数に対してそのグラフがつながっているという連続性は極限を使って定義される。

\begin{definition}[連続性]
$f$を実数の集合$X$上で定義された実数値関数として、$a$を$X$の極限点であり$X$は$a$を元として含むとする。
ここで$f(x)$が$x \to a$で$f(a)$に収束する、つまり
$$
\lim_{x \to a}f(x) = f(a)
$$
の時、関数$f(x)$は点$x = a$であるいは関数$f$は点$a$で\emph{連続}であるという。

同様に、$f$を実数の集合$X$上で定義された実数値関数として、$a$を$X$の右極限点であり$X$は$a$を元として含むとする。
ここで$f(x)$が$x \to a+$で$f(a)$に収束する、つまり
$$
\lim_{x \to a+}f(x) = f(a)
$$
の時、関数$f(x)$は点$x = a$であるいは関数$f$は点$a$で\emph{右連続}であるという。

$f$を実数の集合$X$上で定義された実数値関数として、$a$を$X$の左極限点であり$X$は$a$を元として含むとする。
ここで$f(x)$が$x \to a-$で$f(a)$に収束する、つまり
$$
\lim_{x \to a-}f(x) = f(a)
$$
の時、関数$f(x)$は点$x = a$であるいは関数$f$は点$a$で\emph{左連続}であるという。
\end{definition}

\begin{remark}
集合$X$上の関数$f(x)$が点$x = a$で連続であることをまとめなおすと
\begin{quote}
任意の$\varepsilon > 0$に対して、ある$\delta > 0$が存在して、$|x-a| < \delta$を満たす任意の$x \in X$に対して(十分$a$に近い任意の$x \in X$に対して)$|f(x)-f(x)| < \varepsilon$が成り立つ
\end{quote}
となる。
\end{remark}

関数の極限の性質を引き継いで連続性について以下が成立する。

\begin{proposition}[連続と演算]
関数$f(x)$, $g(x)$が点$x = a$で連続とする。
この時点$x$に対して以下の値を対応させる関数も点$x = a$で連続である。
\begin{enumerate}
\item
$f(x)+g(x)$.
\item
$f(x)-g(x)$.
\item
$f(x)g(x)$.
\item
$g(a) \ne 0$の時、$\frac{f(x)}{g(x)}$.
\item
$|f(x)|$.
\item
$\max\lrset{ f(x), g(x) }$.
\item
$\min\lrset{ f(x), g(x) }$.
\end{enumerate}
右連続・左連続についても同様のことが成り立つ。
\end{proposition}

\begin{proposition}
関数$f(x)$が両側極限点$x = a$で連続であるための必要十分条件は関数$f(x)$が点$x = a$で右連続かつ左連続であることである。
\end{proposition}

\begin{proposition}[連続と合成関数]
関数$f(x)$が点$x = a$で連続で関数$g(y)$が点$y = f(a)$で連続である時、
合成関数$g(f(x))$は点$x = a$で連続である。
\end{proposition}

次の命題は連続性が点の列を表す数列(点列)を使った表現で特徴づけられることを主張する。

\begin{proposition}[点列連続性]
関数$f(x)$が点$x = a$で連続であることは
\begin{quote}
$a$に収束する任意の点列$(a_n)$に対して$f(a_n) \to f(a)$が成り立つ
\end{quote}
ことと同値である。
\end{proposition}

\begin{proof}
$f(x)$が$x = a$で連続として、$a$に収束する点列$(a_n)$を考える。
$\varepsilon > 0$に対して、連続性からある$\delta > 0$が存在して$|x-a| < \delta$を満たす任意の点$x$に対して$|f(x)-f(a)| < \varepsilon$である。
さらにこの$\delta > 0$に対して、数列の収束から自然数$N$が存在して$n \ge N$を満たす任意の自然数$n$に対して$|a_n-a| < \delta$である。
よって、$x$として$a_n$を考えることで、十分大きいすべての$n$に対して$|f(a_n)-f(a)| < \varepsilon$、つまり$f(a_n) \to f(a)$を得る。

逆に$a$に収束する任意の点列$(a_n)$に対して$f(a_n) \to f(a)$のとき、$f$の連続性を示すが、背理法で連続でなかったとする。
この時、ある$\varepsilon > 0$が存在して、任意の$\delta > 0$に対して$|x-a| < \delta$かつ$|f(x)-f(a)| \ge \varepsilon$となる点$x$が存在する。
ここで$n = 1, 2, 3, \cdots$ごとに$\delta = \frac{1}{n}$を考えて上記の条件を満たす$x$が存在することからそのうち一つを$a_n$とおく。
そうして点列$(a_n)_{n = 0}^\infty$を定めるとき、$|a_n-a| < \frac{1}{n}$なので$(a_n)$は$a$に収束する。
一方で$|f(a_n)-f(a)| \ge \varepsilon$より$(f(a_n))$は$f(a)$に収束しないので仮定に矛盾する、よって$f(x)$は$x = a$で連続である。
\end{proof}

この命題の前半部分は特に重要で以降の議論で頻繁に用いられる。
後半部分はこのテキストではあまり用いられないが、連続性を証明するときに便利である。

集合上のすべての点で連続な関数を連続関数という。

\begin{definition}[連続関数]
$f$を実数の集合$X$上で定義された実数値関数として、$X$の点はすべて$X$の極限点であるとする。
ここで、すべての点$a \in X$で$f$が連続である時、関数$f$は集合$X$上の\emph{連続関数}であるという。
\end{definition}

\begin{remark}
「$X$の点はすべて$X$の極限点である」を満たす集合$X$として区間やいくつかの区間の和集合が挙げられる。
\end{remark}

\begin{example}
\label{t_ex_conti}
$c$を実数定数として、定数関数$f(x) = c$と恒等関数$f(x) = x$はともに$\mathbb{R}$上の連続関数である。
\end{example}

直ちに以下が成立する。

\begin{proposition}[連続関数と演算]
\label{t_conti_arith}
関数$f$, $g$を集合$X$上の連続関数とする。
この時$x \in X$に対して以下の値を対応させる関数も集合$X$で連続である。
\begin{enumerate}
\item
$f(x)+g(x)$.
\item
$f(x)-g(x)$.
\item
$f(x)g(x)$.
\item
すべての$x \in X$で$g(x) \ne 0$の時$\frac{f(x)}{g(x)}$.
\item
$|f(x)|$.
\item
$\max\lrset{ f(x), g(x) }$.
\item
$\min\lrset{ f(x), g(x) }$.
\end{enumerate}
\end{proposition}

\begin{proposition}[連続関数の合成]
関数$f(x)$が集合$X$で連続で関数$g(y)$が集合$Y \supset f(X)$で連続である時、
合成関数$g(f(x))$は集合$X$で連続である。
\end{proposition}

\section{中間値の定理、最大値最小値定理、逆関数の定理}

ここでは有界閉区間上の連続関数特有の定理として中間値の定理と最大値最小値定理を示す。

\begin{theorem}[中間値の定理]
関数$f$を有界閉区間$[a, b]$上の連続関数とする。
この時、任意の$\min\{ f(a), f(b) \} \le y \le \max\{ f(a), f(b) \}$に対して
$$
f(x) = y, \quad a \le x \le b
$$
となる点$x$が存在する。

また、$\min\{ f(a), f(b) \} < y < \max\{ f(a), f(b) \}$の時は$a < x < b$と取れる。
\end{theorem}

\begin{corollary}[不動点型の中間値の定理]
関数$f$を有界閉区間$[a, b]$からそこへの連続関数とする$f([a, b]) \subset [a, b]$。
この時、$x = f(x)$を満たす不動点$x \in [a, b]$が存在する。
\end{corollary}

\begin{theorem}[最大値最小値定理]
関数$f$を有界閉区間$[a, b]$上の連続関数とする。
この時、関数$f$は$[a, b]$上で最大と最小を持つ。
\end{theorem}

この二つの定理はともに区間を左と右に二等分し(中間値の定理ならば$x$、最大値最小値定理ならば最大点・最小点)が入る区間を選ぶという操作を繰り返し行い$1$点に縮小させるという論法で証明される。
次の補題はその議論を支えるものである。

\begin{lemma}[区間縮小法]
実数列$(a_n)$と$(b_n)$が各自然数$n$に対して$a_n \le a_{n+1} \le b_{n+1} \le b_n$と$|b_n-a_n| \to 0$を満たすとする。
この時$a_n \to x$, $b_n \to x$となる実数$x$が存在する。

特に$(a_n)$, $(b_n)$を、各自然数$n$に対して$c_n = \frac{a_n+b_n}{2}$として$a_{n+1} = a_n$, $b_{n+1} = c_n$または$a_{n+1} = c_n$, $b_{n+1} = b_n$が成立するように定義する時、$a_n \to x$, $b_n \to x$となる実数$x$が存在する。
\end{lemma}

\begin{proof}
仮定から$(a_n)$は単調増加かつ上に有界なので極限$a$が存在し、$(b_n)$は単調減少かつ下に有界なので極限$b$が存在する。
ここで$|b_n-a_n| \to |b-a|$であるが、もう一つの仮定より$a = b$であり、これを$x$とおけばよい。

後半部分は$(a_n)$, $(b_n)$の定義から$a_n \le a_{n+1} \le b_{n+1} \le b_n$と$|b_n-a_n| = \frac{1}{2^n}|b-a| \to 0$が言えるので$x$が存在することが示される。
\end{proof}

\begin{proof}[中間値の定理の証明]
$f(a) \le f(b)$の場合を示す。
$f(a) \le y \le f(b)$とする。
ここで$(a_n)_{n = 0}^\infty$, $(b_n)_{n = 0}^\infty$を次のように定める。
まず、$a_0 = a$, $b_0 = b$とする。
$n = 0, 1, 2, 3, \cdots$に対して$a_n$, $b_n$が定義できた時$c_n = \frac{a_n+b_n}{2}$として、
\begin{itemize}
\item
$f(c_n) \ge y$の時、$a_{n+1} = a_n$, $b_{n+1} = b_n$とおき、
\item
$f(c_n) < y$の時、$a_{n+1} = c_n$, $b_{n+1} = a_n$とおく。
\end{itemize}
この時、常に$f(a_n) \le y \le f(b_n)$であり、補題より$(a_n)$, $(b_n)$はともにある$a \le x \le b$に収束する。
よって$f$が連続関数であることから$f(x) \le y \le f(x)$より、ほしかった式$f(x) = y$を得る。

$f(a) \ge f(b)$の場合も同様に示される。
$\min\{ f(a), f(b) \} < y < \max\{ f(a), f(b) \}$の時は$x \ne a, b$なので$a < x < b$と取れる。
\end{proof}

\begin{remark}
この証明は方程式$f(x) = y$の解の構成の計算可能なアルゴリズムにもなっており、\emph{二分法}と呼ばれる。
\end{remark}

\begin{proof}[最大値最小値定理の証明]
最大を持つことを示す。
ここで$(a_n)_{n = 0}^\infty$, $(b_n)_{n = 0}^\infty$を次のように定める。
まず、$a_0 = a$, $b_0 = b$とする。
$n = 0, 1, 2, 3, \cdots$に対して$a_n$, $b_n$が定義できた時$c_n = \frac{a_n+b_n}{2}$として、
\begin{itemize}
\item
$\sup_{[a_n, c_n]}f \ge \sup_{[c_n, b_n]}f$の時、$a_{n+1} = a_n$, $b_{n+1} = c_n$とおき、
\item
$\sup_{[a_n, c_n]}f < \sup_{[c_n, b_n]}f$の時、$a_{n+1} = c_n$, $b_{n+1} = b_n$とおく。
\end{itemize}
この時、常に$\sup_{[a_n, b_n]}f = \sup_{[a, b]}f$であり、補題より$(a_n)$, $(b_n)$はともにある$a \le x \le b$に収束する。

ここで$f$が連続関数であることから任意の$\varepsilon > 0$に対して十分$x$に近い任意の$x' \in [a, b]$に対して$|f(x')-f(x)| < \varepsilon$より、
ある$n = 0, 1, 2, 3, \cdots$が存在して、任意の$x' \in [a_n, b_n]$に対して$f(x') < f(x)+\varepsilon$である。
よって$\sup_{[a, b]}f \le f(x)+\varepsilon$で$\varepsilon$は任意より$\sup_{[a, b]}f \le f(x)$で関数$f$は点$x$で最大になることが示された。

最大を持つことも同様に示される。
\end{proof}

\begin{example}
$x$についての方程式$x^3+x-1 = 0$を考える。
$f(x) = x^3+x-1$とおくとこれは連続関数で、$f(0) = -1 < 0$, $f(1) = 1 > 0$より中間値の定理から$f(x) = 0$は少なくとも$0$と$1$の間に実数解をもつ。
\end{example}

% 今後様々な計算をするにあたって、ある関数の値が与えられたときに元のが一意に存在すると便利なことが多い。

実数に対して実数値を対応させる関数が逆関数をもつことを保証するのが以下に定義する(狭義)単調性である。

\begin{definition}[関数の単調性]
$f$を区間$I$上の実数値関数とする。
\begin{itemize}
\item
任意の$a < b$を満たす$a, b \in I$に対して$f(a) < f(b)$が成立するとき$f$は$I$上で\emph{狭義単調増加}するという。
\item
任意の$a < b$を満たす$a, b \in I$に対して$f(a) \le f(b)$が成立するとき$f$は$I$上で\emph{広義単調増加}するという。
\item
任意の$a < b$を満たす$a, b \in I$に対して$f(a) > f(b)$が成立するとき$f$は$I$上で\emph{狭義単調減少}するという。
\item
任意の$a < b$を満たす$a, b \in I$に対して$f(a) \ge f(b)$が成立するとき$f$は$I$上で\emph{広義単調減少}するという。
\end{itemize}
\end{definition}

広義単調増加・広義単調減少の「広義」は省略される。

\begin{theorem}[単調関数の逆関数]
$f$を有界閉区間$[a, b]$上の連続関数とする。
\begin{itemize}
\item
$f$が$[a, b]$上で狭義単調増加するとき、逆関数$f^{-1}$が有界閉区間$[f(a), f(b)]$上の狭義単調増加する連続関数として存在する。
\item
$f$が$[a, b]$上で狭義単調減少するとき、逆関数$f^{-1}$が有界閉区間$[f(b), f(a)]$上の狭義単調減少する連続関数として存在する。
\end{itemize}
\end{theorem}

\begin{proof}
$f$が狭義単調増加するとき示す。
$f(a) \le y \le f(b)$を満たす点$y$に対して中間値の定理より$f(x) = y$となる$a \le x \le b$が存在する。
次にこのような$x$は一意であることを示す。
もし$f(x) = f(x') = y$となる2点$x, x' \in [a, b]$が存在して$x \ne x'$だったとする。
ここで$f$が狭義単調増加することから$x < x'$の場合は$f(x) < f(x')$となり矛盾、$x > x'$の場合は$f(x) > f(x')$となりやはり矛盾である。
よって$x = x'$なので、$f(x) = y$を満たす$x$は一意に存在し$f^{-1}(y)$とおいて$[f(a), f(b)]$上の関数$f^{-1}$を得る。
$f^{-1}$が狭義単調増加であることは同様の議論をすることで証明できる。
つまり、$y < y' \in [f(a), f(b)]$として$f^{-1}(y) \ge f^{-1}(y')$だったとすると、$f$の単調増加性より$y = f(f^{-1}(y)) \le f(f^{-1}(y')) = y'$で矛盾するので、$f^{-1}(y) < f^{-1}(y')$である。
\end{proof}

\section{一様連続性}

証明のためにハイネ・カントールの定理などと呼ばれる次の補題を用意する。

\begin{theorem}[有界閉区間上の連続関数]
\label{t_unif_conti}
有界閉区間$[a, b]$上の連続関数$f$は$[a, b]$上一様連続である。
\end{theorem}

\begin{proof}
$f$が一様連続でないと仮定すると、ある$\varepsilon > 0$が存在して任意の$\delta > 0$に対して$x, y \in [a, b]$, $|x-y| < \delta$が存在して$|f(x)-f(y)| \ge \varepsilon$である。
ここで$n = 1, 2, 3, \cdots$に対して$\delta = n^{-1}$として取れば、$x_n, y_n \in [a, b]$が存在して$|x_n-y_n| < n^{-1}$, $|f(x_n)-f(y_n)| \ge \varepsilon$が成り立つ。
$[a, b]$は有界なのでボルツァーノ・ワイエルシュトラスの定理より部分列を取れば数列$(x_n)$, $(y_n)$は収束し極限をそれぞれ$\bar{x}$, $\bar{y}$とおくと、$|x_n-y_n| < n^{-1} \to 0$より$\bar{x} = \bar{y}$である。
また、$[a, b]$は閉区間より$\bar{x} = \bar{y} \in [a, b]$もわかる。
ここで$f$は$\bar{x} = \bar{y}$で連続より、ある$\delta > 0$が存在して任意の$x \in [a, b]$に対して$|x-\bar{x}| < \delta$ならば$|f(x)-f(\bar{x})| < \frac{\varepsilon}{2}$が成り立つことに注意する。
$n$が大きいと$|x_n-\bar{x}|, |y_n-\bar{y}| < \delta$となるので、そのような$n$に対して三角不等式より
$$
\varepsilon \le |f(x_n)-f(y_n)| \le |f(x_n)-f(\bar{x})|+|f(\var{x})-f(\bar{y})|+|f(\bar{y})-f(y_n)| < \varepsilon
$$
となり矛盾が導かれた。
よって関数$f$は$[a, b]$上一様連続である。
\end{proof}

\section{連続性の度合い}

実数の集合$X$上の関数$f(x)$と点$a \in X$に対して
$$
\omega(r) = \sup\lrset{ |f(x)-f(a)| \mid x \in X, |x-a| \le r }
$$
で定義される関数$\omega(r)$を考える。
この関数は$[0, \infty)$上定義された非負値関数であり$\omega = 0$で$r$が大きくなると上限を取る$x$の範囲が広くなるので広義単調増加し、
定義から任意の$x \in X$に対して
$$
|f(x)-f(a)| \le \omega(|x-a|)
$$
が成立することわかる。

この式で連続性の特徴づけをするのが連続性の度合いである。

\begin{definition}[連続性の度合い]
条件
$$
\omega(r) \to 0 \quad (r \to 0+)
$$
を満たす$[0, \infty)$上定義された非負値の広義単調増加関数$\omega$を\emph{連続性の度合い}という。
\end{definition}

\begin{proposition}
$f(x)$を実数の集合$X$上の関数とする。
\begin{itemize}
\item
関数$f(x)$が点$x = a \in X$で連続であることは連続性の度合い$\omega$が存在して任意の$x \in X$に対して
$$
|f(x)-f(a)| \le \omega(|x-a|)
$$
が成立することと同値である。
\item
関数$f(x)$が連続関数であることは各$a \in X$に対して連続性の度合い$\omega_a$が存在して任意の$x \in X$に対して
$$
|f(x)-f(a)| \le \omega_a(|x-a|)
$$
が成立することと同値である。
\item
関数$f(x)$が一様連続関数であることは連続性の度合い$\omega$が存在して任意の$x, y \in X$に対して
$$
|f(x)-f(y)| \le \omega(|x-y|)
$$
が成立することと同値である。
\end{itemize}
\end{proposition}


\chapter{種々の関数1}

\section{べき関数}

実数$x$と自然数$n = 0, 1, 2, 3, \cdots$に累乗$x^n$を対応させる関数を\emph{自然数べき関数}あるいは\emph{べき関数}という。
例\ref{t_ex_conti}と命題\ref{t_conti_arith}から、べき関数$f(x) = x^n$は$\mathbb{R}$上の連続関数である。

$0$でない実数$x$と$0$でない自然数$n = 1, 2, 3, \cdots$に対して累乗$x^n$は$0$でないことに注意する。
そこで、関数
$$
x^{-n} = (x^n)^{-1}
$$
を定義し、\emph{負の整数べき関数}あるいは\emph{負べき関数}という。
また、これにより整数$n$に対して\emph{整数べき関数}$x^n$を定義する。
$x^{-0} = x^0 = 1$に注意する。
整数べき関数$f(x) = x^n$は常に$\mathbb{R}\setminus\lrset{0}$上の連続関数である。
もちろん$n$が自然数の時は自然数べき関数なので$\mathbb{R}$上の連続関数でもある。

\section{多項式関数}

$a_0, a_1, a_2, a_3, \cdots, a_n$を$n+1$個の実数定数として得られる\emph{多項式関数}
$$
f(x) = a_0+a_1 x+a_2 x^2+a_3 x^3+\cdots+a_n x^n
$$
は$\mathbb{R}$上の連続関数である。
もし$a_n \ne 0$ならば上の多項式関数$f$の\emph{次数}は$n$とし、$f(x) = 0$は特別に次数は$-\infty$とする。

多項式関数の$x \to \pm\infty$での極限では最高次の項$a_n x^n$が支配的になる。

\begin{proposition}[多項式関数の無限大での極限]
$f$を$n$次の多項式関数$f(x) = a_0+a_1 x+a_2 x^2+a_3 x^3+\cdots+a_n x^n$で、$n \ge 1$とする。
\begin{itemize}
\item
$a_n > 0$の時、$\lim_{x \to +\infty}f(x) = +\infty$であり、$n$が偶数ならば$\lim_{x \to -\infty}f(x) = +\infty$、$n$が奇数ならば$\lim_{x \to -\infty}f(x) = -\infty$である。
\item
$a_n < 0$の時、$\lim_{x \to +\infty}f(x) = -\infty$であり、$n$が偶数ならば$\lim_{x \to -\infty}f(x) = -\infty$、$n$が奇数ならば$\lim_{x \to -\infty}f(x) = +\infty$である。
\end{itemize}
\end{proposition}

\begin{remark}
$n$が奇数の時は$\lim_{x \to +\infty}f(x)$と$\lim_{x \to -\infty}f(x)$で符号が異なるので、中間値の定理から$f(x) = 0$の実数解が少なくとも$1$つ存在することがわかる。
$n$が偶数の時は実数解が存在しない場合がある。
\end{remark}

\section{累乗根関数}

非負の実数$x$と$n = 1, 2, 3, \cdots$に対して累乗根を累乗関数$f(y) = y^n$の逆関数から定義したい。
$0 \le a < b$に対して$a^n < b^n$より関数$f$は狭義単調増加な連続関数で$f(0) = 0$, $\lim_{x \to +\infty}f(x) = +\infty$より、
逆関数の定理から非負の実数$x$に対して定義された$f$の逆関数$g(x)$が存在し、$g$も狭義単調増加な連続関数で$g(0) = 0$, $\lim_{y \to +\infty}g(y) = +\infty$が成り立つ。
この逆関数を\emph{累乗根関数}といい$g(x) = \sqrt[n]{x}$と表す。
また、その値$\sqrt[n]{a}$を非負の実数$a$の\emph{$n$乗根}という。
$n$乗根$b = \sqrt[n]{a}$は$y$についての方程式$y^n = a$の一意な非負実数解である。

$n = 1$の時は$\sqrt[n]{a} = a$に他ならない、$n = 2$の時特別に\emph{平方根}、\emph{平方根関数}といい$\sqrt{a} = \sqrt[2]{a}$, $\sqrt{x} = \sqrt[2]{x}$と書く。

\section{指数関数}

$a$を正の実数とする。
ここでは実数$x$に対して指数関数$a^x$を定義する。

まず、$x$が整数$n$の時は整数べき関数の値$a^n$として定義される。
有理数$x = m/n$に対しては累乗根
$$
a^x = \sqrt[n]{a^m}
$$
この値は有理数$x$の表現によらないことに注意する。

実数$x$を考えるにあたって、有理数$x$に対して定義された関数$f(x) = a^x$は$a \ge 1$のとき単調増加で$a \le 1$のとき単調減少であることに注意する。
実数$x$に対して、
$$
a^x =
\begin{cases}
\sup\{ a^y \mid y \in \mathbb{Q}, y \le x \} & (a \ge 1), \\
\inf\{ a^y \mid y \in \mathbb{Q}, y \le x \} & (a \le 1) \\
\end{cases}
$$
と定義する。
$a = 1$の時は$a^x = 1$であることに注意する。

以上によって定義された実数$x$に対する関数$\exp_a(x) = a^x$を$a$を底とする\emph{指数関数}という。
指数関数には以下の性質が成り立つ。
\begin{itemize}
\item
$\exp_a$は$\mathbb{R}$上の連続関数である。
\item
常に$a^x > 0$が満たされる。
\item
$a > 1$のとき$\exp_a$は狭義単調増加で$\lim_{x \to -\infty}a^x = 0$, $\lim_{x \to +\infty}a^x = +\infty$を満たし、
$0 < a < 1$のとき$\exp_a$は狭義単調減少で$\lim_{x \to -\infty}a^x = +\infty$, $\lim_{x \to +\infty}a^x = 0$を満たす。
\item
任意の実数$x$, $y$に対して、\emph{指数法則}
$$
a^{x+y} = a^x a^y, \quad a^0 = 1
$$
と
$$
a^{x y} = (a^x)^y
$$
が成り立つ。
\end{itemize}

指数法則は和の指数関数は指数関数を取ったものに積を取るものになることを言っている。

今後の理論の展開において重要な極限として次がある。

\begin{theorem}
実数列$\qty(\qty(1+\frac{1}{n})^n)_{n = 1}^\infty$は収束する。
\end{theorem}

この時の極限を$e$とおき\emph{ネイピア数}という。
$$
e = \lim_{n \to \infty}\qty(1+\frac{1}{n})^n = 2.7182818284\cdots
$$
である。
ネイピア数は指数関数・対数関数の微分積分において重要な役割を果たす定数である。

\begin{proof}
二項定理より
$$
\qty(1+\frac{1}{n})^n
= \sum_{k = 0}^n \binom{n}{k}\frac{1}{n^k}
= \sum_{k = 0}^n \frac{1}{k!}\frac{n(n-1)\cdots(n-k+1)}{n^k}
= \sum_{k = 0}^n \frac{1}{k!}\qty(1-\frac{1}{n})\cdots\qty(1-\frac{k-1}{n})
$$
である。
ここで$i = 1, \cdots, k-1$に対して$\qty(1-\frac{i}{n}) < \qty(1-\frac{i}{n+1})$より、
$$
\qty(1+\frac{1}{n})^n
< \sum_{k = 0}^n \frac{1}{k!}\qty(1-\frac{1}{n+1})\cdots\qty(1-\frac{k-1}{n+1})
< \sum_{k = 0}^{n+1} \frac{1}{k!}\qty(1-\frac{1}{n+1})\cdots\qty(1-\frac{k-1}{n+1})
= \qty(1+\frac{1}{n+1})^{n+1}.
$$
つまり数列$\qty(\qty(1+\frac{1}{n})^n)_{n = 1}^\infty$は単調増加である。
また、
$$
\qty(1+\frac{1}{n})^n < \sum_{k = 0}^n \frac{1}{k!} < 1+\sum_{k = 1}^n \frac{1}{2^{k-1}} = 1+\frac{1-\frac{1}{2^n}}{1-\frac{1}{2}} < 3.
$$
つまり上に有界も示された。
よって、単調収束定理(定理\ref{t_mono_conv})より数列$\qty(\qty(1+\frac{1}{n})^n)_{n = 1}^\infty$は収束する。
\end{proof}

\begin{remark}
この証明と$n = 1$のときを考えると、$2 < e \le 3$がわかり、だいたいの値として覚えるとよい。
また、$e \le \sum_{k = 0}^\infty \frac{1}{k!}$もわかるが、実は等号が成り立つ。
このことについては微分の章のテイラーの定理や級数の章のべき級数を参照すること。
\end{remark}

ネイピア数$e$を底とする指数関数$\exp_e x = e^x$を単に$\exp$で表す。

\begin{proposition}
\label{t_limit_napier}
$$
\lim_{x \to +\infty}\qty(1+\frac{1}{x})^x = \lim_{x \to -\infty}\qty(1+\frac{1}{x})^x = \lim_{x \to 0}\qty(1+x)^\frac{1}{x} = e.
$$
\end{proposition}

\begin{proof}
まず実数$x$に対して$n = \floor{x}$とおくことで$n \le x < n+1$なので、
$$
\qty(1+\frac{1}{n+1})^n \le \qty(1+\frac{1}{x})^x \le \qty(1+\frac{1}{n})^{n+1}.
$$
よって、$x \to \infty$で$n \to \infty$となり、ネイピア数の定義より最左辺と最右辺がともに$e$に収束することから$\lim_{x \to +\infty}\qty(1+\frac{1}{x})^x = e$である。
次に
$$
\qty(1-\frac{1}{x})^{-x} = \qty(\frac{x}{x-1})^x = \qty(1+\frac{1}{x-1})^x = \qty(1+\frac{1}{x-1})^{x-1}\qty(1+\frac{1}{x-1})
$$
なので、やはり$\lim_{x \to +\infty}\qty(1-\frac{1}{x})^{-x} = e$であり、$x$を$-x$に置き換えることで、$\lim_{x \to -\infty}\qty(1+\frac{1}{x})^x = e$が得られる。
さらに$x$を$\frac{1}{x}$で置き換えることで、
$$
\lim_{x \to 0+}\qty(1+x)^\frac{1}{x} = \lim_{x \to 0-}\qty(1+x)^\frac{1}{x} = e
$$
がわかり、右極限と左極限が一致しているので$\lim_{x \to 0}\qty(1+x)^\frac{1}{x} = e$である。
\end{proof}

\section{対数関数}

$a$を$a > 0$, $a \ne 1$を満たす実数とする。
この時、前節の内容から指数関数$\exp_a$に逆関数が存在し$a$を底とする\emph{対数関数}といい$\log_a$で表す。
正の実数$x$に対して対数関数の値$\log_a x$を$a$を底とする真数$x$の\emph{対数}という。

\begin{itemize}
\item
$\log_a$は$(0, +\infty)$上の連続関数である。
\item
任意の正の実数$x$に対して$\exp_a(\log_a x) = x$で、
任意の実数$y$に対して$\log_a(\exp_a y) = y$が成り立つ。
\item
$a > 1$のとき$\log_a$は狭義単調増加で$\lim_{x \to 0}\log_a x = -\infty$, $\lim_{x \to +\infty}\log_a x = +\infty$を満たし、
$0 < a < 1$のとき$\log_a$は狭義単調減少で$\lim_{x \to 0}\log_a x = +\infty$, $\lim_{x \to +\infty}\log_a x = -\infty$を満たす。
\item
任意の正の実数$x$, $y$に対して、
$$
\log_a x y = \log_a x+\log_a y, \quad \log_a 1 = 0
$$
が成り立つ。
\end{itemize}

対数関数の一番最後の性質は積の対数は対数の和になることを言っている。

ネイピア数$e$を底とする対数を\emph{自然対数}といい、関数$\log_e x$を単に$\log$で表す。

\begin{proposition}
$$
\lim_{x \to 0}\frac{\log(1+x)}{x} = 1,
\quad \lim_{x \to 0}\frac{e^x-1}{x} = 1.
$$
\end{proposition}

\begin{proof}
一つ目の式は命題\ref{t_limit_napier}の三つ目の極限で自然対数を取れば直ちに得られる。
二つ目の式は$t = e^x-1$と置き換えることで、$x \to 0$で$t \to 0$となることと$x = \log(1+t)$より、
$$
\lim_{x \to 0}\frac{e^x-1}{x} = \lim_{t \to 0}\frac{t}{\log(1+t)} = 1
$$
である。
ここですでに得られた一つ目の式を使った。
\end{proof}

\section{三角関数}

直角三角形の辺の長さの比から定まる三角比をもとにした\emph{正弦関数}$\sin x$, \emph{余弦関数}$\cos x$, \emph{正接関数}$\tan x$は次を満たす関数であり、まとめて\emph{三角関数}という。
\begin{itemize}
\item
$\sin x$, $\cos x$はともに$\mathbb{R}$上の実数値連続関数であり、任意の実数$x$に対して
$$
\cos^2 x+\sin^2 x = 1
$$
が成り立つ。
\item
$\sin x$, $\cos x$はともに周期$2\pi$, $\pi = 3.1415926535\cdots$を持つ、つまり任意の実数$x$に対して
$$
\sin(x+2\pi) = \sin x, \quad \cos(x+2\pi) = \cos x
$$
が成り立つ。
\item
$\sin 0 = 0$, $\sin \frac{\pi}{6} = \frac{1}{2}$, $\sin \frac{\pi}{4} = \frac{1}{\sqrt{2}}$, $\sin \frac{\pi}{3} = \frac{\sqrt{3}}{2}$, $\sin \frac{\pi}{2} = 1$で、実数$x$に対して
$$
\sin\left(\frac{\pi}{2}-x\right) = \cos x, \quad \sin(\pi-x) = \sin x, \quad \sin(-x) = -\sin x
$$
が成り立つ。
\item
$\sin x$は$[-\frac{\pi}{2}, \frac{\pi}{2}]$で狭義単調増加で、$\cos x$は$[0, \pi]$で狭義単調減少である。
\item
加法定理、つまり任意の実数$x$, $y$に対して、
$$
\sin(x+y) = \sin x\cos y+\cos x\sin y,
\quad \cos(x+y) = \cos x\cos y-\sin x\sin y
$$
が成り立つ。
\item
極限の式
$$
\lim_{x \to 0}\frac{\sin x}{x} = 1
$$
が成り立つ。
\item
$\tan x = \frac{\sin x}{\cos x}$であり、$x \in \mathbb{R}\setminus\{ \frac{\pi}{2}+n\pi \mid n \in \mathbb{Z}\}$に対して定義される。
\end{itemize}
ただし、途中に出てくる定数$\pi$を\emph{円周率}という。
$(\sin x)^n$などはしばしば$\sin^n x$と記述される。

このテキストではこれらの三角関数は存在するものとして話を進める。
三角関数の定義つまりこれらの性質を満たす関数の構成は、指数関数の複素数への拡張による方法やべき級数による方法、微分方程式の解として特徴づける方法などがあるが、現段階の知識ではうまく定義できない。
ただし最初の方法についてあらましを述べると、指数関数を複素数$z$に拡張して$\exp z = e^z$を構成すると\emph{オイラーの関係式}$e^{i x} = \cos x+i\sin x$によって三角関数$\cos x$と$\sin x$を得る。
複素数に拡張しても複素数$z$, $w$に対して指数法則$e^{z+w} = e^z e^w$と極限$\frac{e^z-1}{z} = 1$ ($z \to 0$)が成り立つので、三角関数の加法定理と極限の式が導かれる。
加法定理より指数法則の方が単純で覚えやすいので、加法定理を忘れたとしても指数法則とオイラーの関係式から導くことができる。

\section{逆三角関数}

正弦関数$\sin y$は$[-\frac{\pi}{2}, +\frac{\pi}{2}]$上で狭義単調増加な連続関数で$\sin(-\frac{\pi}{2}) = -1$, $\sin(+\frac{\pi}{2}) = +1$より、$\sin y$の逆関数として\emph{逆正弦関数}$\arcsin x$が$[-1, +1]$上の$[-\frac{\pi}{2}, +\frac{\pi}{2}]$値の狭義単調増加な連続関数として得られる。
余弦関数$\cos y$は$[0, \pi]$上で狭義単調減少な連続関数で$\cos 0 = +1$, $\cos \pi = -1$より、$\cos y$の逆関数として\emph{逆余弦関数}$\arccos x$が$[-1, +1]$上の$[0, \pi]$値の狭義単調減少な連続関数として得られる。
正接関数$\tan y$は$(-\frac{\pi}{2}, +\frac{\pi}{2})$上で狭義単調増加な連続関数で$\lim_{y \to -\frac{\pi}{2}}\tan y = -\infty$, $\lim_{y \to +\frac{\pi}{2}}\tan y = +\infty$より、$\tan y$の逆関数として\emph{逆正接関数}$\arctan x$が$\mathbb{R}$上の$(-\frac{\pi}{2}, +\frac{\pi}{2})$値の狭義単調増加な連続関数として得られる。
これらの関数をまとめて\emph{逆三角関数}という。

\begin{example}
逆三角関数は三角比を角度に変換し、三角関数と合成させることで三角比や角度を別の三角比や角度に変換できる。
例えば、$x \in [-1, +1]$に対して、$\arcsin x \in [-\frac{\pi}{2}, +\frac{\pi}{2}]$より、
$$
\cos(\arcsin x) = \sqrt{1-\sin^2(\arcsin x)} = \sqrt{1-x^2}
$$
である。
\end{example}

\section{双曲線関数}

実数$x$に対して次で定まる$\mathbb{R}$上の関数として\emph{双曲線正弦関数}$\sinh x$、\emph{双曲線余弦関数}$\cosh x$、\emph{双曲線正接関数}$\tanh x$を導入する。
$$
\sinh x = \frac{e^x-e^{-x}}{2}, \quad \cosh x = \frac{e^x+e^{-x}}{2}, \quad \tanh x = \frac{\sinh x}{\cosh x}.
$$
これらの関数をまとめて\emph{双曲線関数}という。

この定義からすぐに以下の性質がわかる。
\begin{itemize}
\item
$\sinh x$, $\cosh x$はともに$\mathbb{R}$上の実数値連続関数であり、任意の実数$x$に対して
$$
\cosh^2 x-\sinh^2 x = 1
$$
が成り立つ。
\item
$\sinh x$は$\mathbb{R}$で狭義単調増加で、$\cosh x$は$(-\infty, -0]$で狭義単調減少、$[+0, +\infty)$で狭義単調増加である。
\item
$\sinh 0 = 0$, $\lim_{x \to +\infty}\sinh x = +\infty$, $\cosh 0 = 1$, $\lim_{x \to +\infty}\cosh x = +\infty$で、実数$x$に対して
$$
\sinh(-x) = -\sinh x, \quad \cosh(-x) = \cosh x
$$
が成り立つ。
\item
加法定理、つまり任意の実数$x$, $y$に対して、
$$
\sinh(x+y) = \sinh x\cosh y+\cosh x\sinh y,
\quad \cosh(x+y) = \cosh x\cosh y+\sinh x\sinh y
$$
が成り立つ。
\item
極限の式
$$
\lim_{x \to 0}\frac{\sinh x}{x} = 1
$$
が成り立つ。
\item
$\tanh x$は$\mathbb{R}$上の連続関数である。
\end{itemize}
最初の等式は平面上の点$(\cosh x, \sinh x)$が双曲線$x^2-y^2 = 1$上の点であることを表しそこから双曲線関数と呼ばれる。
$(\sinh x)^n$などはしばしば$\sinh^n x$と記述される。

\section{逆双曲線関数}

双曲線正弦関数$\sinh y$は$\mathbb{R}$上で狭義単調増加な連続関数で$\lim_{y \to -\infty}\sinh y = -\infty$, $\lim_{y \to +\infty}\sinh y = +\infty$より、$\sinh y$の逆関数として\emph{逆双曲線正弦関数}$\arsinh x$が$\mathbb{R}$上の$\mathbb{R}$値の狭義単調増加な連続関数として得られる。
双曲線余弦関数$\cosh y$は$[0, +\infty)$上で狭義単調増加な連続関数で$\cosh 0 = 1$, $\lim_{y \to +\infty}\cosh y = +\infty$より、$\cos y$の逆関数として\emph{逆双曲線余弦関数}$\arcosh x$が$[1, +\infty)$上の$[0, +\infty)$値の狭義単調増加な連続関数として得られる。
双曲線正接関数$\tanh y$は$\mathbb{R}$上で狭義単調増加な連続関数で$\lim_{y \to -\infty}\tan y = -1$, $\lim_{y \to +\infty}\tanh y = +1$より、$\tanh y$の逆関数として\emph{逆双曲線正接関数}$\arctan x$が$(-1, +1)$上の$\mathbb{R}$値の狭義単調増加な連続関数として得られる。
これらの関数をまとめて\emph{逆双曲線関数}という。

上では逆双曲線関数を抽象的に定義したが、双曲線関数の表示にもとづいて$\sinh y = \frac{e^y-e^{-y}}{2} = x$を$x$について解くことで、次の表示を得られる。
$$
\arsinh x = \log\left(x+\sqrt{x^2+1}\right),
\quad \arcosh x = \log\left(x+\sqrt{x^2-1}\right) \quad (x \ge 1),
\quad \artanh x = \frac{1}{2}\log\frac{1+x}{1-x} \quad (-1 < x < +1).
$$

\section{その他の関数}

実数$x$に対して絶対値$|x|$を対応させる関数を\emph{絶対値関数}といい$\opabs$や$|x|$と表す。
絶対値関数は$\mathbb{R}$上の連続関数である。

実数$x$に対して次の符号$\sgn(x)$を対応させる関数を\emph{符号関数}といい$\sgn$と表す。
$$
\sgn(x) =
\begin{cases}
+1 & \text{($x > 0$),} \\
0 & \text{($x = 0$),} \\
-1 & \text{($x < 0$).} \\
\end{cases}
$$
符号関数は$\mathbb{R}\setminus\{ 0 \}$で連続関数であるが$0$で連続でない。

実数$x$に対して床$\lfloor x \rfloor$を対応させる関数を\emph{床関数}といい$\floor$や$\lfloor x \rfloor$と表す。
実数$x$に対して天井$\lceil x \rceil$を対応させる関数を\emph{天井関数}といい$\ceil$や$\lceil x \rceil$と表す。
床関数と天井関数は$\mathbb{R}\setminus\mathbb{Z}$で連続関数であるが整数の点で連続でない。
しかしながら、右連続・左連続を考えると、床関数は整数の点でも右連続で、天井関数は整数の点でも左連続である。


\chapter{微分}

\section{微分係数と微分導関数}

この節では実数の集合$X$上で定義された実数値関数$f: X \to \mathbb{R}$と$X$の点$a$を考える。
特に$a$に対して正の数$\delta > 0$が存在して$B_\delta(a) = \lrset{ x \in \mathbb{R} \mid |x-a| < \delta }$は$X$の部分集合になっている状態を考え、このような点$a$を$X$の\emph{内点}という。

\begin{definition}[微分係数]
$f$を実数の集合$X$上で定義された実数値関数として、$a$を$X$の内点とする。
ここで、$x \in X\setminus\lrset{a}$で定義された関数$\frac{f(x)-f(a)}{x-a}$が$x \to a$で収束する時、
関数$f(x)$は点$x = a$で\emph{微分可能}であるといい極限を\emph{微分係数}という。
$y = f(x)$の点$x = a$での微分係数を
$$
f'(a), \quad \dv{f}{x}{(a)}, \quad \eval{\dv{f}{x}}_{x = a}, \quad \eval{\dv{x}(f(x))}_{x = a}, \quad \eval{y'}_a, \quad \eval{\dv{y}{x}}_a
$$
などで表す。
つまり、
$$
f'(a) = \lim_{x \to a}\frac{f(x)-f(a)}{x-a} = \lim_{h \to 0}\frac{f(a+h)-f(a)}{h}
$$
である。
\end{definition}

\begin{remark}
微分可能であるためには連続である必要がある。
\end{remark}

微分を考える動機として接線の計算が挙げられる。
つまり、関数$y = f(x)$のグラフに対して点$(x, y) = (a, f(a))$と点$(x, y) = (a+h, f(a+h))$ ($h \ne 0$)を通る直線は
$$
y = \frac{f(a+h)-f(a)}{h}(x-a)+f(a)
$$
となるが、$f$が$a$で微分可能とすると$h$を$0$に近づけることで直線の方程式
\begin{equation}
\label{e_tangent}
y = f'(a)(x-a)+f(a)
\end{equation}
を得る。
関数$y = f(x)$のグラフとこの直線は視覚的には接しているように見えることが多く、
実際図形的に定義される円(の一部)の接線と上記の直線は一致する。
そこで関数$y = f(x)$と点$x = a$に対して上記の方程式\ref{e_tangent}で定まる直線を$y = f(x)$の$x = a$での\emph{接線}という。

それはさておき、以下のことは複雑な形の関数の微分を計算する際に基礎的である。

\begin{proposition}[和と定数倍の微分]
関数$f(x)$, $g(x)$が点$x = a$で微分可能として、$c$を実数の定数とする。
この時、和$f(x)+g(x)$と定数倍$c f(x)$も$x = a$で微分可能で、
$$
\eval{\dv{x}(f(x)+g(x))}_{x = a} = f'(a)+g'(a),
\quad \eval{\dv{x}(c f(x))}_{x = a} = c f'(a)
$$
が成立する。
\end{proposition}

このことを微分の線形性という。

\begin{proof}
$x \to a$で
$$
\frac{(f(x)+g(x))-(f(a)+g(a))}{x-a}
= \frac{f(x)-f(a)}{x-a}+\frac{g(x)-g(a)}{x-a}
\to f'(a)+g'(a),
$$
$$
\frac{c f(x)-c f(a)}{x-a}
= c\frac{f(x)-f(a)}{x-a}
\to c f'(a)
$$
より示される。
\end{proof}

\begin{proposition}[積の微分]
関数$f(x)$, $g(x)$が点$x = a$で微分可能とする。
この時、積$f(x)g(x)$も$x = a$で微分可能で、
$$
\eval{\dv{x}(f(x)g(x))}_{x = a} = f'(a)g(a)+f(a)g'(a)
$$
が成立する。
\end{proposition}

\begin{proof}
$x \to a$で
$$
\frac{f(x)g(x)-f(a)g(a)}{x-a}
% = \frac{f(x)g(x)-f(a)g(x)+f(a)g(x)-f(a)g(a)}{x-a}
= \frac{f(x)-f(a)}{x-a}g(x)+f(a)\frac{g(x)-g(a)}{x-a}
\to f'(a)g(a)+f(a)g'(a)
$$
より示される。
\end{proof}

\begin{proposition}[商の微分]
関数$f(x)$, $g(x)$が点$x = a$で微分可能で$g(a) \ne 0$とする。
この時、積$\frac{f(x)}{g(x)}$も$x = a$で微分可能で、
$$
\eval{\dv{x}(\frac{f(x)}{g(x)})}_{x = a} = \frac{f'(a)g(a)-f(a)g'(a)}{g(a)^2}
$$
が成立する。
\end{proposition}

\begin{proof}
$x \to a$で
$$
\frac{\frac{f(x)}{g(x)}-\frac{f(a)}{g(a)}}{x-a}
= \frac{f(x)g(a)-f(a)g(x)}{g(x)g(a)(x-a)}
= \frac{(f(x)-f(a))g(a)-f(a)(g(x)-g(a))}{g(x)g(a)(x-a)}
\to \frac{f'(a)g(a)-f(a)g'(a)}{g(a)^2}
$$
より示される。
\end{proof}

\begin{proposition}[合成関数の微分]
関数$f(x)$が点$x = a$で微分可能で$g(y)$が点$y = f(x)$で微分可能とする。
この時、合成関数$f(g(x))$も$x = a$で微分可能で、
$$
\eval{\dv{x}(g(f(x)))}_{x = a} = g'(f(a))f'(a)
$$
が成立する。
\end{proposition}

\begin{proof}
$x \to a$で$f(x) \to f(a)$に注意して、
$$
\frac{g(f(x))-g(f(a))}{x-a}
= \frac{g(f(x))-g(f(a))}{f(x)-f(a)}\frac{f(x)-f(a)}{x-a}
\to g'(f(a))f'(a)
$$
より示される。
\end{proof}

この公式は$y = f(x)$, $z = g(y)$として
$$
\dv{x}{z} = \dv{y}{z}\dv{x}{y}
$$
と考えると覚えやすい。

関数の極限に右極限・左極限あったように微分の概念にも右微分と左微分が定義できる。

\begin{definition}[片側微分]
$f$を実数の集合$X$上で定義された実数値関数として、$a$を$X$の内点とする。
\begin{itemize}
\item
$x \in X\setminus\lrset{a}$で定義された関数$\frac{f(x)-f(a)}{x-a}$が右からの極限$x \to a+$で収束する時、
関数$f(x)$は点$x = a$で\emph{右微分可能}であるといい極限を\emph{右微分係数}という。
$y = f(x)$の点$x = a$での右微分係数を
$$
f'_+(a), \dv{f}{x^+}{(a)}, \quad \eval{\dv{f}{x^+}}_{x = a}, \quad \eval{\dv{x^+}(f(x))}_{x = a}, \quad \eval{\dv{y}{x^+}}_a
$$
などで表す。
つまり、
$$
f'_+(a) = \lim_{x \to a+}\frac{f(x)-f(a)}{x-a} = \lim_{h \to 0+}\frac{f(a+h)-f(a)}{h}
$$
である。
\item
$x \in X\setminus\lrset{a}$で定義された関数$\frac{f(x)-f(a)}{x-a}$が左からの極限$x \to a-$で収束する時、
関数$f(x)$は点$x = a$で\emph{左微分可能}であるといい極限を\emph{左微分係数}という。
$y = f(x)$の点$x = a$での左微分係数を
$$
f'_-(a), \dv{f}{x^-}{(a)}, \quad \eval{\dv{f}{x^-}}_{x = a}, \quad \eval{\dv{x^-}(f(x))}_{x = a}, \quad \eval{\dv{y}{x^-}}_a
$$
などで表す。
つまり、
$$
f'_-(a) = \lim_{x \to a-}\frac{f(x)-f(a)}{x-a} = \lim_{h \to 0-}\frac{f(a+h)-f(a)}{h}
$$
である。
\end{itemize}
\end{definition}

右微分・左微分の記号は$x$の肩でなく上の$\dd$の肩に符号を書くことが多いが、本テキストでは様々な事情によりこの記法を採用する。

微分可能であることの必要十分条件は右微分係数と左微分係数が一致することである。

\begin{proposition}
\label{t_semidiff}
関数$f = f(x)$が点$x = a$で微分可能であることの必要十分条件は、
関数$f = f(x)$が点$x = a$で右微分可能かつ左微分可能で
$$
f'_+(a) = f'_-(a)
$$
が成り立つことである。
\end{proposition}

\begin{proof}
\end{proof}

実数の集合$X$で$X$のすべての点が$X$の内点になっているものを\emph{開集合}という。
開区間は開集合である。

\begin{definition}[微分導関数]
$f$を開集合$X$上で定義された実数値関数とする。
ここで、$X$のすべての点で関数$f$が微分可能であるとき$f$は$X$上微分可能であるという。
この時、$x \in X$に対して微分係数$f'(x)$を対応させる集合$X$上の実数値関数を$f$の\emph{微分導関数}という。
$y = f(x)$の微分導関数を
$$
f', \quad \dv{f}{x}, \quad \dv{x}(f(x)), \quad y', \quad \dv{y}{x}
$$
などで表す。
\end{definition}

微分導関数はしばしば\emph{導関数}と略される。

これまでの内容から以下がわかる。

\begin{proposition}[四則演算の導関数]
関数$f$, $g$が開集合$X$で微分可能とする。
この時、和$f(x)+g(x)$、差$f(x)-g(x)$、積$f(x)g(x)$、各$x \in X$で$g(x) \ne 0$のとき商$\frac{f(x)}{g(x)}$も$X$で微分可能で、
$$
\dv{x}(f(x)+g(x)) = f'(x)+g'(x),
\quad \dv{x}(f(x)-g(x)) = f'(x)-g'(x),
$$
$$
\dv{x}(f(x)g(x)) = f'(x)g(x)+f(x)g'(x),
\quad \dv{x}\qty(\frac{f(x)}{g(x)}) = \frac{f'(x)g(x)-f(x)g'(x)}{g(x)^2}
$$
が成立する。
\end{proposition}

\begin{proposition}[合成関数の導関数]
関数$f$が開集合$X$で微分可能で、関数$g$が開集合$Y$で微分可能とする。
$f(X) \subset Y$のとき、合成関数$g(f(x))$は$X$で微分可能で、
$$
\dv{x}(g(f(x))) = g'(f(x))f'(x)
$$
が成立する。
\end{proposition}

\begin{proposition}[逆関数の導関数]
関数$f$が開区間$X$で微分可能で狭義単調増加で逆関数を$f^{-1}$とする。
この時各$x \in X$に対して$f'(x) \ne 0$であり、$f^{-1}(y)$は開区間$Y = f(X)$で微分可能で、
$$
\dv{y}(f^{-1}(y)) = (f'(f^{-1}(y)))^{-1}
$$
が成立する。
\end{proposition}

\section{種々の関数の微分}

$c$を実数として定数関数$c$は$\mathbb{R}$で微分可能で、
$$
\dv{x}(c)
= \lim_{h \to 0}\frac{c-c}{h}
= 0
$$
である。

実数$x$と$n = 1, 2, 3, \cdots$に対してべき関数$x^n$は$\mathbb{R}$で微分可能で、
$$
\begin{aligned}
\dv{x}(x^n)
&= \lim_{h \to 0}\frac{(x+h)^n-x^n}{h}
= \lim_{h \to 0}\frac{\binom{n}{0}x^n+\binom{n}{1}h x^{n-1}+\binom{n}{2}h^2 x^{n-2}+\cdots+\binom{n}{n}h^n-x^n}{h} \\
&= \lim_{h \to 0}\qty(n x^{n-1}+\binom{n}{2}h x^{n-2}+\cdots+\binom{n}{n}h^{n-1})
= n x^{n-1}
\end{aligned}
$$
である。

$0$でない実数$x$と$n = 1, 2, 3, \cdots$に対して負べき関数$x^{-n}$は$\mathbb{R}\setminus\lrset{0}$で微分可能で、
$$
\begin{aligned}
\dv{x}(x^{-n})
&= \lim_{h \to 0}\frac{(x+h)^{-n}-x^{-n}}{h}
= \lim_{h \to 0}\frac{x^n-(x+h)^n}{(x+h)^n x^n h}
= \lim_{h \to 0}\frac{-n x^{n-1}-\binom{n}{2}h x^{n-2}+\cdots+\binom{n}{n}h^{n-1}}{(x+h)^n x^n} \\
&= \frac{-n x^{n-1}}{x^{2 n}}
= -n x^{-n-1}
\end{aligned}
$$
である。

定数関数とべき関数と負べき関数の結果を合わせて整数$n$に対して、
$$
\dv{x}(x^n) = n x^{n-1}
$$
とまとめて書けるが、$x$の範囲が異なることに注意する。

多項式関数は$\mathbb{R}$で微分可能で、$a_0, a_1, a_2, a_3, \cdots, a_n$を$n+1$個の実数定数として
$$
\dv{x}(a_0+a_1 x+a_2 x^2+a_3 x^3+\cdots+a_n x^n) = a_1+2 a_2 x+3 a_3 x^2+\cdots+n a_n x^{n-1}
$$
である。

指数関数の微分を考える際にはまずネイピア数$e$を底とする指数関数$\exp x = e^x$を考える。
指数関数$\exp$は$\mathbb{R}$で微分可能で、命題より
$$
\dv{x}(\exp(x)) = \dv{x}(e^x) = \lim_{h \to 0}\frac{e^{x+h}-e^x}{h} = \lim_{h \to 0}\frac{e^h-1}{h}e^x = e^x
$$
である。
一般の底$a > 0$に対しては$a = e^{\log a}$に注意して合成関数の微分より
$$
\dv{x}(\exp_a(x)) = \dv{x}(a^x) = \dv{x}(e^{x\log a}) = e^{x\log a}\log a = a^x\log a
$$
である。

対数関数$\log_a$は指数関数の逆関数であることから$(0, +\infty)$で微分可能で、
$$
\dv{x}(\log x) = \qty(\eval{\dv{y}(e^y)}_{y = \log x})^{-1} = x^{-1} = \frac{1}{x},
$$
$$
\dv{x}(\log_a x) = \qty(\eval{\dv{y}(a^y)}_{y = \log_a x})^{-1} = (x\log a)^{-1} = \frac{1}{x\log a}
$$
である。

正の数$x$と実数$a$に対して関数$x^a$は$(0, +\infty)$で微分可能で、$x = e^{\log x}$に注意して、
$$
\dv{x}(x^a) = \dv{x}(e^{a\log x}) = e^{a\log x}\frac{a}{x} = a x^{a-1}
$$
である。
このような方法は関数$f(x)$の微分を計算するときに対数を取った関数$\log f(x)$の微分を考えているので、対数微分法と呼ばれる。

\begin{example}
$f(x) = x^x$ ($x > 0$)とおくと、
$$
f'(x) = \dv{x}(x^x) = \dv{x}(e^{x\log x}) = e^{x\log x}\qty(\log x\frac{x}{x}) = (1+\log x)x^x.
$$
\end{example}

正弦関数$\sin$は$\mathbb{R}$で微分可能で、正弦関数の和積公式と極限の公式より
$$
\dv{x}(\sin x)
= \lim_{h \to 0}\frac{\sin(x+h)-\sin(x)}{h}
= \lim_{h \to 0}\frac{2\cos(x+\frac{h}{2})\sin(\frac{h}{2})}{h}
= \lim_{h \to 0}\cos(x+\frac{h}{2})\frac{\sin(\frac{h}{2})}{\frac{h}{2}}
= \cos x.
$$
余弦関数$\cos$も$\mathbb{R}$で微分可能で、
$$
\dv{x}(\cos x)
= \lim_{h \to 0}\frac{\cos(x+h)-\cos(x)}{h}
= \lim_{h \to 0}\frac{-2\sin(x+\frac{h}{2})\sin(\frac{h}{2})}{h}
= -\lim_{h \to 0}\sin(x+\frac{h}{2})\frac{\sin(\frac{h}{2})}{\frac{h}{2}}
= -\sin x.
$$
正接関数$\tan$は$(-\frac{\pi}{2}, +\frac{\pi}{2})$で微分可能で、
$$
\dv{x}(\tan x)
= \dv{x}\qty(\frac{\sin x}{\cos x})
= \frac{\dv{x}(\sin x)\cos x-\sin x\dv{x}(\cos x)}{\cos^2 x}
= \frac{\cos^2 x+\sin^2 x}{\cos^2 x}
= \frac{1}{\cos^2 x}.
$$

逆正弦関数$\arcsin$は$(-1, +1)$で微分可能で、
$$
\dv{x}(\arcsin x)
= \qty(\eval{\dv{y}(\sin y)}_{y = \arcsin x})^{-1}
= (\cos(\arcsin x))^{-1}
= \frac{1}{\sqrt{1-\sin^2(\arcsin x)}}
% = \qty(\sqrt{1-x^2})^{-1}
= \frac{1}{\sqrt{1-x^2}}.
$$
逆余弦関数$\arccos$も$(-1, +1)$で微分可能で、
$$
\dv{x}(\arccos x)
= \qty(\eval{\dv{y}(\cos y)}_{y = \arccos x})^{-1}
= (-\sin(\arccos x))^{-1}
= -\frac{1}{\sqrt{1-\cos^2(\arccos x)}}
= -\frac{1}{\sqrt{1-x^2}}.
$$
逆正接関数$\arctan$は$\mathbb{R}$で微分可能で、
$$
\dv{x}(\arctan x)
= \qty(\eval{\dv{y}(\tan y)}_{y = \arctan x})^{-1}
= (\frac{1}{\cos^2(\arctan x)})^{-1}
= \frac{1}{1+\tan^2(\arctan x)}
= \frac{1}{1+x^2}.
$$
ここで$\cos^2 x+\sin^2 x = 1$から従う公式$1+\tan^2 x = \frac{1}{\cos^2 x}$を用いた。

双曲線正弦関数$\sinh x$は$\mathbb{R}$で微分可能で、
$$
\dv{x}(\sinh x)
= \dv{x}\qty(\frac{e^x-e^{-x}}{2})
= \frac{e^x+e^{-x}}{2}
= \cosh x.
$$
双曲線余弦関数$\cosh x$も$\mathbb{R}$で微分可能で、
$$
\dv{x}(\cosh x)
= \dv{x}\qty(\frac{e^x+e^{-x}}{2})
= \frac{e^x-e^{-x}}{2}
= \sinh x.
$$
双曲線正接関数$\tanh x$は$\mathbb{R}$で微分可能で、
$$
\dv{x}(\tanh x)
= \dv{x}\qty(\frac{\sinh x}{\cosh x})
= \frac{\cosh^2 x-\sinh^2 x}{\cosh^2 x}
= \frac{1}{\cosh^2 x}.
$$

逆双曲線正弦関数$\arsinh x$は$\mathbb{R}$で微分可能で、
$$
\dv{x}(\arsinh x)
= \qty(\eval{\dv{y}(\sinh y)}_{y = \arsinh x})^{-1}
= (\cosh(\arsinh x))^{-1}
= \frac{1}{\sqrt{1+\sinh^2(\arsinh x)}}
= \frac{1}{\sqrt{1+x^2}}.
$$
逆双曲線余弦関数$\arcosh x$は$(1, +\infty)$で微分可能で、
$$
\dv{x}(\arcosh x)
= \qty(\eval{\dv{y}(\cosh y)}_{y = \arcosh x})^{-1}
= (\sinh(\arcosh x))^{-1}
= \frac{1}{\sqrt{\cosh^2(\arcosh x)-1}}
= \frac{1}{\sqrt{x^2-1}}.
$$
逆双曲線正接関数$\artanh x$は$(-1, +1)$で微分可能で、
$$
\dv{x}(\artanh x)
= \qty(\eval{\dv{y}(\tanh y)}_{y = \artanh x})^{-1}
= \qty(\frac{1}{\cosh^2(\artanh x)})^{-1}
= \frac{1}{1-\tanh^2(\artanh x)}
= \frac{1}{1-x^2}.
$$
ここで$\cosh^2 x-\sinh^2 x = 1$から従う公式$1-\tanh^2 x = \frac{1}{\cosh^2 x}$を用いた。

絶対値関数$\opabs$は$\mathbb{R}\setminus\lrset{0}$で微分可能で、
$$
\dv{x}(\opabs(x))
= \dv{x}(|x|)
=
\begin{cases}
\dv{x}(+x) & (x > 0) \\
\dv{x}(-x) & (x < 0) \\
\end{cases}
=
\begin{cases}
+1 & (x > 0) \\
-1 & (x < 0) \\
\end{cases}
= \sgn(x).
$$
$x = 0$では微分可能でないことに注意する。
しかしながら、$\opabs$は$x = 0$で右微分可能かつ左微分可能であり、
$$
\eval{\dv{x^+}(\opabs(x))}_{x = a} = \eval{\dv{x^+}(|x|)}_{x = a} = +1,
\quad \eval{\dv{x^-}(\opabs(x))}_{x = a} = \eval{\dv{x^-}(|x|)}_{x = a} = -1,
$$
が成り立つ。

符号関数$\sgn$は$\mathbb{R}\setminus\lrset{0}$で微分可能で、
$$
\dv{x}(\sgn(x))
=
\begin{cases}
\dv{x}(+1) & (x > 0) \\
\dv{x}(-1) & (x < 0) \\
\end{cases}
= 0.
$$
$x = 0$では微分可能でないことに注意する。

床関数$\floor$と天井関数$\ceil$はともに$\mathbb{R}\setminus\mathbb{Z}$で微分可能で、
$$
\dv{x}(\floor(x))
= \dv{x}(\ceil(x))
= 0.
$$
$x \in \mathbb{Z}$では微分可能でないことに注意する。
しかしながら、$\floor$は$x \in \mathbb{Z}$で右微分可能であり、$\ceil$は$x \in \mathbb{Z}$で左微分可能である。

ここで、対数関数$\log$は$(0, \infty)$で微分できるが、導関数$\frac{1}{x}$は$\mathbb{R}\setminus\lrset{0}$で定義されている。
そこで絶対値関数との合成を取ることで関数の定義される集合を拡張するという技法が知られている。
つまり、$0$でない実数$x$に対して
$$
\dv{x}(\log|x|)
=
\begin{cases}
\dv{x}(\log(+x)) & (x > 0) \\
\dv{x}(\log(-x)) & (x < 0) \\
\end{cases}
=
\begin{cases}
\frac{+1}{+x} & (x > 0) \\
\frac{-1}{-x} & (x < 0) \\
\end{cases}
= \frac{1}{x}
$$
である。
同様のことは逆双曲線正接関数$\artanh$についてもいえるが、こちらは$\artanh x = \frac{1}{2}\log\frac{1+x}{1-x}$であることを思い出して以下のようにする。
つまり、$\pm 1$でない実数$x$に対して
$$
\dv{x}\qty(\frac{1}{2}\log\abs{\frac{1+x}{1-x}}) = \frac{1}{1-x^2}
$$
となる。

以上をまとめると各種関数の微分は以下のようになる。

\begin{proposition}[種々の関数の微分]
$x$を実数として以下が成立する。
$$
\dv{x}(c) = 0 \quad (c \in \mathbb{R}),
\quad \dv{x}(x^n) = n x^{n-1} \quad (n = 1, 2, 3, \cdots),
\quad \dv{x}(x^{-n}) = -n x^{-n-1} \quad (n = 1, 2, 3, \cdots, x \ne 0).
$$
$$
\dv{x}(a_0+a_1 x+a_2 x^2+a_3 x^3+\cdots+a_n x^n) = a_1+2 a_2 x+3 a_3 x^2+\cdots+n a_n x^{n-1} \quad (n = 0, 1, 2, 3, \cdots, a_0, \cdots, a_n \in \mathbb{R}).
$$
$$
\dv{x}(x|x|^{a-1}) = a |x|^{a-1} \quad (a \in \mathbb{R}, x \ne 0).
$$
$$
\dv{x}(\exp(x)) = \dv{x}(e^x) = e^x,
\quad \dv{x}(\exp_a(x)) = \dv{x}(a^x) = (\log a)a^x \quad (a > 0).
$$
$$
\dv{x}(\log|x|) = \frac{1}{x} \quad (x \ne 0),
\quad \dv{x}(\log_a|x|) = \frac{1}{(\log a)x} \quad (a > 0, a \ne 0, x \ne 0).
$$
$$
\dv{x}(\sin x) = \cos x,
\quad \dv{x}(\cos x) = -\sin x,
\quad \dv{x}(\tan x) = \frac{1}{\cos^2 x} \quad (x \ne \frac{\pi}{2}+n\pi, n \in \mathbb{Z}).
$$
$$
\dv{x}(\arcsin x) = \frac{1}{\sqrt{1-x^2}} \quad (-1 < x < +1),
\quad \dv{x}(\arccos x) = -\frac{1}{\sqrt{1-x^2}} \quad (-1 < x < +1),
\quad \dv{x}(\arctan x) = \frac{1}{1+x^2}.
$$
$$
\dv{x}(\sinh x) = \cosh x,
\quad \dv{x}(\cosh x) = \sinh x,
\quad \dv{x}(\tanh x) = \frac{1}{\cosh^2 x}.
$$
$$
\dv{x}(\arsinh x) = \frac{1}{\sqrt{1+x^2}},
\quad \dv{x}(\arcosh x) = \frac{1}{\sqrt{x^2-1}} \quad (|x| > 1),
\quad \dv{x}(\artanh x) = \frac{1}{1-x^2} \quad (-1 < x < +1).
$$
$$
\dv{x}\qty(\frac{1}{2}\log\abs{\frac{1+x}{1-x}}) = \frac{1}{1-x^2} \quad (x \ne \pm 1).
$$
$$
\dv{x}(\opabs(x)) = \dv{x}(|x|) = \sgn(x) \quad (x \ne 0),
\quad \dv{x}(\sgn(x)) = 0 \quad (x \ne 0).
$$
$$
\dv{x}(\floor(x)) = \dv{x}(\ceil(x)) = 0 \quad (x \notin \mathbb{Z}).
$$
\end{proposition}

\section{微分と増減}

\begin{definition}[極大・極小]
$f$を実数の集合$X$上の実数値関数とする。
\begin{itemize}
\item
$X$の内点$a$であって、ある正の数$\delta > 0$が存在して$f(a) = \max_{B_\delta(a)} f$となる$a \in X$のことを$f$の\emph{極大点}といい、$f$は点$a$で極大、またこの時の$f(a)$を$f$の\emph{極大値}という。
つまり、$a$を$f$の極大点とするとある正の数$\delta > 0$が存在して任意の$x \in B_\delta(a)$つまり$|x-a| < \delta$を満たす点$x \in X$に対して$f(x) \le f(a)$が成り立つ。
\item
$X$の内点$a$であって、ある正の数$\delta > 0$が存在して$f(a) = \min_{B_\delta(a)} f$となる$a \in X$のことを$f$の\emph{極小点}といい、$f$は点$a$で極小、またこの時の$f(a)$を$f$の\emph{極小値}という。
つまり、$a$を$f$の極小点とするとある正の数$\delta > 0$が存在して任意の$x \in B_\delta(a)$つまり$|x-a| < \delta$を満たす点$x \in X$に対して$f(x) \ge f(a)$が成り立つ。
\end{itemize}
\end{definition}

極大・極小は対象となる点の近くに限れば最大・最小となっている状況であり、関数に対して複数あることがある(ないこともある)。
また、最大点・最小点は必ず極大点・極小点である。

\begin{example}
関数$f(x) = 1-\frac{1}{2}x^2+\frac{1}{24}x^4$において、$x = 0$は極大点であるが、$\lim_{x \to \pm\infty}f(x) = +\infty$なので、最大点ではない。
\end{example}

次の定理は非常に重要である。

\begin{theorem}[最大値原理]
\label{t_maxp}
実数の集合$X$上の関数$f$が$X$の内点$a$で最大または極大または最小または極小とする。
ここで、$f$が$a$で微分可能であるとすると、$f'(a) = 0$が成立する。
\end{theorem}

\begin{proof}
$a$が極大点のときを考える。
つまり、ある$\delta > 0$が存在して任意の$x \in B_\delta(a)$に対して$f(x) \le f(a)$が成り立つとする。
この時
$$
\frac{f(x)-f(a)}{x-a} \le 0 \quad (a < x < a+\delta),
\quad \frac{f(x)-f(a)}{x-a} \ge 0 \quad (a-\delta < x < a).
$$
したがって、
\begin{equation}
\label{e_maxp_semi}
f'_+(a) \le 0,
\quad f'_-(a) \ge 0
\end{equation}
であり、$a$で微分可能なので命題\ref{t_semidiff}より、
$$
f'(a) = f'_+(a) = f'_-(a) = 0
$$
である。
最大点の時は極大点なので同じ証明でよい。
最小点、極小点の時は同様の議論をすることで\eqref{e_maxp_semi}の代わりに
\begin{equation}
\label{e_minp_semi}
f'_+(a) \ge 0,
\quad f'_-(a) \le 0
\end{equation}
が得られて同じ結論が導かれる。
\end{proof}

\begin{remark}
証明を見ればわかる通り微分可能でなくとも、関数$f$が極大点$a$で右微分可能ないし左微分可能とすると不等式\eqref{e_maxp_semi}が成り立ち、関数$f$が極小点$a$で右微分可能ないし左微分可能とすると不等式\eqref{e_minp_semi}が成り立つ。
\end{remark}

この定理が重要な理由は関数$f$の最大点・最小点を見つける問題が方程式$f'(x) = 0$を解くことに帰着されることにある。

\begin{example}
\label{t_gauss_est}
$\mathbb{R}$上の連続関数
$$
f(x) = x e^{-x}
$$
を考える。
この関数は$f(1) = e^{-1} > 0$, $\lim_{x \to -\infty}f(x) = -\infty$, $\lim_{x \to +\infty} = 0$なので、最大値を持つ。
ここで、微分導関数は$f'(x) = (1-x)e^{-x}$であり、最大点においてこれが$0$になるが、$x = 1$以外ありえない。
よって$f(x)$は$x = 1$で最大となることが結論付けられる。
\end{example}

次の不等式は一般に成り立つ。

\begin{theorem}[ヤングの不等式]
\label{t_young_ineq}
$a$, $b$を非負の実数、$p$, $q$を
\begin{equation}
\label{e_holder_conj}
\frac{1}{p}+\frac{1}{q} = 1, \quad p, q > 1
\end{equation}
を満たす実数とする。
このとき不等式
$$
a b \le \frac{a^p}{p}+\frac{b^q}{q}
$$
が成り立つ。
\end{theorem}

\begin{proof}
$x \in [0, \infty)$に対して
$$
f(x) = \frac{a^p}{p}+\frac{x^q}{q}-a x
$$
とおくと、$x \in (0, \infty)$に対して
$$
f'(x) = x^{q-1}-a.
$$
よって$f(x)$は$x = a^{\frac{1}{q-1}}$で最小となるので、
$$
f(b) \ge \frac{a^p}{p}+\frac{a^{\frac{q}{q-1}}}{q}-a^{1+\frac{1}{q-1}} = a^p\qty(\frac{1}{p}+\frac{1}{q}-1) = 0.
$$
したがってほしかった不等式が得られた。
\end{proof}

最大値原理は極大・極小の必要条件を与えるが、次の定理は十分条件の一つを片側極限を使って与える。

\begin{theorem}
$f$を実数の集合$X$上の関数、$a$を$X$の内点とする。
\begin{itemize}
\item
$f$が$a$で右微分可能かつ左微分可能で
$$
f'_+(a) < 0,
\quad f'_-(a) > 0
$$
を満たす時、関数$f$は点$a$で極大である。
\item
$f$が$a$で右微分可能かつ左微分可能で
$$
f'_+(a) > 0,
\quad f'_-(a) < 0
$$
を満たす時、関数$f$は点$a$で極小である。
\end{itemize}
\end{theorem}

\begin{proof}
\end{proof}

微分を使って関数の増減などの挙動を調べる際に基礎的になるのが次の平均値の定理である。

\begin{theorem}[平均値の定理]
有界閉区間$[a, b]$上の関数$f$が$[a, b]$上で連続で$(a, b)$上で微分可能とする。
この時、ある$a < c < b$が存在して
$$
f'(c) = \frac{f(b)-f(a)}{b-a}
$$
が成り立つ。
\end{theorem}

そしてその証明は次の特別な場合に示せばよい。

\begin{theorem}[ロルの定理]
有界閉区間$[a, b]$上の関数$f$が$[a, b]$上で連続で$(a, b)$上で微分可能とする。
ここで$f(a) = f(b)$の時、ある$a < c < b$が存在して$f'(c) = 0$が成り立つ。
\end{theorem}

\begin{proof}[ロルの定理の証明]
$f$は有界閉区間$[a, b]$上の連続関数より中間値の定理から最大点$c_+$と最小点$c_-$が存在する。
ここで、$f(c_+) = f(c_-)$の時は最大値と最小値が一致するので$f(x)$は定数関数で、各$x \in [a, b]$に対して$f(x) = f(a) = f(b)$かつ$x \in (a, b)$に対して$f'(x) = 0$である。
よって、この場合は$c = \frac{a+b}{2}$とすればよい。
そうでない場合は$c_+$か$c_-$のうち少なくとも一方$c$が$f(c) \ne f(a) = f(b)$より、$a < c < b$である。
さらに最大値原理(\ref{t_maxp})より$f'(c) = 0$なので、定理の証明が完成した。
\end{proof}

\begin{proof}[平均値の定理の証明]
関数$F$を
$$
F(x) = f(x)-\frac{f(b)-f(a)}{b-a}(x-a)
$$
とおくと、$F(a) = F(b) = f(a)$なので、ロルの定理が使えて$F'(c) = 0$となる$a < c < b$が存在する。
したがって、$f'(c) = \frac{f(b)-f(a)}{b-a}$である。
\end{proof}

定数関数の微分は常に$0$であるが、区間においては逆も成り立つ。
このことを平均値定理の応用として示す。

\begin{theorem}
区間$I$上の連続関数$f$がすべての内点$x$で$f'(x) = 0$を満たすならば、$f$は定数関数である。
\end{theorem}

\begin{proof}
$a < b$を満たす$I$の2点$a, b$を取る。
この時、$f$は$[a, b]$上連続で、$(a, b)$上微分可能なので、$f'(c) = \frac{f(b)-f(a)}{b-a}$となる点$c \in (a, b)$が存在する。
$c$は$I$の内点なので仮定より$f'(c) = 0$より$f(a) = f(b)$がわかる。
つまり任意の2点での$f$の値が等しいので、$f$は定数関数であることが結論付けられる。
\end{proof}

さらに詳しく見ると以下が成り立つ。

\begin{theorem}
$f$を区間$I$上の連続関数とする。
\begin{itemize}
\item
$f$がすべての内点$x$で$f'(x) > 0$を満たすならば、$f$は狭義単調増加である。
\item
$f$がすべての内点$x$で$f'(x) \ge 0$を満たすならば、$f$は広義単調増加である。
\item
$f$がすべての内点$x$で$f'(x) < 0$を満たすならば、$f$は狭義単調減少である。
\item
$f$がすべての内点$x$で$f'(x) \le 0$を満たすならば、$f$は広義単調減少である。
\end{itemize}
\end{theorem}

\begin{proof}
省略
\end{proof}

この定理を用いると、微分が$0$となる点で区間を分割することで関数の増減を知ることができる。

\begin{example}
$f(x) = 1-\frac{1}{2}x^2+\frac{1}{24}x^4$を考えると$f'(x) = -x+\frac{1}{6}x^3$で$f'(x) = 0$を解くと$x = 0, \pm\sqrt{6}$である。
したがってこの関数は$(-\infty, -\sqrt{6}]$で狭義単調減少し$-\sqrt{6}$で極小となり$[-\sqrt{6}, 0]$で狭義単調増加し$0$で極大となり$[0, +\sqrt{6}]$で狭義単調減少し$+\sqrt{6}$で極小となり$[+\sqrt{6}, +\infty)$で狭義単調増大する。
\end{example}

\section{ロピタルの定理}

この節では極限の問題を微分に帰着させるロピタルの定理を紹介する。
まず、その準備として次のコーシーの平均値の定理を示す。

\begin{theorem}[コーシーの平均値の定理]
$f$, $g$は有界閉区間$[a, b]$上の連続関数で$(a, b)$上で微分可能とする。
ここで$g(a) \ne g(b)$かつ任意の$a < x < b$に対して$g'(x) \ne 0$ならば、ある$a < c < b$が存在して
$$
\frac{f'(c)}{g'(c)} = \frac{f(b)-f(a)}{g(b)-g(a)}
$$
が成り立つ。
\end{theorem}

\begin{proof}
$g(a) \ne g(b)$に注意して関数$F$を
$$
F(x) = f(x)-\frac{f(b)-f(a)}{g(b)-g(a)}(g(x)-g(a))
$$
とおくと、$F(a) = F(b) = f(a)$なので、ロルの定理が使えて$F'(c) = 0$となる$a < c < b$が存在する。
したがって、$f'(c) = \frac{f(b)-f(a)}{g(b)-g(a)}g'(c)$であり、整理してほしかった等式を得る。
\end{proof}

\begin{theorem}[ロピタルの定理1]
$a$を開区間$I$の内点、$f$, $g$は$I\setminus\lrset{a}$上の連続関数で$I\setminus\lrset{a}$上で微分可能であり各$x \in I\setminus\lrset{a}$に対して$g'(x) \ne 0$とする。
ここで、$x \to a$で$f(x) \to 0$, $g(x) \to 0$であり、$\frac{f'(x)}{g'(x)}$が収束するならば、$\frac{f(x)}{g(x)}$も収束して
$$
\lim_{x \to a}\frac{f(x)}{g(x)} = \lim_{x \to a}\frac{f'(x)}{g'(x)}
$$
が成り立つ。
\end{theorem}

\begin{remark}
この定理は$a = \pm \infty$のときでも同様のものが成り立つ。
\end{remark}

\begin{proof}
$f(a) = 0$, $g(a) = 0$とおいて$f$, $g$の定義域を$I$に拡張すると$f$, $g$は$I$上の連続関数である。
ここで$[a, x] \subset I$を満たす点$x$を考え、$[a, x]$にコーシーの平均値の定理を用いれば、ある$a < c < x$が存在して
$$
\frac{f'(c)}{g'(c)} = \frac{f(x)-f(a)}{g(x)-g(a)} = \frac{f(x)}{g(x)}
$$
である。
このような$x$に対してこのような$c$の中から一つ選んで$c(x)$と定めると$a < c(x) < x$なので、$x \to a+$とすると$c(x) \to a$となることと$\lim_{x \to a}\frac{f'(x)}{g'(x)}$が存在することから、
$$
\lim_{x \to a+}\frac{f(x)}{g(x)} = \lim_{x \to a+}\frac{f'(c(x))}{g'(c(x))} = \lim_{x \to a+}\frac{f'(x)}{g'(x)}.
$$
同様にして$[x, a] \subset I$を満たす点$x$を考えることで$x \to a-$に対する式も得られて、この定理の証明が完成する。
\end{proof}

\begin{theorem}[ロピタルの定理2]
$a$を開区間$I$の内点、$f$, $g$は$I\setminus\lrset{a}$上の連続関数で$I\setminus\lrset{a}$上で微分可能であり各$x \in I\setminus\lrset{a}$に対して$g'(x) \ne 0$とする。
ここで、$x \to a$で$f(x) \to +\infty$, $g(x) \to +\infty$であり、$\frac{f'(x)}{g'(x)}$が収束するならば、$\frac{f(x)}{g(x)}$も収束して
$$
\lim_{x \to a}\frac{f(x)}{g(x)} = \lim_{x \to a}\frac{f'(x)}{g'(x)}
$$
が成り立つ。
\end{theorem}

\begin{remark}
この定理は$a = \pm \infty$のときでも同様のものが成り立つ。
\end{remark}

この定理の証明は難しく極限の定義つまり$\varepsilon$論法までさかのぼる必要がある。

\begin{proof}
% https://mathematics-pdf.com/pdf/lhopital.pdf
$\varepsilon > 0$を固定すると、$\delta > 0$が存在して任意の$a < x \le a+\delta$に対して$\abs{\frac{f'(x)}{g'(x)}-l} < \varepsilon$である。
ここで、$a < x < y < a+\delta$なる点$x, y \in I$を考え、$[x, y]$にコーシーの平均値を用いれば、ある$x < c < y$が存在して
$$
\frac{f'(c)}{g'(c)} = \frac{f(y)-f(x)}{g(y)-g(x)}
$$
つまり変形して、
$$
\frac{f(x)}{g(x)} = \frac{f'(c)}{g'(c)}\frac{1-\frac{g(y)}{g(x)}}{1-\frac{f(y)}{f(x)}}
$$
が成り立つ。
ここで、先に$x \to a+$とすることを考えると、$f(x) \to +\infty$, $g(x) \to +\infty$であることから、極限が存在すれば\todo{厳密にする}
$$
\lim_{x \to a}\frac{f(y)}{g(y)} = \lim_{x \to a+}\frac{f'(c(x, y))}{g'(c(x, y))}.
$$
次いで、$y \to a+$とすれば$c \to a+$より、
$$
\lim_{x \to a}\frac{f(y)}{g(y)} = \lim_{x \to a+}\frac{f'(x)}{g'(x)}.
$$
左極限についても同様にして、この定理の証明が完成する。
\end{proof}

\section{リプシッツ連続関数}

微分と結びついた連続性の概念としてリプシッツ連続性がある。
その定義は以下のとおりである。

\begin{definition}[リプシッツ連続関数]
$f$を区間$I$上の関数とする。
ここで、$f$に対して次を満たす実数定数$L$が存在するとき、$f$はリプシッツ連続あるいは$L$-リプシッツ連続という。
\begin{quote}
任意の$x, y \in I$に対して$|f(x)-f(y)| \le L|x-y|$が成り立つ。
\end{quote}
\end{definition}

\begin{example}
定数関数$f(x) = c$は$0$-リプシッツ連続であり、
一次関数$f(x) = x$は$1$-リプシッツ連続である。
しかしながら二次関数$f(x) = x^2$はリプシッツ連続でない。
これは$|f(x)-f(y)| = |x+y||x-y|$であり、$L$をいくら大きく設定しても$x, y$が大きいと$|x+y|$がそれを超えてしまうためである。
\end{example}

\begin{remark}
リプシッツ連続な関数はすべて連続関数である。
\end{remark}

次に示すようにリプシッツ連続であることの十分条件は微分導関数が有界であることである。

\begin{proposition}
$f$を区間$I$上の連続関数ですべての内点$x$で微分可能とする。
ここで、微分導関数$f'$が有界のとき、$f$はリプシッツ連続である。
より詳しくは実数$L$が存在して全ての内点$x$に対して$|f'(x)| \le L$のとき、$f$は$L$-リプシッツ連続である。
\end{proposition}

\begin{proof}
$x, y \in I$を取ると、中間値の定理より$|f(x)-f(y)| = |f'(c)||x-y|$となる$c$が存在するので、$|f(x)-f(y)| \le L|x-y|$である。
\end{proof}

\begin{example}
このではこのことをもとに$n = 1, 2, 3, \cdots$と$x, y \in \mathbb{R}$に対して不等式
$$
|\sin^n x-\sin^n y| \le |x^n-y^n|
$$
が成立することを示す。

まず$n = 1$の時は正弦関数$\sin$の$1$-リプシッツ連続性に他ならず、
$$
\abs{\dv{x}(\sin x)} = |\cos x| \le 1
$$
より成立する。

一般の$n$に対しては$x$, $y$をそれぞれ$x^{\frac{1}{n}}$, $y^{\frac{1}{n}}$で置き換えて$f(x) = \sin^n x^{\frac{1}{n}}$が$[0, +\infty)$で$1$-リプシッツ連続であることを示せばよい。
そこで微分導関数を計算すると$x \in (0, +\infty)$に対して、
$$
f'(x) = n\sin^{n-1}x^{\frac{1}{n}}\cos x^{\frac{1}{n}}\frac{1}{n}x^{\frac{1}{n}-1} = \qty(\frac{\sin x^{\frac{1}{n}}}{x^{\frac{1}{n}}})^{n-1}\cos x^{\frac{1}{n}}
$$
ここですでに示した$n = 1$の場合で$y = 0$とした時の$|\sin x| \le |x|$を用いれば、$|f'(x)| \le 1$がわかりほしかった不等式が示された。
\end{example}

\section{二階微分}

\section{凸関数}

\section{高階微分}

\section{テイラーの定理}

\section{漸近展開}


\chapter{積分}

\section{積分の導入}

有界閉区間$[a, b]$上の関数$f(x)$が与えられたときに、特に$f(x)$が常に正のとき、$x$軸と$y = f(x)$、$x = a$および$x = b$で囲まれた部分の「面積」を計算するにはどうすればよいだろうか。
この疑問から出発する数学の理論が\emph{積分}である。
もし$f$が定数関数ならば囲まれた部分は長方形になるので「たてかけるよこ」で面積が計算できる。
$f$が一次関数の場合は囲まれた部分は台形になるので台形の面積の公式が使えるだろう。
しかしながらより一般の関数の場合はたとえ二次関数であっても一筋縄ではいかないことがわかる。
これは面積(積分)を考える際にどうしても無限や極限の概念が必要になるためである。
以下で展開するのはこの問題に解決の方法を与えた理論で発案者の名を取ってリーマン積分と呼ばれる。
ちなみに積分の理論はリーマン積分の他に積分の定義の仕方によってルベーグ積分などいろいろあるが、ここでは歴史上最初の積分の厳密な定式化であるリーマン積分について紹介する。

面積を計算する時の基礎は先ほど述べた長方形の面積の公式「たてかけるよこ」であり、
これは$f$が定数関数の$f(x) = c$である場合$x$軸と$y = c$、$x = a$および$x = b$で囲まれた部分(長方形)の「面積」が$c(b-a)$で計算できることに対応する。
ここで、$c$が負の場合も$c(b-a)$は負の値として定義できることに注意する。

ここから一気に一般の関数$f(x)$を考える。
リーマン積分の基礎的な考え方は区間$[a, b]$を細かい区間に分割することである。

そのためにまず、区間$[a, b]$の分割を定式化する。
区間$[a, b]$の\emph{分割}$P$とは
$$
a = p_0 \le p_1 \le p_2 \le p_3 \le \cdots \le p_N = b
$$
を満たす点の並び$p_0, p_1, p_2, p_3, \cdots, p_N$のことである($N = 1, 2, 3, \cdots$)。
つまり、区間$[a, b]$を\emph{小区間}$[p_0, p_1], [p_1, p_2], [p_2, p_3], \cdots, [p_{N-1}, p_N]$に分割している。
また、\emph{分割の大きさ}を小区間の大きさの中で最も大きいもの、つまり
$$
\abs{P} = \max\lrset{ p_n-p_{n-1} \mid n = 1, 2, 3, \cdots, N }
$$
とする。

ここで小区間$[p_{n-1}, p_n]$ ($n = 1, 2, 3, \cdots, N$)において$x$軸と$y = f(x)$、$x = p_{n-1}$および$x = p_n$で囲まれた部分を考えると一般には長方形とは限らないが、分割の大きさが小さいと長方形とみなしても全体の面積を計算するのには差し支えないと考える。
そこで点$t_n \in [p_{n-1}, p_n]$をとりあえずは好きにとって、長方形の「たて」に相当するとして$f(t_n)$を採用し、(たてに細長の)長方形の面積を$f(t_n)(p_n-p_{n-1})$とする。
この時の$t_n$を小区間$[p_{n-1}, p_n]$の\emph{代表点}という。
これらの面積を足し合わせて得られる全体の面積
$$
\sum_{n = 1}^N f(t_n)(p_n-p_{n-1})
$$
を$f$の区間$[a, b]$の分割$p_0, p_1, p_2, p_3, \cdots, p_N$と代表点$t_1, t_2, t_3, \cdots, t_N$の\emph{リーマン和}という。

\emph{リーマン積分}はリーマン和の分割の大きさを$0$に近づけるときの極限として定義する。
より詳しくは分割$P$や代表点$t_n$の取り方によらず$\abs{P} \to 0$となるように分割の個数を$N \to \infty$とするときにリーマン和がある実数$l$に収束するならば関数$f$は区間$[a, b]$上で\emph{リーマン積分可能}といいその$l$を関数$f$の区間$[a, b]$上の\emph{定積分}といい$\int_a^b f(x)\dd{x}$と表す。
また、この時の関数$f$を\emph{被積分関数}という。
ただし、この表現だと「分割や代表点の取り方によらず」というあいまいな部分があり、上限・下限を用いたより正確で洗練された定義を後で導入する。

ひとまずはこれを定積分の定義として簡単な関数の積分を求めてみよう。
最初に考えるのは基礎的な関数である定数関数$f(x) = c$である。
この時、分割と代表点に対してリーマン和は
$$
\sum_{n = 1}^N f(t_n)(p_n-p_{n-1}) = \sum_{n = 1}^N c(p_n-p_{n-1}) = c\sum_{n = 1}^N (p_n-p_{n-1}) = c(p_N-p_0) = c(b-a)
$$
となり、分割や代表点の取り方によらない値になっているので、定積分は
$$
\int_a^b c\dd{x} = c(b-a)
$$
となり、長方形の面積の公式と等価である。

次に$f$が一次関数$f(x) = k x+m$のとき、いきなり一般の分割と代表点を考えるのは難しいので、
特殊な分割として区間$[a, b]$の$N = 1, 2, 3, \cdots$等分つまり$p_n = a+\frac{n}{N}(b-a)$ ($n = 0, 1, 2, 3, \cdots, N$)を代表点として小区間の右端点$t_n = p_n$ ($n = 1, 2, 3, \cdots, N$)をそれぞれ感がることにしよう。
この時、リーマン和は
$$
\begin{aligned}
\sum_{n = 1}^N f(t_n)(p_n-p_{n-1})
&= \frac{1}{N}(b-a)\sum_{n = 1}^N f(a+\frac{n}{N}(b-a))
= \frac{1}{N}(b-a)\sum_{n = 1}^N k(a+\frac{n}{N}(b-a))+m \\
% = \frac{1}{N}(b-a)\sum_{n = 1}^N k a+m+\frac{1}{N}k(b-a)n \\
&= \frac{1}{N}(b-a)\qty((k a+m)N+\frac{1}{N}k(b-a)\frac{1}{2}N(N+1)) \\
&= (b-a)\qty((k a+m)+\frac{1}{2}k(1+N^{-1})(b-a))
\end{aligned}
$$
となり、$N \to \infty$つまり分割を細かくすることを考えると、このリーマン和は
$$
(b-a)\qty((k a+m)+\frac{1}{2}k(b-a)) = \frac{1}{2}((k a+m)+(k b+m))(b-a)
$$
に収束し、これは台形の面積の公式と等価である。
ただし、これはあくまで特殊な分割と代表点に対してしか示していないので、まだこれが定積分とは言えない。
しかしながらリーマン積分可能である十分条件を後でいくつか紹介し、それによれば一次関数は常にリーマン積分可能で上記の値が定積分である。
また、上記の計算方法は一般のリーマン積分可能な関数に適用でき、特殊な分割や代表点の取り方にもとづいて定積分を計算することを\emph{区分求積法}という。
例えば、
$$
\int_a^b f(x)\dd{x} = \lim_{N \to \infty}\frac{1}{N}(b-a)\sum_{n = 1}^N f(a+\frac{n}{N}(b-a))
$$
とする計算法である。

二次関数や$n$次関数に対しても区分求積法により定積分が計算できるが、リーマン和を計算する段階で式が複雑になりがちである。
そこで定積分の計算を微分との兼ね合いで容易に計算する手法が知られていて、これによれば積分の計算は被積分関数の原始関数や不定積分と呼ばれる関数を見つけることに帰着される。
この時に原理的な役割を果たすのが微分積分学の基本定理と呼ばれる定理になる。
この計算方法は非常に強力で様々な関数の定積分が計算できるようになる。
そのため区分求積法を反対に使って今までの知識では扱えなかった極限の問題を積分の問題に帰着させることもできたりする。

また、これまで定積分は定義の都合上有界閉区間上の有界関数、例えば連続関数を主に考えていたが、
応用上重要なガンマ関数やベータ関数を定義するには開区間や有界でない区間で定積分を考える必要がある。
つまり端点までは連続でなく発散している場合や無限遠方を考えたい場合などがある。
このような要求に対する積分の理論が広義積分であり、定積分が定義される関数の範囲の拡張を試みる。

この章では以上の流れで説明を進めていく。

\section{上積分下積分と定積分}

$f(x)$を有界閉区間$[a, b]$上の有界な関数とする。
区間$[a, b]$の分割$P$を$a = p_0 \le p_1 \le p_2 \le p_3 \le \cdots \le p_N = b$とする。
この時、リーマン和の代わりに次の\emph{上リーマン和}と\emph{下リーマン和}を考える。
$$
\overline{S}(f, P) = \sum_{n = 1}^N \sup_{x \in [p_{n-1}, p_n]}f(x)(p_n-p_{n-1}),
\quad \underline{S}(f, P) = \sum_{n = 1}^N \inf_{x \in [p_{n-1}, p_n]}f(x)(p_n-p_{n-1}).
$$
$f$を有界な関数としているので、$\overline{S}(f, P)$, $\underline{S}(f, P)$は無限大にならず実数値になることに注意する。

\begin{proposition}
任意の分割$P$に対して、$\inf_{x \in [a, b]}f(x)(b-a) \le \underline{S}(f, P) \le \overline{S}(f, P) \le \sup_{x \in [a, b]}f(x)(b-a)$。
\end{proposition}

\begin{proof}
これは$\inf_{x \in [a, b]}f(x) \le \inf_{x \in [p_{n-1}, p_n]}f(x) \le \sup_{x \in [p_{n-1}, p_n]}f(x) \le \sup_{x \in [a, b]}f(x)$から自然に示される。
\end{proof}

ここから分割$P$を細かくすることを考える。
分割$P'$を$a = p'_0 \le p'_1 \le p'_2 \le p'_3 \le \cdots \le p'_M = b$として、$\lrset{ p_0, p_1, p_2, p_3, \cdots, p_N } \subset \lrset{ p'_0, p'_1, p'_2, p'_3, \cdots, p'_M }$となっている時、分割$P'$は分割$P$の\emph{細分}という。

\begin{proposition}
任意の分割$P$とその細分$P'$に対して、$\overline{S}(f, P') \le \overline{S}(f, P)$と$\underline{S}(f, P) \le \underline{S}(f, P')$が成り立つ。
\end{proposition}

\begin{proof}
まず、$\overline{S}(f, P) \le \sup_{x \in [a, b]}f(x)(b-a)$, $\inf_{x \in [a, b]}f(x)(b-a) \le \underline{S}(f, P)$から、$P$が区間を複数に分けないような分割の時に成り立っていることに注意する。
一般の場合では細分は小区間ごとに分割したものをまとめていることからそれらを足し合わせると主張が成り立つことがわかる。
\end{proof}

\begin{proposition}
任意の分割$\overline{P}$と$\underline{P}$に対して、$\underline{S}(f, \underline{P}) \le \overline{S}(f, \overline{P})$が成り立つ。
\end{proposition}

\begin{proof}
これは分割$\overline{P}$と$\underline{P}$の点をまとめて並び替えなおして得られる分割を$P'$とすると、$P'$は$\overline{P}$と$\underline{P}$の細分になることから、
今までに示した二つの命題より
$$
\underline{S}(f, \underline{P}) \le \underline{S}(f, P') \le \overline{S}(f, P') \le \overline{S}(f, \overline{P})
$$
が得られる。
\end{proof}

ここから$\overline{P}$を固定したまま$\underline{P}$についての上限を取ったのち$\overline{P}$についての下限を取ることを考える。
積分の候補として次の上積分と下積分
$$
\overline{S}(f) = \inf_P \overline{S}(f, P),
\quad \underline{S}(f) = \sup_P \underline{S}(f, P)
$$
を導入すると以上の議論から
$$
\underline{S}(f) \le \overline{S}(f)
$$
を得る。

\begin{definition}[定積分]
関数$f$が有界閉区間$[a, b]$上で\emph{積分可能}あるいは\emph{可積分}であるとは上記の式において等号が成り立つ場合をいい、その時の等しい値$\underline{S}(f) = \overline{S}(f)$を関数$f$の区間$[a, b]$上の\emph{定積分}といい
$$
\int_{[a, b]} f(x)\dd{x},
\quad \int_a^b f(x)\dd{x}
$$
などで表す。
また$a < b$に対して、
$$
\int_b^a f(x)\dd{x} = -\int_a^b f(x)\dd{x}
$$
と定義する。
\end{definition}

\begin{remark}
1点からなる区間の時は
$$
\int_a^a f(x)\dd{x} = 0
$$
であることに注意する。
\end{remark}

\begin{example}
導入でも触れた通り定数関数$f(x) = c$は任意の有界閉区間で積分可能で
$$
\int_a^b c\dd{x} = c(b-a)
$$
が成り立つ。
\end{example}

\begin{example}
有界な関数$f(x)$を$x \in [0, 1]$に対して$x$が有理数のとき$f(x) = 1$で$x$が無理数のとき$f(x) = 0$と定義する。
この時任意の小区間$[p, q]$ ($p < q$)に対して間に有理数と無理数が存在するので、$\overline{S}(f, P) = 1$と$\overline{S}(f, P) = 0$が成り立ち、この関数$f$は積分可能でない。
\end{example}

以下では積分可能であることの十分条件として主に単調関数と連続関数を紹介する。
まず、準備として次を示す。

\begin{proposition}
$f$を有界閉区間$[a, b]$上の有界な関数とする。
この時、$f$が$[a, b]$上で積分可能であることの必要十分条件は、
分割の列$(P_n)$であって$\abs{P_n} \to 0$かつ$\overline{S}(f, P_n)-\underline{S}(f, P_n) \to 0$を満たすものが存在することである。
\end{proposition}

\begin{proof}
まず、上積分下積分が下限上限で定義されていることから分割の列$(\overline{P}_n)$と$(\underline{P}_n)$であって$\overline{S}(f, \overline{P}_n) \to \overline{S}(f)$と$\underline{S}(f, \underline{P}_n) \to \underline{S}(f)$を満たすものが存在することに注意する。
ここで各$n$に対して$\overline{P}_n$と$\underline{P}_n$をまとめて並び替えなおして必要があれば分割をさらに細かくすることで$\abs{P_n} \to 0$となる分割の列$(P_n)$を作る。
この時$P_n$は$\overline{P}_n$と$\underline{P}_n$の細分になっているので、
$$
\overline{S}(f, P_n)-\underline{S}(f, P_n) \le \overline{S}(f, \overline{P}_n)-\underline{S}(f, \underline{P}_n) \to \overline{S}(f)-\underline{S}(f)
$$
が成り立つ。
よって、$f$が積分可能であるときほしかった分割の列が得られた。

逆に$P_n$が存在するとき、
$$
\overline{S}(f)-\underline{S}(f) \le \overline{S}(f, P_n)-\underline{S}(f, P_n) \to 0
$$
となっていることから、$f$は積分可能である。
\end{proof}

\begin{theorem}[単調関数の可積分性]
$f$を有界閉区間$[a, b]$上の広義単調増加または広義単調減少な関数とする。
この時、$f$は$[a, b]$上で積分可能である。
\end{theorem}

\begin{proof}
議論はほぼ同じなので、広義単調増加の場合のみ示す。
関数の単調性から$f$の小区間$[p_{n-1}, p_n]$での上限と下限は端点で達成するので、上リーマン和と下リーマン和は
$$
\overline{S}(f, P) = \sum_{n = 1}^N f(p_n)(p_n-p_{n-1}),
\quad \underline{S}(f, P) = \sum_{n = 1}^N f(p_{n-1})(p_n-p_{n-1})
$$
となりその差は
$$
\overline{S}(f, P)-\underline{S}(f, P) = \sum_{n = 1}^N (f(p_n)-f(p_{n-1}))(p_n-p_{n-1})
$$
となる。
これを評価すると
$$
\overline{S}(f, P)-\underline{S}(f, P) \le \sum_{n = 1}^N (f(p_n)-f(p_{n-1}))\abs{P} = (f(b)-f(a))\abs{P}
$$
であり、分割を細かくすると右辺はいくらでも小さくできるので、最終的に$f$が可積分であることがわかる。
\end{proof}

\begin{theorem}[連続関数の可積分性]
$f$を有界閉区間$[a, b]$上の連続関数とする。
この時、$f$は$[a, b]$上で積分可能である。
\end{theorem}

証明のためには有界閉区間上の連続関数は一様連続に更新できることを思い出す(定理\ref{t_unif_conti})。
また、ここでは連続性の度合いを使って一様連続性を書き下している。

\begin{proof}
$f$は有界閉区間$[a, b]$上の連続関数なので定理\ref{t_unif_conti}より$f$は$[a, b]$上の一様連続関数、つまり連続性の度合い$\omega$が存在し任意の$x, y \in [a, b]$に対して$|f(x)-f(y)| < \omega(|x-y|)$が成り立つ。
ここから小区間$[p, q] \subset [a, b]$に対して、$[p, q]$上の連続関数$f$の最大点$\bar{x}$と最小点$\var{y}$を取ると、
$$
\sup_{x \in [p, q]}f(x)-\inf_{x \in [p, q]}f(x) = f(\bar{x})-f(\bar{y}) \le \omega(|\bar{x}-\bar{y}|) \le \omega(q-p)
$$
である。
これを踏まえると上リーマン和と下リーマン和のその差は
$$
\overline{S}(f, P)-\underline{S}(f, P)
\le \sum_{n = 1}^N \omega(p_n-p_{n-1})(p_n-p_{n-1})
\le \sum_{n = 1}^N \omega(\abs{P})(p_n-p_{n-1})
= \omega(\abs{P})(b-a)
$$
となる。
よって、分割を細かくすると右辺は$0$に収束するので、最終的に$f$が可積分であることがわかる。
\end{proof}

以降では定積分の性質をいくつか述べる。

\begin{proposition}[積分の線形性]
区間$[a, b]$上で積分可能な関数$f$, $g$と実数の定数$c$に対して、
$$
\int_a^b (f(x)+g(x))\dd{x} = \int_a^b f(x)\dd{x}+\int_a^b f(x)\dd{x},
\quad \int_a^b c f(x)\dd{x} = c\int_a^b f(x)\dd{x}
$$
が成り立つ。
\end{proposition}

\begin{proof}
和の上限はそれぞれの上限を取ったものの和で上から抑えられることから、
$$
\overline{S}(f+g, P) = \sum_{n = 1}^N \sup_{x \in [p_{n-1}, p_n]}(f(x)+g(x))(p_n-p_{n-1}) \le \overline{S}(f, P)+\overline{S}(g, P).
$$
同様にして
$$
\underline{S}(f+g, P) = \sum_{n = 1}^N \inf_{x \in [p_{n-1}, p_n]}(f(x)+g(x))(p_n-p_{n-1}) \ge \underline{S}(f, P)+\underline{S}(g, P).
$$
したがって
$$
\underline{S}(f, P)+\underline{S}(g, P) \le \underline{S}(f+g, P) \le \overline{S}(f+g, P) \le \overline{S}(f, P)+\overline{S}(g, P)
$$
であり、$\underline{S}(f)+\underline{S}(g) \le \underline{S}(f+g) \le \overline{S}(f+g) \le \overline{S}(f)+\overline{S}(g)$が言える。
ここで、$f$と$g$が積分可能であることから$f+g$も積分可能でほしい式が成り立つことがわかる。
もう一つの定数倍の式は簡単に示されるので詳細は省略する。
\end{proof}

この命題では和と定数倍またそれを組み合わせて差の積分の公式が得られるが、関数の積や商については一般に定積分の公式はない。
代わりに後に述べる部分積分を用いて積分を計算することになる。

\begin{proposition}[積分の大小関係]
\label{t_int_order}
区間$[a, b]$上で積分可能な関数$f$, $g$が各$x \in [a, b]$に対して$f(x) \le g(x)$を満たすとき、
$$
\int_a^b f(x)\dd{x} \le \int_a^b g(x)\dd{x}
$$
が成り立つ。
特に区間$[a, b]$上で積分可能な関数$f$に対して、
$$
\abs{\int_a^b f(x)\dd{x}} \le \int_a^b \abs{f(x)}\dd{x}
$$
が成り立つ。
\end{proposition}

積分とは和のようなものだと思えば、和を取って絶対値を取るよりも絶対値の和を考えた方が大きいということで、自然に受け入れられるはずである。

\begin{proof}
下限や上限は大小関係を保つことから、$\overline{S}(f, P) \le \overline{S}(g, P)$が成り立ち、ここから$\overline{S}(f) \le \overline{S}(g)$なので、求めたい不等式が得られる。
また、$-\abs{f(x)} \le f(x) \le \abs{f(x)}$なので、もう一つの不等式もいえる。
\end{proof}

\begin{proposition}[積分の端点修正]
区間$[a, b]$上の関数$f$に対して、区間$[a, b]$上で積分可能な関数$\tilde{f}$であって各$x \in (a, b)$で$f(x) = \tilde{f}(x)$が成り立つものが存在したとするとき、
$f$も積分可能で
$$
\int_a^b f(x)\dd{x} = \int_a^b \tilde{f}(x)\dd{x}
$$
が成り立つ。
\end{proposition}

\begin{proof}
\end{proof}

\begin{proposition}[積分の区間]
\label{t_int_interval}
$a \le c \le b$として、関数$f$は区間$[a, c]$と区間$[c, b]$上で積分可能とする。
このとき、
$$
\int_a^b f(x)\dd{x} = \int_a^c f(x)\dd{x}+\int_c^b f(x)\dd{x}
$$
が成り立つ。
\end{proposition}

\begin{proof}
区間$[a, c]$の分割$P_1$と区間$[c, b]$の分割$P_2$を任意にとって、それらを合わせて得られる区間$[a, b]$の分割を$P$とおく。
この時、
$$
\overline{S}(f, P) = \overline{S}(f, P_1)+\overline{S}(f, P_2),
\quad \underline{S}(f, P) = \underline{S}(f, P_1)+\underline{S}(f, P_2)
$$
が成り立つ。
ここで$P_1$, $P_2$は任意の分割で$P$は特殊な分割なので、
$$
\overline{S}(f) \le \int_a^c f(x)\dd{x}+\int_c^b f(x)\dd{x},
\quad \underline{S}(f) \ge \int_a^c f(x)\dd{x}+\int_c^b f(x)\dd{x}
$$
となるからほしかった等式が得られる。
\end{proof}

\begin{remark}
この命題は積分の区間を分割できることを示していて例えば有限個の区間にわけて関数$f$がどの小区間でも積分可能であれば全体で積分可能である。
つまり、小区間で(端点で修正して)連続または単調になっていればよい。
\end{remark}

\section{微分積分学の基本定理}

微分と積分の間の関係性を示すのが微分積分学の基本定理である。
その表現のために必要になる概念が原始関数と不定積分である。
高校数学ではこの二つは同じようなものであったが、微分の理論で生じるのが原始関数で積分の理論で生じるのが不定積分であるという違いがある。

\begin{definition}[原始関数]
$f(x)$を開区間$I$上の関数とする。
$I$上の微分可能な関数$F(x)$であって各$x \in I$に対して
\begin{equation}
\label{e_prim}
F'(x) = f(x)
\end{equation}
が成り立つとき、$F$を$f$の\emph{原始関数}という。
\end{definition}

これに対して、定積分の端点を変数にして得られる関数が不定積分である。

\begin{definition}[不定積分]
$f(x)$を閉区間$I$上の積分可能な関数として、$a \in I$とする。
ここで$x \in I$に対して
\begin{equation}
\label{e_indef_int}
F(x) = \int_a^x f(t)\dd{t}
\end{equation}
とすることで定義される$I$上の関数$F$を$f$の\emph{不定積分}という。
\end{definition}

\emph{微分積分学の基本定理}は連続関数の不定積分はその関数の原始関数であることを主張する。

\begin{theorem}[微分積分学の基本定理]
$f$を開区間$I$上の連続関数とする。
\begin{itemize}
\item[(1)]
このとき$a \in I$に対して\eqref{e_indef_int}で定まる不定積分$F(x)$は$f(x)$の原始関数である、つまり\eqref{e_prim}が成り立つ。
\item[(2)]
$F$を$f$の原始関数の一つとすると任意の$a, b \in I$に対して、
$$
\int_a^b f(x)\dd{x} = F(b)-F(a)
$$
が成り立つ。
\end{itemize}
\end{theorem}

\begin{proof}
\begin{itemize}
\item[(1)]
$x \in I$に対して$F'(x)$を書き下すと命題\ref{t_int_interval}を使って、
$$
F'(x)
= \lim_{h \to 0}\frac{F(x+h)-F(x)}{h}
= \lim_{h \to 0}\frac{1}{h}\qty(\int_a^{x+h} f(t)\dd{t}-\int_a^x f(t)\dd{t})
= \lim_{h \to 0}\frac{1}{h}\int_x^{x+h} f(t)\dd{t}
$$
となり、$f$の$x$での連続性の度合い$\omega$を取ることで、
$$
\abs{\frac{1}{h}\int_x^{x+h} f(t)\dd{t}-f(x)}
= \abs{\frac{1}{h}\int_x^{x+h} (f(t)-f(x))\dd{t}}
\le \frac{1}{h}\int_x^{x+h} \abs{f(t)-f(x)}\dd{t}
\le \frac{1}{h}\int_x^{x+h} \omega(h)\dd{t}
= \omega(h)
$$
右辺は$h \to 0$で$0$に収束するので、結論として$F'(x) = f(x)$がわかる。
\item[(2)]
$f$の不定積分を$G$とおく、つまり
$$
G(x) = \int_a^x f(t)\dd{t}
$$
とすると(1)より$G$は$f$の原始関数である。
したがって、$G(x)-F(x)$は、微分導関数が$f(x)-f(x) = 0$より、定数関数なので、
$$
F(b)-F(a) = G(b)-G(a) = \int_a^b f(x)\dd{x}-\int_a^a f(x)\dd{x} = \int_a^b f(x)\dd{x}
$$
である。
\end{itemize}
\end{proof}

この微分積分学の基本定理を用いれば定積分を計算することは原始関数を一つ見つけることに帰着される。
また、$F(b)-F(a)$のことを
$$
\lreval{F}_a^b, \quad \lreval{F(x)}_{x = a}^b, \quad \lreval{F(x)}_a^b
$$
などと略記する。

\begin{example}
一次関数$f(x) = k x+m$に対して、二次関数$F(x) = \frac{k}{2}x^2+m x$は$f(x)$の原始関数であることがすぐわかるので、
$$
\int_a^b (k x+m)\dd{x} = \lreval*{\frac{k}{2}x^2+m x}_a^b = \frac{k}{2}(b^2-a^2)+m(b-a) = \frac{1}{2}((k a+m)+(k b+m))(b-a)
$$
と(区分求積法より)簡単に計算できる。
\end{example}

ここでは連続関数$f(x)$が与えられたときに$fの$すべての原始関数、つまり$f$の原始関数全体の集合について考えよう。
この集合を
$$
\int f(x)\dd{x}
$$
と書くことにする。
$F$, $G$を$f$の原始関数とすると、$G(x)-F(x)$は、微分導関数が$f(x)-f(x) = 0$より、定数関数になる。
つまり、$f$の原始関数$F$を一つ見つけたら原始関数全体の集合は
$$
\int f(x)\dd{x} = \lrset{ F(x)+C }
$$
とでき、この時の$C$を積分定数という。
上の記述はしばしば
$$
\int f(x)\dd{x} = F(x)+C
$$
と略記される。
積分定数もしばしば省略されるが、微分方程式の理論では積分定数は省略されないので、このテキストでは省略せずに書くことにする。

以下では微分の公式から直ちに得られる積分の公式を列挙する。
一部被積分関数が連続である範囲が実数全体でない場合もあるので注意する。

\begin{proposition}[種々の関数の積分1]
$$
\int x^n\dd{x} = \frac{1}{n+1}x^{n+1}+C \quad (n = 0, 1, 2, 3, \cdots),
\quad \int x^n\dd{x} = \frac{1}{n+1}x^{n+1}+C \quad (n = -2, -3, \cdots, x \ne 0),
$$
$$
\int |x|^a\dd{x} = \frac{1}{a+1}x|x|^a+C \quad (a \ne -1, x \ne 0).
$$
$$
\int e^x\dd{x} = e^x+C,
\quad \int a^x\dd{x} = \frac{1}{\log a}a^x+C \quad (a > 0, a \ne 1).
$$
$$
\int \frac{1}{x}\dd{x} = \log|x|+C \quad (x \ne 0),
$$
$$
\int \sin x\dd{x} = -\cos x+C,
\quad \int \cos x\dd{x} = \sin x+C,
\quad \int \frac{1}{\cos^2 x}\dd{x} = \tan x+C \quad (x \ne \frac{\pi}{2}+n\pi, n \in \mathbb{Z}).
$$
$$
\int \frac{1}{\sqrt{1-x^2}}\dd{x} = \arcsin x+C \quad (-1 < x < +1),
\quad \int \frac{1}{1+x^2}\dd{x} = \arctan x+C.
$$
$$
\int \sinh x\dd{x} = \cosh x+C,
\quad \int \cosh x\dd{x} = \sinh x+C,
\quad \int \frac{1}{\cosh^2 x}\dd{x} = \tanh x+C.
$$
$$
\int \frac{1}{\sqrt{1+x^2}}\dd{x} = \arsinh x+C,
\quad \int \frac{1}{1-x^2}\dd{x} = \frac{1}{2}\log\abs{\frac{1+x}{1-x}}+C \quad (x \ne \pm 1).
$$
\end{proposition}

\begin{remark}
上記では細かく書いたが、いくつかは簡単に
$$
\int x^a\dd{x} =
\begin{cases}
\frac{1}{a+1}x^{a+1}+C & \text{($a \ne -1$),} \\
\log|x|+C & \text{($a = -1$)} \\
\end{cases}
$$
と覚えておけばよい。
\end{remark}

また、被積分関数が単純な関数についての積分として次が挙げられる。

\begin{proposition}[種々の関数の積分2]
$$
\int \log |x|\dd{x} = x\log |x|-x+C \quad (x \ne 0).
$$
$$
\int \tan x\dd{x} = -\log|\cos x|+C \quad (x \ne \frac{\pi}{2}+n\pi, n \in \mathbb{Z}).
$$
$$
\int \tanh x\dd{x} = \log(\cosh x)+C.
$$
\end{proposition}

\begin{proof}
証明は右辺を微分することで確かめられるので省略する。
\end{proof}

\section{部分積分と置換積分}

まず、次が成り立つことに注意する。

\begin{proposition}
連続関数$f$, $g$に対して、
$$
\int (f(x)+g(x))\dd{x} = \int f(x)\dd{x}+\int f(x)\dd{x}
$$
が成り立つ。
\end{proposition}

積の微分や合成関数の微分に対応する積分の計算法が部分積分と置換積分である。

\begin{theorem}[部分積分]
$f(x)$, $g(x)$を$C^1$級関数とするとき、
$$
\int f(x)g'(x)\dd{x} = f(x)g(x)-\int f'(x)g(x)\dd{x}
$$
が成り立つ。
\end{theorem}

\begin{proof}
$F(x) = f(x)g(x)$とおくと積の微分より
$$
\dv{x}(F(x)) = f'(x)g(x)+f(x)g'(x).
$$
つまり$f(x)g(x)$は$f'(x)g(x)+f(x)g'(x)$の原始関数より
$$
\int f'(x)g(x)+f(x)g'(x)\dd{x} = f(x)g(x)+C.
$$
よって整理して示すべき式を得る。
\end{proof}

\begin{example}
部分積分により
$$
\int x e^x\dd{x} = \int x (e^x)'\dd{x} = x e^x-\int (x)' e^x\dd{x} = x e^x-\int e^x\dd{x} = (x-1)e^x+C.
$$
\end{example}

\begin{example}
部分積分により
$$
\int \log x\dd{x} = \int (x)'\log x\dd{x} = x\log x-\int x(\log x)'\dd{x} = x\log x-\int \dd{x} = x\log x-x+C.
$$
\end{example}

\begin{theorem}[置換積分]
$f(x)$を連続関数、$\phi(t)$を$C^1$級関数とするとき、$x = \phi(t)$とおくと、
$$
\int f(\phi(t))\phi'(t)\dd{t} = \int f(x)\dd{x}
$$
が成り立つ。
\end{theorem}

\begin{remark}
置換積分は$\dv{t}{x}\dd{t} = \dd{x}$と考えると覚えやすい。
\end{remark}

\begin{proof}
$F$を連続関数$f(x)$の原始関数(の一つ)とおくと合成関数の微分より
$$
\dv{t}(F(\phi(t))) = F'(\phi(t))\phi'(t) = f(\phi(t))\phi'(t).
$$
つまり$F(\phi(t))$は$f(\phi(t))\phi'(t)$の原始関数より、$x = \phi(t)$と置換することで
$$
\int f(\phi(t))\phi'(t)\dd{x} = F(\phi(t))+C = F(x)+C = \int f(x)\dd{x}.
$$
\end{proof}

\begin{example}
置換積分により$y = x^2$とおくと
$$
\int x e^{x^2}\dd{x} = \int \frac{1}{2}(x^2)' e^{x^2}\dd{x} = \int \frac{1}{2}e^y\dd{y} = \frac{1}{2}e^y+C = \frac{1}{2}e^{x^2}+C.
$$
\end{example}

部分積分と置換積分を定積分に適用するとき、部分積分はそのままだが置換積分は積分範囲に注意が必要である。

\begin{theorem}[定積分の部分積分]
$f(x)$, $g(x)$を有界閉区間$[a, b]$を含む開区間で$C^1$級関数とするとき、
$$
\int_a^b f(x)g'(x)\dd{x} = \lreval{f(x)g(x)}_a^b-\int_a^b f'(x)g(x)\dd{x}
$$
が成り立つ。
\end{theorem}

\begin{theorem}[定積分の置換積分]
$[\alpha, \beta]$を$t$の有界閉区間、$I$を$x$の区間とし、$f$を$I$上の連続関数、$x = \phi(t)$を$[\alpha, \beta]$を含む開区間で$C^1$級関数であって値域が$I$に入っているとする。
ここで$\phi(\alpha) = a$, $\phi(\beta) = b$とするとき、
$$
\int_\alpha^\beta f(\phi(t))\phi'(t)\dd{t} = \int_a^b f(x)\dd{x}
$$
が成り立つ。
\end{theorem}

部分積分と置換積分の応用として逆関数の積分がある。

\begin{theorem}[逆関数の積分]
$f$を連続関数として、$F$を$f$の原始関数の一つとする。
$f$が逆関数$f^{-1}$を持つとすると、
$$
\int f^{-1}(x)\dd{x} = x f^{-1}(x)-F(f^{-1}(x))+C
$$
が成り立つ。
\end{theorem}

\begin{proof}
$y = f^{-1}(x)$と置換すると、$x = f(y)$より
$$
\int f^{-1}(x)\dd{x} = \int y f'(y)\dd{y}.
$$
部分積分より
$$
\int f^{-1}(x)\dd{x} = y f(y)-\int f(y)\dd{y} = y f(y)-F(y)+C.
$$
よって、$f(f^{-1}(x)) = x$に注意して$x$で書き直せばほしかった式が得られる。
\end{proof}

\begin{example}
$f(x) = \exp(x) = e^x$を考えると原始関数は$F(x) = e^x = f(x)$、逆関数は$f^{-1}(x) = \log x$より
$$
\int \log x\dd{x} = \int f^{-1}(x)\dd{x} = x f^{-1}(x)-F(f^{-1}(x))+C = x\log x-x+C.
$$
\end{example}

\section{種々の積分の計算}

この章では積分を計算する系統だった技法をいくつか紹介することで様々な関数の積分が計算できるようにする。

実数係数の有理式つまり実数係数の多項式$p(x) \ne 0$と実数係数の多項式$q(x)$の分数の形$\frac{q(x)}{p(x)}$になっている関数は積分が計算できる。
これは部分分数分解によりどの有理式も次の形の有理式の和として表されることによる。

\begin{itemize}
\item[(i)]
単項式$a x^n$ ($a \in \mathbb{R}$, $n = 0, 1, 2, 3, \cdots$)。
\item[(ii)]
一次式の累乗$\frac{a}{(x-b)^n}$ ($a, b \in \mathbb{R}$, $n = 1, 2, 3, \cdots$)。
\item[(iii)]
判別式が負な二次式の累乗で次の形$\frac{2 a(x-b)}{((x-b)^2+c^2)^n}$ ($a, b, c \in \mathbb{R}$, $n = 1, 2, 3, \cdots$)。
\item[(iv)]
判別式が負な二次式の累乗$\frac{a}{((x-b)^2+c^2)^n}$ ($a, b, c \in \mathbb{R}$, $c \ne 0$, $n = 1, 2, 3, \cdots$)。
\end{itemize}

これらのうち(i)と(ii)は$x^n$, $x^{-n}$の積分を使えばよく、(iii)は$y = (x-b)^2+c^2$と置換することで$y^{-n}$の積分に帰着される。
(iv)は置換$y = \frac{x-b}{c}$により、
$$
I_n = \int \frac{1}{(x^2+1)^n}\dd{x}
$$
の計算に帰着され、部分積分からこの$I_n$が満たす漸化式
$$
I_{n+1} = \frac{x}{2 n(x^2+1)^n}+\frac{2 n-1}{2 n}I_n, \quad I_1 = \arctan x+C
$$
により計算できる(一般項は難しいが与えれた$n$に対する$I_n$は計算できる)。

\begin{example}
\end{example}

根号を含む関数$R(x, \sqrt[n]{\frac{a x+b}{c x+d}})$($R$は実数係数の有理式、$a, b, c, d \in \mathbb{R}$, $a d-b c \ne 0$, $n = 1, 2, 3, \cdots$)も積分が計算できる。
これは$y = \sqrt[n]{\frac{a x+b}{c x+d}}$と置換することで$y$についての有理式の積分に帰着されることによる。

\begin{example}
\end{example}

別の形の根号を含む関数$R(x, \sqrt{a x^2+b x+c})$も積分が計算できる。

\begin{example}
\end{example}

さらに$\sin x$, $\cos x$を含む関数$R(\sin x, \cos x)$($R$は実数係数の有理式)は$t = \tan(\frac{x}{2})$という置換により$t$についての有理式の積分に帰着される。

\begin{example}
\end{example}

特に$R(\sin^2 x, \cos^2 x)$($R$は実数係数の有理式)の形の関数の場合は$t = \tan x$という置換によりさらに簡単に積分が求められる。

\begin{example}
\end{example}

\section{積分の漸化式}

この節では被積分関数に自然数定数$n$があるような積分を考える。
もちろん$\int x^n\dd{x} = \frac{1}{n+1}x^{n+1}+C$のような簡単な例もあるが、一般項が簡単には求まらないことが普通である。
そこで部分積分や置換積分を使って漸化式を立てて目的の積分を計算するという技法が取られる。

\begin{example}[ウォリス積分]
$n = 0, 1, 2, 3, \cdots$に対して積分
$$
\int \sin^n x\dd{x}
$$
を考える。
この積分は部分積分を使うことにより$n \ge 2$に対して、
$$
\int \sin^n x\dd{x}
= \int (-\cos x)'\sin^{n-1} x\dd{x}
= -\cos x\sin^{n-1} x+\int (n-1)\cos x\sin^{n-2} x\cos x\dd{x}
= -\cos x\sin^{n-1} x+(n-1)\int \sin^{n-2} x\dd{x}-(n-1)\int \sin^n x\dd{x}
$$
となるので、積分の漸化式
$$
\int \sin^n x\dd{x} = -\frac{1}{n}\cos x\sin^{n-1} x+\frac{n-1}{n}\int \sin^{n-2} x\dd{x}
$$
を得る。
最初の$n = 0, 1$の場合だけ積分を計算したらこの漸化式から
\begin{align*}
&\int \sin^0 x\dd{x} = x+C, \\
&\int \sin^1 x\dd{x} = -\cos x+C, \\
&\int \sin^2 x\dd{x} = -\frac{1}{2}\cos x\sin x+\frac{1}{2}x+C, \\
&\int \sin^3 x\dd{x} = -\frac{1}{3}\cos x\sin^2 x-\frac{2}{3}\cos x+C
\end{align*}
などと順に計算することができる。

また、定積分
\begin{equation}
\label{e_wallis_int}
I_n = \int_0^{\frac{\pi}{2}} \sin^n x\dd{x} = \int_0^{\frac{\pi}{2}} \cos^n x\dd{x}
\end{equation}
の場合を考えよう。
なお、$\sin$の定積分と$\cos$の定積分が等しいことは$x$を$\frac{\pi}{2}-x$と置換すればわかる。
この積分$I_n$を\emph{ウォリス積分}という。
不定積分が得られているので定積分に対する漸化式は
$$
I_n = \frac{n-1}{n}I_{n-2} \quad (n \ge 2),
\quad I_0 = \frac{\pi}{2},
\quad I_1 = 1
$$
となる。
\end{example}

\begin{theorem}[ウォリスの公式]
\label{t_wallis_limit}
ウォリス積分\eqref{e_wallis_int}で定まる数列$(I_n)_{n = 0}^\infty$は
$$
\lim_{n \to \infty}\sqrt{n}I_n = \sqrt{\frac{\pi}{2}}
$$
\end{theorem}

\begin{proof}
ウォリス積分の漸化式より
$$
(n+1)I_{n+1}I_n = n I_n I_{n-1}
$$
なので、数列$(n I_n I_{n-1})_n$は定数列であり、
$$
n I_n I_{n-1} = I_1 I_0 = \frac{\pi}{2}
$$
がわかる。

ここで、$0 \le x \le \frac{\pi}{2}$で$\sin^{n+1} x \le \sin^n x \le \sin^{n-1} x$より
$$
I_{n+1} \le I_n \le I_{n-1}
$$
であり、$n I_n \ge 0$をかけて
$$
n I_{n+1}I_n = \frac{n}{n+1}\frac{\pi}{2} \le n I_n^2 \le n I_n I_{n-1} = \frac{\pi}{2}
$$
を得る。
したがってはさみうちの原理より結論の式を得る。
\end{proof}

\section{積分の不等式}

この節では積分を含んだいくつかの有名な不等式を紹介し、応用として(本来の応用先ではないが)積分が計算できない問題に対して評価を与える。

\begin{proposition}[連続関数に対する積分の大小関係]
$f(x)$, $g(x)$を有界閉区間$[a, b]$上の連続関数とする。
各$x \in [a, b]$に対して$f(x) \le g(x)$を満たすとき、
$$
\int_a^b f(x)\dd{x} \le \int_a^b g(x)\dd{x}
$$
が成り立つ。
等号成立条件は任意の$x \in [a, b]$に対して$f(x) = g(x)$が成り立つことである。
\end{proposition}

\begin{proof}
\end{proof}

\begin{theorem}[コーシー・シュワルツの不等式]
$f(x)$, $g(x)$を有界閉区間$[a, b]$上の連続関数とする。
このとき不等式
$$
\qty(\int_a^b f(x)g(x)\dd{x})^2 \le \qty(\int_a^b f(x)^2\dd{x})\qty(\int_a^b g(x)^2\dd{x})
$$
が成り立つ。
\end{theorem}

\begin{proof}
任意の実数$t$に対して
$$
\int_a^b (f(x)t+g(x))^2\dd{x} = \qty(\int_a^b f(x)^2\dd{x})t^2+2\qty(\int_a^b f(x)g(x)\dd{x})t+\qty(\int_a^b g(x)^2\dd{x}) \ge 0
$$
に注意する。
これは$t$についての二次不等式なので判別式が非正であり、ほしかった不等式を得る。
\end{proof}

\begin{example}
$[a, b] = [0, 1]$, $f(x) = \sqrt{1-x^4}$, $g(x) = 1$としてコーシー・シュワルツの不等式を用いれば、
$$
\int_0^1 \sqrt{1-x^4}\dd{x} \le \sqrt{\int_0^1 (1-x^4)\dd{x}} = \frac{2}{\sqrt{5}}
$$
を得る。
\end{example}

\begin{theorem}[ヘルダーの不等式]
$f(x)$, $g(x)$を有界閉区間$[a, b]$上の連続関数として、$p$, $q$を\eqref{e_holder_conj}を満たす実数とする。
このとき不等式
$$
\int_a^b |f(x)g(x)|\dd{x} \le \qty(\int_a^b |f(x)|^p\dd{x})^{\frac{1}{p}}\qty(\int_a^b |g(x)|^q\dd{x})^{\frac{1}{q}}
$$
が成り立つ。
\end{theorem}

\begin{proof}
$$
A = \qty(\int_a^b |f(x)|^p\dd{x})^{\frac{1}{p}},
\quad B = \qty(\int_a^b |g(x)|^q\dd{x})^{\frac{1}{q}}
$$
とおく。
$A = 0$のときは$f(x) = 0$、$B = 0$のときは$g(x) = 0$となり不等式は成立するので、$A > 0$かつ$B > 0$の場合を考える。
各$x \in [a, b]$に対してヤングの不等式(命題\ref{t_young_ineq})より
$$
\abs{\frac{f(x)}{A}\cdot\frac{g(x)}{B}} \le \frac{1}{p}\frac{\abs{f(x)}^p}{A^p}+\frac{1}{q}\frac{\abs{g(x)}^q}{B^q}
$$
なので、積分して
$$
\int_a^b \abs{\frac{f(x)}{A}\cdot\frac{g(x)}{B}}\dd{x} \le \frac{1}{p}+\frac{1}{q} = 1.
$$
よってほしかった不等式が得られる。
\end{proof}

\begin{remark}
$p = 2$, $q = 2$のヘルダーの不等式はコーシー・シュワルツの不等式に他ならない。
\end{remark}

\begin{remark}
$p = 1$に対応する場合は$q = \infty$となり、そのままでは考えられないが、
$$
\int_a^b |f(x)g(x)|\dd{x} \le \qty(\int_a^b |f(x)|\dd{x})\max_{x \in [a, b]}|g(x)|
$$
が対応する不等式と考えることができる。
\end{remark}

\begin{theorem}[ミンコフスキーの不等式]
$f(x)$, $g(x)$を有界閉区間$[a, b]$上の連続関数として、$p$を$1$以上の実数とする。
このとき不等式
$$
\qty(\int_a^b |f(x)+g(x)|^p\dd{x})^{\frac{1}{p}} \le \qty(\int_a^b |f(x)|^p\dd{x})^{\frac{1}{p}}+\qty(\int_a^b |g(x)|^p\dd{x})^{\frac{1}{p}}
$$
が成り立つ。
\end{theorem}

\begin{proof}
途中で三角不等式を一回使うことで
$$
\begin{aligned}
\int_a^b |f(x)+g(x)|^p\dd{x}
&= \int_a^b |f(x)+g(x)||f(x)+g(x)|^{p-1}\dd{x} \\
% \le \int_a^b (|f(x)|+|g(x)|)|f(x)+g(x)|^{p-1}\dd{x}
&\le \int_a^b |f(x)||f(x)+g(x)|^{p-1}\dd{x}+\int_a^b |g(x)||f(x)+g(x)|^{p-1}\dd{x}
\end{aligned}
$$
を得る。
$p = 1$のときはすでに結論を得ていることに注意して、以降では$p > 1$の場合を考える。
ここで、
$$
A = \qty(\int_a^b |f(x)|^p\dd{x})^{\frac{1}{p}},
\quad B = \qty(\int_a^b |g(x)|^p\dd{x})^{\frac{1}{p}}
$$
とおき、
$q$を\eqref{e_holder_conj}を満たす実数つまり$q = \frac{p}{p-1}$として取ると、ヘルダーの不等式から
$$
\begin{aligned}
\int_a^b |f(x)+g(x)|^p\dd{x}
&\le A\qty(\int_a^b |f(x)+g(x)|^{(p-1)q}\dd{x})^{\frac{1}{q}}+B\qty(\int_a^b |f(x)+g(x)|^{(p-1)q}\dd{x})^{\frac{1}{q}} \\
&= (A+B)\qty(\int_a^b |f(x)+g(x)|^p\dd{x})^{\frac{p-1}{p}}
\end{aligned}
$$
よってほしかった不等式が得られる。
\end{proof}

\begin{theorem}[イェンセンの不等式]
\end{theorem}

\section{積分型の平均値の定理}

\section{区分求積法の応用}

\section{広義積分}

有界閉区間とは限らない区間上の連続関数あるいは単調関数(広義単調増加関数と広義単調減少関数)に対する定積分である広義積分は区間を有界閉区間で近似することで定義される。
このとき、極限の有無によって広義積分の可能性が分かれることに注意する。

\begin{definition}[広義積分]
$f$を区間$I$上の連続関数あるいは単調関数とする。
\begin{itemize}
\item
$I = [a, b)$($b$は実数または正の無限大)の場合、極限
$$
\lim_{t \to b-}\int_a^t f(x)\dd{x}
$$
が存在するとき$f$は$[a, b)$上で\emph{広義積分可能}であるまたは\emph{広義積分が収束する}といい、その極限を$f$の$[a, b)$上の\emph{広義積分}という。
\item
$I = (a, b]$($a$は実数または負の無限大)の場合、極限
$$
\lim_{t \to a+}\int_t^b f(x)\dd{x}
$$
が存在するとき$f$は$(a, b]$上で\emph{広義積分可能}であるまたは\emph{広義積分が収束する}といい、その極限を$f$の$(a, b]$上の\emph{広義積分}という。
\item
$I = (a, b)$($a$は実数または負の無限大、$b$は実数または正の無限大)の場合、$a < c < b$に対して極限
$$
\lim_{t \to a+}\int_t^c f(x)\dd{x}, \quad \lim_{t \to b-}\int_c^t f(x)\dd{x} 
$$
の両方が存在するとき$f$は$(a, b)$上で\emph{広義積分可能}であるまたは\emph{広義積分が収束する}といい、その極限の和を$f$の$(a, b)$上の\emph{広義積分}という。
\end{itemize}
ここで、$f$の$I = [a, b), (a, b], (a, b)$上の広義積分を
$$
\int_I f(x)\dd{x},
\quad \int_a^b f(x)\dd{x}
$$
で表す。
\end{definition}

\begin{remark}
開区間$I = (a, b)$において、広義積分可能かどうかや広義積分の値は$c$の取り方によらない。
\end{remark}

次の命題は広義積分の理論において重要であり、指数が$a = -1$のところで切り替わることは記憶するに値する。

\begin{proposition}
$a$を実数とするとき、
$$
\int_1^\infty x^a\dd{x} =
\begin{cases}
+\infty & (a \ge -1), \\
\frac{1}{-a-1} & (a < -1).
\end{cases}
$$
$$
\int_0^1 x^a\dd{x} =
\begin{cases}
\frac{1}{a+1} & (a > -1), \\
+\infty & (a \le -1).
\end{cases}
$$
\end{proposition}

\begin{proof}
前半は$a \ne -1$のとき$t \to \infty$で、
$$
\int_1^t x^a\dd{x} = \lreval*{\frac{1}{a+1}x^{a+1}}_1^t = \frac{1}{a+1}t^{a+1}-\frac{1}{a+1} \to
\begin{cases}
+\infty & (a > -1), \\
\frac{1}{-a-1} & (a < -1).
\end{cases}
$$
また、$a = -1$のときは
$$
\int_1^t x^{-1}\dd{x} = \lreval*{\log x}_1^t = \log t \to +\infty.
$$
後半は$a \ne -1$のとき$t \to 0+$で、
$$
\int_t^1 x^a\dd{x} = \lreval*{\frac{1}{a+1}x^{a+1}}_t^1 = \frac{1}{a+1}-\frac{1}{a+1}t^{a+1} \to
\begin{cases}
\frac{1}{a+1} & (a > -1). \\
+\infty & (a < -1).
\end{cases}
$$
また、$a = -1$のときは
$$
\int_t^1 x^{-1}\dd{x} = \lreval*{\log x}_t^1 = -\log t \to +\infty.
$$
以上より示された。
\end{proof}

以下では広義積分の値を追求するよりも収束するかどうかについて議論する。

\begin{definition}[広義積分の絶対収束]
区間$I$上の連続関数あるいは単調関数$f$に対して広義積分
$$
\int_I |f(x)|\dd{x}
$$
が収束する時、$f$は$I$上で\emph{絶対広義積分可能}または広義積分が\emph{絶対収束}するという。
\end{definition}

\begin{remark}
広義積分が絶対収束する場合
$$
\int_I |f(x)|\dd{x} < \infty
$$
と表し、
絶対収束しない場合
$$
\int_I |f(x)|\dd{x} = \infty
$$
と表す。
\end{remark}

\begin{remark}
広義積分が絶対収束するならばそれは収束している。
\end{remark}

\begin{proposition}[広義積分の比較判定法]
$f$を区間$I$上の連続関数あるいは単調関数とする。
\begin{itemize}
\item
二つの条件
$$
|f(x)| \le g(x) \quad \forall x \in I,
\quad \int_I g(x) < \infty
$$
を満たす$I$上の非負値の連続関数あるいは単調関数$g$が存在するならば、
$f$の$I$上での広義積分は絶対収束し、
$$
\abs{\int_I f(x)\dd{x}} \le \int_I \abs{f(x)}\dd{x} \le \int_I g(x)\dd{x}
$$
が成り立つ。
\item
二つの条件
$$
|f(x)| \ge g(x) \quad \forall x \in I,
\quad \int_I g(x) = \infty
$$
を満たす$I$上の非負値の連続関数あるいは単調関数$g$が存在するならば、
$f$の$I$上での広義積分は絶対収束せず、
$$
\int_I \abs{f(x)}\dd{x} = \infty
$$
である。
\end{itemize}
\end{proposition}

\begin{remark}
この時の非負値関数$g$を$f$の\emph{優関数}という。
\end{remark}

\begin{example}
広義積分
$$
\int_0^\infty e^{-x^2}\dd{x}
$$
は絶対収束する。
実際
$$
e^{-x^2} \le \min\lrset{ \frac{1}{e x^2}, 1 }
$$
であることから従う($e^{-x^2} \le e^{-1}x^{-2}$については例\ref{t_gauss_est}参照)。
\end{example}

\begin{example}
広義積分
$$
\int_0^\infty \frac{\sin x}{x}\dd{x}
$$
は収束するが、絶対収束しない。
\end{example}


\chapter{級数}

\section{級数の収束}

数列$(a_n)_{n = 0}^\infty$の項$a_0, a_1, a_2, a_3, \cdots$の和について考える。
このような和のことを級数と呼び、形式的には
$$
\sum_{n = 0}^\infty a_n = a_0+a_1+a_2+a_3+\cdots
$$
と表すが、無限個の項の和なのでしっかり定義する必要がある。

\begin{definition}[級数]
数列$(a_n)_n$と$N = 0, 1, 2, 3, \cdots$に対して、初項から項$a_N$までの\emph{部分和}
$$
A_N = \sum_{n = 0}^N a_n = a_0+a_1+a_2+a_3+\cdots+a_N
$$
を定義する。
ここで部分和を並べて得られる数列$(A_N)_N$がある数$S$に収束する時、
\emph{級数}$\sum_n a_n$は収束するといい、この時の極限$S$を数列$(a_n)_n$の級数の値または\emph{和}と呼び$\sum_n a_n$で表す。
また、部分和の数列$(A_N)_N$が発散するとき、級数$\sum_n a_n$は発散するという。
級数の表記は
$$
\sum_{n = 0}^\infty a_n, \quad \sum_{n = 1}^\infty a_n, \quad \sum_{n \in \mathbb{N}}a_n, \quad \sum_n a_n, \quad \sum a_n
$$
などがある。
\end{definition}

\begin{example}
$c$を実数として等比数列$(c^n)_{n = 0}^\infty$を考える。
対応する級数
$$
\sum_{n = 0}^\infty c^n = 1+c+c^2+c^3+\cdots
$$
を等比級数という。
$c \ne 1$の時、部分和は
$$
\sum_{n = 0}^N c^n = \frac{c^{N+1}-1}{c-1} = \frac{1-c^{N+1}}{1-c}
$$
であり、$-1 < c < 1$のとき収束し級数の値は
$$
\sum_{n = 0}^\infty c^n = \frac{1}{1-c}
$$
である。
それ以外の$c$の場合は発散する。
$c = 1$の場合も$\sum_{n = 0}^N c^n = N$より発散する。
\end{example}

次の命題は級数が収束する必要条件を与える。

\begin{proposition}
\label{t_ser_conv}
数列$(a_n)$に対して、級数$\sum a_n$が収束するならば数列$(a_n)$は$0$に収束する。
\end{proposition}

\begin{proof}
各$n$に対して、
$$
a_{n+1} = A_{n+1}-A_n
$$
が成り立つ。
級数$\sum a_n$が収束するため、$(A_n)$と$(A_{n+1})$はともに和$S$に収束することから、数列$(a_{n+1})$ひいては$(a_n)$は$S-S = 0$に収束する。
\end{proof}

\begin{remark}
\label{t_harm_ser}
この命題の逆は成立しない。
例えば$n = 1, 2, 3, \cdots$に対して$a_n = \frac{1}{n}$とおいた級数
$$
\sum_{n = 1}^\infty \frac{1}{n} = \frac{1}{1}+\frac{1}{2}+\frac{1}{3}+\cdots
$$
は収束しない。
実際、
$$
A_{2^2} = 1+\frac{1}{2}+\frac{1}{3}+\frac{1}{4} > 1+\frac{1}{2}+\frac{1}{4}+\frac{1}{4} = 1+\frac{1}{2}+\frac{1}{2},
$$
$$
A_{2^3} = 1+\frac{1}{2}+\frac{1}{3}+\frac{1}{4}+\frac{1}{5}+\frac{1}{6}+\frac{1}{7}+\frac{1}{8} > 1+\frac{1}{2}+\frac{1}{4}+\frac{1}{4}+\frac{1}{8}+\frac{1}{8}+\frac{1}{8}+\frac{1}{8} = 1+\frac{1}{2}+\frac{1}{2}+\frac{1}{2}
$$
で、
続けると
$$
A_{2^k} > 1+\frac{k}{2}
$$
がわかり、これは収束しない。
\end{remark}

級数(数列)の中でも特別なのが次の正項級数である。

\begin{definition}[正項級数]
数列$(a_n)$が各自然数$n$に対して$a_n \ge 0$を満たすとき、級数$\sum a_n$を\emph{正項級数}という。
\end{definition}

\begin{proposition}
正項級数$\sum a_n$が収束するための必要十分条件は部分和の数列$(A_N)$が上に有界であることである。
\end{proposition}

\begin{proof}
$\sum a_n$が正項級数であることから、各自然数$N$に対して
$$
A_{N+1}-A_N = a_{N+1} \ge 0
$$
より、数列$(A_N)$は単調増加である。
したがって$(A_N)$が上に有界ならば、単調収束定理(定理\ref{t_mono_conv})より、収束する。
逆は命題\ref{t_conv_bdd}より従う。
\end{proof}

このため正項級数の場合に限って級数$\sum a_n$が収束することを
$$
\sum a_n < \infty
$$
と表し、発散することを
$$
\sum a_n = \infty
$$
と表す。

\begin{definition}[級数の絶対収束と条件収束]
級数$\sum a_n$に対して正項級数$\sum |a_n|$が収束するとき、級数$\sum a_n$は\emph{絶対収束}するという。
絶対収束しないが収束するとき、級数は\emph{条件収束}するという。
\end{definition}

\begin{proposition}
級数$\sum a_n$が絶対収束するならばそれは収束している。
\end{proposition}

\begin{proof}
$N \in \mathbb{N}$に対して、
$$
S_N = \sum_{n \le N} a_n,
\quad \bar{S}_N = \sum_{n \le N} \abs{a_n}
$$
とおくと、
$$
\abs{S_N-S_M} = \abs{\sum_{n = M+1}^N a_n} \le \abs{\sum_{n = M+1}^N \abs{a_n}} = \abs{\bar{S}_N-\bar{S}_M}.
$$
よって、$N$について上極限を$M$について下極限をそれぞれ取ると$\bar{S}_N$, $\bar{S}_M$は仮定より収束することに注意して、
$$
\limsup_{N \to \infty}S_N-\liminf_{N \to \infty}S_N \le \lim_{N \to \infty}\bar{S}_N-\lim_{N \to \infty}\bar{S}_N = 0.
$$
したがって上極限と下極限が一致したので$S_N$さらには級数は収束する。
\end{proof}

\begin{remark}
級数が絶対収束するならばそれは収束している。
\end{remark}

ここまでくると級数の理論は広義積分の理論と似ていることに気が付く。
つまり、級数の収束は部分和の極限で定義していることは広義積分の積分区間を有界閉区間で近似していたことに対応するし、
級数でも広義積分でも絶対収束が登場する。
実際次の命題は級数版の比較判定法である。

\begin{proposition}[級数の比較判定法]
$(a_n)$を数列とする。
\begin{itemize}
\item
二つの条件
$$
|a_n| \le b_n \quad \forall n,
\quad \sum b_n < \infty
$$
を満たす正項級数$\sum b_n$が存在するならば、
級数$\sum a_n$は絶対収束し、
$$
\abs{\sum a_n} \le \sum \abs{a_n} \le \sum b_n
$$
が成り立つ。
\item
二つの条件
$$
|a_n| \ge b_n \quad \forall n,
\quad \sum b_n = \infty
$$
を満たす正項級数$\sum b_n$が存在するならば、
級数$\sum a_n$は絶対収束せず、
$$
\sum \abs{a_n} = \infty
$$
である。
\end{itemize}
\end{proposition}

級数を等比級数と比較することで収束性をある程度判定することができる。

\begin{theorem}[ダランベールの判定法]
$(a_n)$をどの項も$0$でない数列とする。
極限
$$
\lim_{n \to \infty}\abs{\frac{a_{n+1}}{a_n}} = c
$$
が存在し、$c < 1$ならば級数$\sum a_n$は絶対収束し、$c > 1$ならば級数$\sum a_n$は発散する。
\end{theorem}

\begin{theorem}[コーシーの判定法]
$(a_n)$を数列とする。
上極限
$$
\limsup_{n \to \infty}\sqrt[n]{\abs{a_n}} = c
$$
について、$c < 1$ならば級数$\sum a_n$は絶対収束し、$c > 1$ならば級数$\sum a_n$は発散する。
\end{theorem}

\begin{remark}
どちらの判定法も$c = 1$の場合は級数が収束するか発散するかわからない。
\end{remark}

\begin{example}
級数
$$
\sum_{n = 0}^\infty \frac{n}{2^n} = \frac{0}{1}+\frac{1}{2}+\frac{2}{2^2}+\frac{3}{2^3}+\cdots
$$
は(絶対)収束する。
実際、$a_n = \frac{n}{2^n}$とおくと、$n \to \infty$で
$$
\abs{\frac{a_{n+1}}{a_n}} = \frac{n+1}{n}\frac{2^n}{2^{n+1}} = \frac{n+1}{n}\frac{1}{2} \to \frac{1}{2} < 1
$$
より、ダランベールの判定法が使えて、収束する。
\end{example}

\begin{example}
級数
$$
\sum_{n = 1}^\infty \frac{1}{n^2} = \frac{1}{1^2}+\frac{1}{2^2}+\frac{1}{3^2}+\cdots
$$
ではダランベールの判定法とコーシーの判定法はともに$c = 1$となってしまい、級数が収束するか発散するか判定できない。
なお、実際にはすぐ次でわかる通り収束する。
\end{example}

このような級数に対しては次の定理で広義積分の判定に帰着させるとよい。

\begin{theorem}[級数と広義積分の比較]
$f$を$[0, +\infty)$上定義された単調減少する非負値の関数とする。
このとき級数
$$
\sum_{n = 0}^\infty f(n)
$$
が(絶対)収束することと広義積分
$$
\int_0^\infty f(x)\dd{x}
$$
が(絶対)収束することは同値である。
\end{theorem}

\begin{proof}
$[0, +\infty)$上の関数$\overline{g}(x)$と$\underline{g}(x)$を
$$
\overline{g}(x) = f(\lrfloor{x}),
\quad \underline{g}(x) = f(\lrceil{x})
$$
で定義すると、$f$は単調減少することから$\overline{g}(x)$と$\underline{g}(x)$は
$$
\underline{g}(x) \le f(x) \le \overline{g}(x)
$$
を満たす単調減少関数である。
またここで、
$$
\sum_{n = 0}^\infty f(n) = \int_0^\infty \overline{g}(x)\dd{x}
$$
もわかるので、
$$
\sum_{n = 1}^\infty f(n) \le \int_0^\infty f(x)\dd{x} \le \sum_{n = 0}^\infty f(n)
$$
が従い、級数と広義積分の比較判定ができる。
\end{proof}

例えば次のことがわかる。

\begin{proposition}
$a$を実数とするとき、
$$
\sum_{n = 0}^\infty n^a
\begin{cases}
= \infty & (a \ge -1), \\
< \infty & (a < -1).
\end{cases}
$$
\end{proposition}

\section{交代級数と条件収束}

各項の符号が常に正(正確には非負)なのが正項級数であったが、項の符号が順番に入れ替わっていく級数を考える。

\begin{definition}[交代級数]
数列$(a_n)$が各自然数$n$に対して
$$
a_n a_{n+1} \le 0
$$
を満たすとき、級数$\sum a_n$を\emph{交代級数}という。
\end{definition}

絶対収束しない場合でも交代級数であれば収束することがある。
次の定理はある種の交代級数の収束性を保証する。

\begin{theorem}[ライプニッツの判定法]
数列$(a_n)_{n = 0}^\infty$が単調減少し$n \to \infty$で$0$に収束するとする。
このとき、交代級数$\sum_{n = 0}^\infty (-1)^n a_n$は収束し、各$N = 0, 1, 2, 3, \cdots$に対して
$$
\abs{\sum_{n = 0}^N (-1)^n a_n-\sum_{n = 0}^\infty (-1)^n a_n} \le a_{N+1}
$$
を満たす。
\end{theorem}

\begin{proof}
$N = 0, 1, 2, 3, \cdots$に対して部分和を$A_N = \sum_{n = 0}^N (-1)^n a_n$とおく。
このとき、$K = 0, 1, 2, 3, \cdots$に対して
$$
A_{2(K+1)} = A_{2 K}-a_{2 K+1}+a_{2 K+2} \le A_{2 K},
$$
$$
A_{2(K+1)+1} = A_{2 K+1}+a_{2 K+2}-a_{2 K+3} \le A_{2 K+1}
$$
より
$$
A_1 \le A_3 \le A_5 \le A_7 \le \cdots \le A_6 \le A_4 \le A_2 \le A_0
$$
がわかる。
したがって単調収束定理より$(A_{2 K})_K$と$(A_{2 K+1})_K$は収束し極限をそれぞれ$\overline{L}$と$\underline{L}$とおくと$\underline{L} \le \overline{L}$である。
また、ここで
$$
A_{2 K}-A{2 K+1} = a_{2 K} \to 0
$$
より$\overline{L}-\underline{L} \le 0$つまり$\overline{L} = \underline{L} = L$である。
あとは$|A_N-L| \le a_{N+1}$を示す。
\begin{itemize}
\item
$N = 2 K$の場合、
$$
|A_N-L| = A_{2 K}-L \le A_{2 K}-A_{2 K+1} = a_{2 K+1} = a_{N+1}.
$$
\item
$N = 2 K+1$の場合、
$$
|A_N-L| = L-A_{2 K+1} \le A_{2 K+2}-A_{2 K+1} = a_{2 K+2} = a_{N+1}.
$$
\end{itemize}
以上より示すべき不等式が示されて、直ちに$A_N \to L$つまり交代級数が収束することが示される。
\end{proof}

\begin{example}
\label{t_alt_harm_ser}
交代級数
$$
\sum_{n = 1}^\infty \frac{(-1)^{n-1}}{n} = \frac{1}{1}-\frac{1}{2}+\frac{1}{3}-\cdots
$$
は対応する正項級数$\sum_{n = 1}^\infty \frac{1}{n}$が収束しないので絶対収束しない(注意\ref{t_harm_ser}参照)。
一方で$a_n = \frac{1}{n}$はライプニッツの判定法の仮定を満たすので、収束することがわかる。
つまり、条件収束する。
\end{example}

交代級数に対するこの収束の判定法を一般化して次の収束の条件が知られている。

\begin{theorem}[ディリクレの判定法]
実数列$(a_n)_{n = 0}^\infty$は単調減少し$n \to \infty$で$0$に収束するとする。
数列$(b_n)_{n = 0}^\infty$は部分和の数列$(B_N)$が有界、つまりある実数$M$が存在し任意の$N = 0, 1, 2, 3, \cdots$に対して$\abs{\sum_{n = 0}^N b_n} \le M$が成り立つとする。
このとき級数$\sum_{n = 0}^\infty a_n b_n$は収束する。
\end{theorem}

\begin{proof}
まず、部分和を$S_N = \sum_{n = 0}^N a_n b_n$とおく。
ここで、部分和分(命題\ref{t_sum_part})より$B_{-1} = 0$と考えることで、
$$
S_N = \sum_{n = 0}^N a_n(B_n-B_{n-1}) = a_{N+1}B_N+\sum_{n = 0}^\infty (a_n-a_{n+1})B_n
$$
ここで$(B_N)$が有界で$a_N \to 0$より$a_{N+1}B_N \to 0$である。
また、仮定$a_{n+1} \le a_n$を使うと、
$$
\abs{(a_n-a_{n+1})B_n} \le (a_n-a_{n+1})\abs{B_n} \le (a_n-a_{n+1})M
$$
で
$$
\sum_{n = 0}^N (a_n-a_{n+1})M = (a_0-a_{N+1})M \le a_0 M < \infty
$$
なので比較判定法から$\sum_{n = 0}^\infty (a_n-a_{n+1})B_n$も収束し、$(S_N)$の収束も示された。
\end{proof}

\begin{example}
交代級数の例\ref{t_alt_harm_ser}の類似として級数
$$
\sum_{n = 1}^\infty \frac{\sin\frac{2\pi}{3}n}{n} = \frac{\frac{\sqrt{3}}{2}}{1}-\frac{\frac{\sqrt{3}}{2}}{2}+\frac{0}{3}+\cdots
$$
を考える。
$a_n = \frac{1}{n}$, $b_n = \sin\frac{2\pi}{3}n$とするとディリクレの判定法の仮定を満たすので、条件収束することがわかる。
一般に和$\sum_{n = M}^N \sin n\theta$は計算できる(後の章を参照)ことから、級数
$$
\sum_{n = 1}^\infty \frac{\sin n}{n} = \frac{\sin 1}{1}+\frac{\sin 2}{2}+\frac{\sin 3}{3}+\cdots
$$
などはやはり条件収束することがわかる。
\end{example}

\section{べき級数}

\begin{definition}[べき級数]
数列$(a_n)_{n = 0}^\infty$と数$a$と$x$に対して定まる次の形の級数
$$
\sum_{n = 0}^\infty a_n(x-a)^n = a_0+a_1(x-a)+a_2(x-a)^2+a_3(x-a)^3+\cdots
$$
を\emph{べき級数}または\emph{整級数}という。
\end{definition}

べき級数は単一の$x$のみ考えるなら特殊な形の級数であるが、重要な点は$x$を変数、べき級数を関数と見て議論を展開する点にある。

\begin{remark}
べき級数$\sum_{n = 0}^\infty a_n(x-a)^n$は$x = a$のとき$a_0$に必ず収束する。
\end{remark}

\begin{proposition}[べき級数の収束性]
べき級数$\sum_{n = 0}^\infty a_n(x-a)^n$が点$x = b \ne a$で収束するならば、$|x-a| < |b-a|$を満たすすべての点$x$でこのべき級数は絶対収束する。
\end{proposition}

\begin{proof}
級数$\sum_{n = 0}^\infty a_n b^n$が収束するので、命題\ref{t_ser_conv}より数列$(a_n(b-a)^n)_{n = 0}^\infty$は$0$に収束し、特に有界である。
そこである実数$M$が存在し、任意の$n = 0, 1, 2, 3, \cdots$に対して$|a_n(b-a)^n| \le M$である。
ここで
$$
|a_n(x-a)^n| = |a_n(b-a)^n|\abs{\frac{x-a}{b-a}}^n \le M\abs{\frac{x-a}{b-a}}^n
$$
であり、$|x-a| < |b-a|$であることから右辺から定まる正項級数は収束するので、
級数$\sum_{n = 0}^\infty a_n(x-a)^n$は絶対収束する。
\end{proof}

この命題をもとに各べき級数には固有な次の値が定義できる。

\begin{definition}[収束半径]
べき級数$\sum_{n = 0}^\infty a_n(x-a)^n$に対して
$$
\sup\lrset{|b-a| \mid \text{級数$\sum_{n = 0}^\infty a_n(b-a)^n$が収束する}}
$$
は非負の実数または正の無限大で、べき級数$\sum_{n = 0}^\infty a_n(x-a)^n$の\emph{収束半径}という。
\end{definition}

\begin{remark}
$r$を収束半径とするとき、$|x-a| > r$に対してはべき級数は必ず発散し、$|x-a| < r$に対しては先の命題よりべき級数は必ず絶対収束する。
$r = \infty$の場合では任意の$x$でべき級数は絶対収束する。
\end{remark}

\begin{remark}
収束半径という用語に「半径」という単語が入っているのはべき級数の理論を$a_n$や$x$が複素数の場合に拡張したときに絶対収束が言える$x$の範囲$|x-a| < r$が複素平面での円板になっていることによる。
実際、この章の内容は複素数の範囲に拡張でき、後の複素指数関数の部分で使われる。
\end{remark}

級数の収束に関するダランベールの判定法やコーシーの判定法から収束半径について次の公式が知られている。

\begin{theorem}
べき級数$\sum_{n = 0}^\infty a_n(x-a)^n$を考える。
\begin{itemize}
\item
さらに$(a_n)$はどの項も$0$でない数列として、極限
$$
\lim_{n \to \infty}\abs{\frac{a_{n+1}}{a_n}} = c
$$
が存在したとするとき、べき級数の収束半径は逆数$c^{-1}$となる。
\item
上極限
$$
\limsup_{n \to \infty}\sqrt[n]{\abs{a_n}} = c
$$
について、べき級数の収束半径は逆数$c^{-1}$となる。
\end{itemize}
\end{theorem}

\begin{example}
数$c$に対してべき級数
$$
\sum_{n = 0}^\infty c^n x^n = 1+c x+c^2 x^2+c^3 x^3+\cdots
$$
を考える。
$\lim_{n \to \infty}\abs{\frac{c^{n+1}}{c^n}} = |c|$よりこのべき級数の収束半径は$|c|^{-1}$である。
\end{example}

\begin{example}
数$c$に対してべき級数
$$
\sum_{n = 0}^\infty c^n x^{2 n} = 1+c x^2+c^2 x^4+c^3 x^6+\cdots
$$
を考える。
$\lim_{n \to \infty}\sqrt[2 n]{\abs{c^n}} = \sqrt{|c|}$よりこのべき級数の収束半径は$\sqrt{|c|}^{-1}$である。
\end{example}

ここからはいよいよ関数としてのべき級数を考えよう。
まず、べき級数$\sum_{n = 0}^\infty a_n(x-a)^n$の収束半径を$r$とするとき、関数
$$
f(x) = \sum_{n = 0}^\infty a_n(x-a)^n
$$
が$|x-a| < r$を満たす数$x$に対して定義されることに注意する。
また、$N = 0, 1, 2, 3, \cdots$に対して部分和を
$$
f_N(x) = \sum_{n = 0}^N a_n(x-a)^n
$$
とおく。
この部分和は$N$次以下の多項式である。

\begin{theorem}[べき級数の連続性]
収束半径が$r$であるべき級数$f(x) = \sum_{n = 0}^\infty a_n(x-a)^n$は$|x-a| < r$で連続である。
\end{theorem}

\begin{proof}
$|c-a| < r$を満たす数$c$と$\varepsilon > 0$を考える。
$\delta_1 = \frac{r-|c-a|}{2}$とおき、$b = a+|c-a|+\delta_1 = a+\frac{r+|c-a|}{2}$とおくと$|b-a| < r$なので、
$(f_N(b))_N$は$f(b)$に絶対収束することから、
$$
\sum_{n = N+1}^\infty |a_n||b-a|^n < \varepsilon
$$
となる自然数$N = 0, 1, 2, 3, \cdots$が存在する。
次に$f_N(x)$は多項式関数より、特に連続関数なので、ある$\delta_2 > 0$が存在し、任意の$|x-c| < \delta_2$を満たす$x$に対して$|f_N(x)-f_N(c)| < \varepsilon$である。
ここで$\delta = \min\lrset{ \delta_1, \delta_2 } > 0$とおくと$|x-c| < \delta$なる数$x$に対して、$|x-a| < |b-a| < r$より
$$
|f(x)-f_N(x)| = \abs{\sum_{n = N+1}^\infty a_n(x-a)^n} \le \sum_{n = N+1}^\infty |a_n||x-a|^n \le \sum_{n = N+1}^\infty |a_n||b-a|^n < \varepsilon
$$
である。
したがって以上より
$$
|f(x)-f(c)| \le |f(x)-f_N(x)|+|f_N(x)-f_N(c)|+|f_N(c)-f(c)|
\le \varepsilon+\varepsilon+\varepsilon = 3\varepsilon
$$
なので、$f(x)$は点$x = c$で連続であることがわかる。
\end{proof}

\begin{theorem}[べき級数の項別積分]
収束半径が$r$であるべき級数$f(x) = \sum_{n = 0}^\infty a_n(x-a)^n$に対して
$$
\int_a^b f(x)\dd{x} = \sum_{n = 0}^\infty \frac{a_n}{n+1}(b-a)^{n+1} = a_0(b-a)+\frac{a_1}{2}(b-a)^2+\frac{a_2}{3}(b-a)^3+\frac{a_3}{4}(b-a)^4+\cdots
$$
が任意の$|b-a| < r$を満たす実数$b$に対して成り立つ。
\end{theorem}

\begin{proof}
べき級数$\sum_{n = 0}^\infty \frac{a_n}{n+1}(b-a)^{n+1}$の収束半径が$r$であることに注意する。
$N = 0, 1, 2, 3, \cdots$に対して、
$$
\begin{aligned}
\abs{\int_a^b f(x)\dd{x}-\int_a^b f_N(x)\dd{x}}
&\le \abs{\int_a^b |f(x)-f_N(x)|\dd{x}}
\le \abs{\int_a^b \sum_{n = N+1}^\infty |a_n||x-a|^n\dd{x}} \\
&\le \abs{\int_a^b \sum_{n = N+1}^\infty |a_n||b-a|^n\dd{x}}
\le \sum_{n = N+1}^\infty |a_n||b-a|^{n+1}.
\end{aligned}
$$
よって$N \to \infty$で、$\int_a^b f_N(x)\dd{x} \to \int_a^b f(x)\dd{x}$であり、
$$
\int_a^b f_N(x)\dd{x} = \sum_{n = 0}^N \frac{a_n}{n+1}(b-a)^{n+1}
$$
であることから定理の結論が得られる。
\end{proof}

\begin{theorem}[べき級数の項別積分]
収束半径が$r$であるべき級数$f(x) = \sum_{n = 0}^\infty a_n(x-a)^n$は微分可能で
$$
f'(x) = \sum_{n = 1}^\infty n a_n(x-a)^{n-1} = \sum_{n = 0}^\infty (n+1) a_{n+1}(x-a)^n = a_1+2 a_2(x-a)+3 a_3(x-a)^2+\cdots
$$
が任意の$|x-a| < r$を満たす実数$x$に対して成り立つ。
\end{theorem}

\begin{proof}
べき級数$g(x) = \sum_{n = 1}^\infty n a_n(x-a)^{n-1}$の収束半径が$r$であることに注意する。
そこで前の定理より$g(x)$を項別積分すると
$$
\int_a^x g(t)\dd{t} = \sum_{n = 1}^\infty a_n(x-a)^n = f(x)-a_0
$$
よって、$f$は$g$の原始関数なので、$f$を微分すると$g$を得る。
\end{proof}

\begin{remark}
べき級数を微分して収束半径が変わらないべき級数になるので、微分することを何回も繰り返すことができ、べき級数は無限回微分可能であることわかる。
\end{remark}

\section{種々のべき級数}

この節ではこれまでに登場した関数とべき級数を結びつける。
自明なこととして多項式関数はそのままべき級数と見ることができる。

一般の(滑らかな)関数とべき級数を結びつけるのが次の定理でテイラーの定理から直ちに従う。

\begin{theorem}
$f(x)$を無限回微分可能な関数、$r$を非負の実数または正の無限大とする。
ここで、$|x| < r$なる実数$x$に対して
$$
\lim_{n \to \infty}\sup_{\theta \in [0, 1]}\frac{f^{(n)}(a+\theta(x-a))}{n!}(x-a)^n = 0
$$
が成り立つならば、
\begin{equation}
\label{e_taylor_exp}
f(x) = \sum_{n = 0}^\infty \frac{f^{(n)}(a)}{n!}(x-a)^n = 1+f'(a)(x-a)+\frac{f''(a)}{2}(x-a)^2+\frac{f^{(3)}(a)}{3!}(x-a)^3+\cdots \quad (|x| < r)
\end{equation}
が成り立つ。
\end{theorem}

このときの\eqref{e_taylor_exp}のことを関数$f(x)$の点$x = a$での\emph{テーラー展開}という。
また、$a = 0$のときのテーラー展開のことを\emph{マクローリン展開}という。

次の一覧は各種関数のテーラー展開(マクローリン展開)をまとめたものである。

\begin{proposition}[種々の関数のテーラー展開]
$x$を実数として以下が成立する。
\begin{equation}
\label{e_geom_power}
\frac{1}{1-x} = \sum_{n = 0}^\infty x^n = 1+x+x^2+x^3+\cdots \quad (|x| < 1).
\end{equation}
\begin{equation}
\label{e_exp_power}
\exp(x) = e^x = \sum_{n = 0}^\infty \frac{1}{n!}x^n = 1+x+\frac{1}{2}x^2+\frac{1}{3!}x^3+\cdots.
\end{equation}
\begin{equation}
\label{e_log_power}
\log(1+x) = \sum_{n = 1}^\infty \frac{(-1)^{n-1}}{n}x^n = x-\frac{1}{2}x^2+\frac{1}{3}x^3-\cdots \quad (|x| < 1).
\end{equation}
$$
\sin x = \sum_{k = 0}^\infty \frac{(-1)^k}{(2 k+1)!}x^{2 k+1} = x-\frac{1}{6}x^3+\frac{1}{5!}x^5-\cdots,
$$
$$
\cos x = \sum_{k = 0}^\infty \frac{(-1)^k}{(2 k)!}x^{2 k} = 1-\frac{1}{2}x^2+\frac{1}{4!}x^4-\cdots.
$$
$$
\sinh x = \sum_{k = 0}^\infty \frac{1}{(2 k+1)!}x^{2 k+1} = x+\frac{1}{6}x^3+\frac{1}{5!}x^5+\cdots,
$$
$$
\cosh x = \sum_{k = 0}^\infty \frac{1}{(2 k)!}x^{2 k} = 1+\frac{1}{2}x^2+\frac{1}{4!}x^4+\cdots.
$$
\end{proposition}

一般にテーラーの定理の剰余項が$0$に収束することを示すのは大変なので、既知のべき級数の項別微分、項別積分を利用して様々なテーラー展開を求めるとよい。

\begin{remark}
\eqref{e_geom_power}は等比級数からも導出することができる。
さらにそれを項別積分すると、
$$
\int_0^x \frac{1}{1-x}\dd{x} = -\log(1-x) = \sum_{n = 1}^\infty \frac{1}{n}x^n = x+\frac{1}{2}x^2+\frac{1}{3}x^3+\cdots \quad (|x| < 1).
$$
よって、$\log(1+x)$の式\eqref{e_log_power}を得ることができる。
\end{remark}

\begin{example}
関数$e^{x^2}$のテーラー展開は\eqref{e_exp_power}の$x$に$x^2$を代入することで
$$
e^{x^2} = \sum_{n = 0}^\infty \frac{1}{n!}x^{2 n} = 1+x^2+\frac{1}{2}x^4+\frac{1}{3!}x^6+\cdots
$$
と求めることができる。
なお、このべき級数の収束半径は$\infty$である。
\end{example}


\chapter{種々の関数2}

\section{複素指数関数}

指数関数はテーラー展開\eqref{e_exp_power}が知られているが、
べき級数の節で触れたようにべき級数の理論は複素数の範囲に自然と拡張できるので、
指数関数をべき級数によって定義することで定義される範囲を拡張することができる。
つまり、複素数$z$に対して
$$
\exp(z) = e^z = \sum_{n = 0}^\infty \frac{1}{n!}z^n = 1+z+\frac{1}{2}z^2+\frac{1}{3!}z^3+\cdots
$$
とする。

このときべき級数を比較すると、オイラーの公式
$$
e^{i\theta} = \cos\theta+i\sin\theta,
\quad e^{-i\theta} = \cos\theta-i\sin\theta
$$
が成立することがわかる。
そこで、これまで明確に定義されていなかった三角関数を複素指数関数を使って、
$$
\cos x = \frac{e^{i x}+e^{-i x}}{2},
\quad \sin x = \frac{e^{i x}-e^{-i x}}{2 i}
$$
によって定義することができる。
この形によっても三角関数が双曲線関数と似ていることがわかる。

この三角関数が必要な要件を満たしていることを確認していく。
まず、複素指数関数のテーラー展開から三角関数のテーラー展開が得られる。
特に、実数$x$に対して$\cos x$, $\sin x$は実数値連続関数である。

証明は本テキストの範囲外なので省略するが、指数法則
$$
e^{z+w} = e^z e^w
$$
が複素数$z, x$に対して成り立つことを認めれば、三角関数の加法定理が導かれる。
また、$\sin x$のべき級数から極限
$$
\lim_{x \to 0}\frac{\sin x}{x} = 1
$$
がわかる。

\section{シンク関数}

今まで例などで度々登場していた関数$\frac{\sin x}{x}$はシンク関数と呼ばれる。
正確には実数全体で定義された関数
$$
\sinc x =
\begin{cases}
\frac{\sin x}{x} & (x \ne 0), \\
1 & (x = 0) \\
\end{cases}
$$
を\emph{シンク関数}という。

シンク関数は極限の式から$\mathbb{R}$上の連続関数であることがわかるが、
べき級数での表示
$$
\sinc x = \sum_{k = 0}^\infty \frac{(-1)^k}{(2 k+1)!}x^{2 k} = 1-\frac{1}{6}x^2+\frac{1}{5!}x^4-\cdots
$$
が成り立つので、無限回微分可能な関数である。

シンク関数が登場した場面を思い出すと広義積分と級数の条件収束の例としてである。

ここでは広義積分$\int_0^\infty \sinc x\dd{x}$が収束するが、絶対収束しないことを示そう。
まず$\int_0^1 \sinc x$は普通の定積分であることに注意する。
$t > 1$に対して部分積分より
$$
\int_1^t \sinc x\dd{x} = \int_1^t \frac{(-\cos x)'}{x}\dd{x} = \lreval{\frac{(-\cos x)'}{x}}_1^t-\int_1^t \frac{\cos x}{x^2}\dd{x}
$$
となる。
ここで、
$$
\abs{\frac{\cos x}{x^2}} \le \frac{1}{x^2},
\quad \int_1^\infty \frac{1}{x^2}\dd{x} < \infty
$$
なので、広義積分$\int_1^\infty \frac{\cos x}{x^2}\dd{x}$は収束し、広義積分$\int_0^\infty \sinc x\dd{x}$も収束が言える。
一方で自然数$N = 0, 1, 2, 3, \cdots$に対して
$$
\int_0^{2\pi N} |\sinc x|\dd{x} = \sum_{n = 1}^N \int_{2\pi(n-1)}^{2\pi n} \frac{|\sin x|}{x}\dd{x} \ge \sum_{n = 1}^N \int_{2\pi(n-1)}^{2\pi n} \frac{|\sin x|}{2\pi n}\dd{x} = \sum_{n = 1}^N \frac{2}{\pi n}
$$
であり、$N \to \infty$とすると右辺は正の無限大に発散するので、広義積分$\int_0^\infty \sinc x\dd{x}$は絶対収束しない。

ちなみに広義積分$\int_0^\infty \sinc x\dd{x}$の値は$\frac{\pi}{2}$であることが知られていて\emph{ディリクレ積分}と呼ばれるが、証明には複素解析学の知識が必要であり本テキストの範囲を逸脱するので取り扱わない。

\section{ガウス関数}

やはり広義積分の部分で登場した$x$について関数
$$
\exp(-x^2) = e^{-x^2}
$$
は\emph{ガウス関数}と呼ばれる。
あるいは実数定数$a$と正定数$c$をつけて
$$
\frac{1}{\sqrt{2\pi}c}\exp\qty(-\frac{(x-a)^2}{2 c^2}) = \frac{1}{\sqrt{2\pi}c}e^{-\frac{(x-a)^2}{2 c^2}}
$$
と一般化することもあるが、最初のもののみ考える。
ガウス関数は確率統計学で現れ、正規分布と強い結びつきのある重要な関数である。

ガウス関数はべき級数での表示
$$
\exp(-x^2) = \sum_{n = 0}^\infty \frac{1}{n!}(-x^2)^n = 1-\frac{1}{2}x^2+\frac{1}{4!}x^4-\frac{1}{6!}x^6+\cdots
$$
があり、収束半径は$\infty$である。

一方で原始関数はこれまでに登場した関数で表すことができないことが知られている(ガウスの誤差関数と呼ばれる)。
しかしながら広義積分
$$
\int_{-\infty}^{+\infty}\exp(-x^2)\dd{x} = \int_{-\infty}^{+\infty} e^{-x^2}\dd{x}
$$
の値は計算できる。
これを\emph{ガウス積分}という。

ガウス積分の標準的かつ簡便な計算方法は重積分の極座標変換を用いるものだが、本テキストの範囲外なので、ここでは煩雑ではあるが範囲内の知識で説明できる方法を紹介する。
そのためにまず不等式
$$
1-x^2 \le e^{-x^2} \le \frac{1}{1+x^2}
$$
を用意する。
これは微分法により任意の実数$y$に対して$e^y \ge 1+y$であることから、$y = \pm x^2$を当てはめて整理すれば得られる。
この不等式を$n = 1, 2, 3, \cdots$乗して積分すると次が得れれる。
$$
\int_0^1 (1-x^2)^n\dd{x} \le \int_0^1 e^{-n x^2}\dd{x} \le \int_0^1 \frac{1}{(1+x^2)^n}\dd{x}.
$$
ここで、左辺は$x = \sin\theta$と置換することにより
$$
\int_0^1 (1-x^2)^n\dd{x} = \int_0^{\frac{\pi}{2}} \cos^{2 n+1}\theta\dd{\theta},
$$
真ん中の式は$\sqrt{n}x$を$x$と置換することにより
$$
\int_0^1 e^{-n x^2}\dd{x} = \frac{1}{\sqrt{n}} e^{-x^2}\dd{x},
$$
右辺は$x = \tan\theta$と置換することにより
$$
\int_0^1 \frac{1}{(1+x^2)^n}\dd{x} = \int_0^{\frac{\pi}{4}} \cos^{2 n-2}\theta\dd{\theta} \le \int_0^{\frac{\pi}{2}} \cos^{2 n-2}\theta\dd{\theta}
$$
なので、ウォリス積分$I_n = \int_0^{\frac{\pi}{2}} \cos^{2 n+1}\theta\dd{\theta}$を用いて、
$$
\sqrt{n}I_{2 n+1} \le \int_0^{\sqrt{n}} e^{-x^2}\dd{x} \le \sqrt{n}I_{2 n-2}
$$
である。
したがって、$n \to \infty$とするとウォリスの公式(定理\ref{t_wallis_limit})よりガウス積分の値が
$$
\int_0^\infty \exp(-x^2)\dd{x} = \int_0^\infty e^{-x^2}\dd{x} = \frac{1}{\sqrt{2}}\cdot\sqrt{\frac{\pi}{2}} = \frac{\sqrt{\pi}}{2},
\quad \int_{-\infty}^{+\infty} \exp(-x^2)\dd{x} = \int_{-\infty}^{+\infty} e^{-x^2}\dd{x} = \sqrt{\pi}
$$
であることがわかる。

\section{ガンマ関数}

正の実数$s > 0$に対して広義積分
$$
\int_0^\infty x^{s-1}e^{-x}\dd{x}
$$
は収束する。
これを示すためには被積分関数$f(x) = x^{s-1}e^{-x}$に対して優関数を$g(x) = \frac{1}{x^2}$の定数倍として取るために
$$
\frac{f(x)}{g(x)} = x^{s+1}e^{-x} = h(x)
$$
の有界性を示す。
実際、
$$
h'(x) = (s+1)x^s e^{-x}-x^{s+1}e^{-x} = (s+1-x)x^s e^{-x}
$$
より$h(x)$は$[0, s+1]$で単調増加し、$[s+1, \infty)$で単調減少するので、$x = s+1$で最大となる。
よって、$f(x) \le \max h \frac{1}{x^2}$であり積分$\int_1^\infty \frac{1}{x^2}\dd{x}$は収束するので、広義積分$\int_1^\infty x^{s-1}e^{-x}\dd{x}$も収束する。
$\int_0^1 x^{s-1}e^{-x}\dd{x}$については$f(x) \le x^{s-1}$であり、$s > 0$に注意すると、積分$\int_0^1 x^{s-1}\dd{x}$は収束するので、やはり収束する。

以上より$s > 0$に対して広義積分の値を取る関数
$$
\Gamma(s) = \int_0^\infty x^{s-1}e^{-x}\dd{x}
$$
が定義でき、\emph{ガンマ関数}という。

\section{ベータ関数}

% 
\chapter{一様分布論}

\section{一様分布定理}

この章では一様分布定理と呼ばれる今までとは少し趣の異なる極限に関する定理を紹介する。
内容的には高度なものになるので読み物としてみた方がよいかもしれない。

その一様分布定理を述べるために次の用語を定義する。

\begin{definition}[一様分布列]
各$n$に対して$x_n \in [0, 1)$を満たす数列$(x_n)_{n = 1}^\infty$が$[0, 1)$で\emph{一様分布}するとは任意の$0 \le a < b \le 1$に対して
$$
\lim_{N \to \infty}\frac{1}{N}\#\lrset{ n = 1, \cdots, N \mid a \le x_n < b } = b-a
$$
が成り立つことである。
ただし、有限集合$A$に対して、自然数として定まる元の個数を$\# A$と書く。
\end{definition}

この条件は区間$[a, b)$に入る$x_n$の割合が$b-a$であるとみなせる。
以降では$n = 1, \cdots, N$をしばしば$n \le N$と表現する。

$[0, 1)$という区間は実数$x$の床(整数部分)$\lrfloor{x}$を引いた小数部分$\lrfrac{x}$の属する集合として現れた。
そこで一般の実数列$(x_n)$に対しても小数部分の列$(\lrfrac{x})$が一様分布するかどうかで一様分布列の概念を導入する。
なお、$\lrfrac{x}$という記号は集合の記号と紛らわしいがよく使われるので本テキストでも採用し文脈によってどちらの意味か判断される。

\begin{remark}
小数部分の演算について任意の実数$x, y$に対して、$\lrfrac{x} \ge \lrfrac{y}$である場合は$\lrfrac{x-y} = \lrfrac{x}-\lrfrac{y}$であり、そうでない場合は$\lrfrac{x-y} = \lrfrac{x}-\lrfrac{y}+1$であることに注意する。
\end{remark}

以上の準備の下で一様分布定理は次のように表現される。

\begin{theorem}[一様分布定理]
$\theta$を無理数とするとき、数列$(\lrfrac{n\theta})_{n = 1}^\infty$は一様分布する。
\end{theorem}

$[0, 1)$の$0$と$1$と引っ付けてわっかのように見なすと、$(\lrfrac{n\theta})_n$というのはわっかをぐるぐるまわる感じになっており回転と呼ばれ$\theta$は角度のような役割を果たす。

この定理において$\theta$が無理数であることが非常に重要で、例えば$\theta$が有理数$\frac{1}{p}$の場合は$n$が$p$増えるともとの位置に戻るため、$\lrfrac{n\theta}$の取りうる値は$\lrset{ \frac{0}{p}, \cdots \frac{p-1}{p} }$となり一様分布にはならない。
無理数の場合は表現が正確ではないが分母$p$が無限大のような状況になっていて、$(\lrfrac{n\theta})_n$は$[0, 1)$上均等にまんべんなく分布するということを定理は主張している。

一様分布定理と関連の深い話題として次のクロネッカーの稠密定理がある。

\begin{theorem}[クロネッカーの稠密定理]
$\theta$を無理数とするとき、数列$(\lrfrac{n\theta})_{n = 1}^\infty$は稠密である、つまり任意の$0 \le a < b \le 1$に対して$a \le \lrfrac{k\theta} < b$を満たす$k = 1, \cdots$が存在する。
\end{theorem}

一様分布するならば自動的に稠密であるので、クロネッカーの稠密定理は一様分布定理より弱い内容を主張する定理である。
しかしながら、今回の一様分布定理の証明ではクロネッカーの稠密定理を利用するので証明を与える。

\begin{proof}[クロネッカーの稠密定理の証明]
まず、$N = \lrceil{\frac{2}{b-a}}$とするとある$l = 0, \cdots, N-1$が存在して$a \le \frac{l}{N} < \frac{l+1}{N} \le b$とできることに注意する。
この$N = 1, 2, 3, \cdots$に対して次の議論をする。
つまり区間$[0, 1)$を$N$等分すると$(\lrfrac{n\theta})_{n = 1}^{N+1}$のうち$2$つは同じ小区間の中に入り$i < j$とおく。
このとき$\theta$が無理数であることから、$0 < |\lrfrac{i\theta}-\lrfrac{j\theta}| < \frac{1}{N}$である。
よって、$k$を$j-i$とすると$0 < \lrfrac{k\theta} < \frac{1}{N}$または$1-\frac{1}{N} < \lrfrac{k\theta} < 1$であり、
これを何倍かすると$a \le \frac{l}{N} \le \lrfrac{nk\theta} < \frac{l+1}{N} \le b$となる$n$が存在するので稠密性が示された。
\end{proof}

一様分布定理の証明はワイルの証明が知られているがここでは別の証明を与える。
まず、次の補題を準備する。

\begin{lemma}
$(x_n)_{n = 1}^\infty$を実数列とする。
数列$(\lrfrac{x_n})$が一様分布することと任意の実数$a$に対して
$$
\lim_{N \to \infty}\qty(\frac{1}{N}\sum_{n \le N}\lrfrac{x_n}-\frac{1}{N}\sum_{n \le N}\lrfrac{x_n+a}) = 0
$$
が成り立つことは同値である。
\end{lemma}

\begin{proof}
まず一様分布であることの定義より、$(\lrfrac{x_n})$が一様分布することは任意の$0 \le b \le 1$に対して
$$
\lim_{N \to \infty}\frac{1}{N}\#\lrset{ n \le N \mid x_n < b } = b
$$
が成り立つことと同値であることに注意する。
実数$b$に対して、
$$
\begin{aligned}
\frac{1}{N}\sum_{n \le N}\lrfrac{x_n-b}
&= \frac{1}{N}\sum_{\lrfrac{x_n} \ge \lrfrac{b}}(\lrfrac{x_n}-\lrfrac{b})+\frac{1}{N}\sum_{\lrfrac{x_n} < \lrfrac{b}}(\lrfrac{x_n}-\lrfrac{b}+1) \\
&= \frac{1}{N}\sum_{n \le N}\lrfrac{x_n}+\frac{1}{N}\#\lrset{ n \le N \mid \lrfrac{x_n} < \lrfrac{b} }-\lrfrac{b}.
\end{aligned}
$$
ここから補題が成立することがわかる。
\end{proof}

\begin{proof}[一様分布定理の証明]
$0 \le a < 1$と$\varepsilon > 0$を固定する。
$(\lrfrac{n\theta})_{n = 1}^\infty$の稠密性は示されているので、
$$
a \le \lrfrac{k\theta} < a+\varepsilon
$$
となる$k = 1, 2, 3, \cdots$が存在する。
ここで任意の整数$n$に対して
$$
\lrfrac{n\theta} < \lrfrac{(n-k)\theta+a}+\varepsilon
$$
を示す。
実際、
$$
\lrfrac{(n-k)\theta+a} = \lrfrac{n\theta-(k\theta-a)} \ge \lrfrac{n\theta}-\lrfrac{k\theta-a}
$$
で、$a \le \lrfrac{k\theta} < a+\varepsilon$より、
$$
\lrfrac{(n-k)\theta+a} \ge \lrfrac{n\theta}-\lrfrac{k\theta}-a > \lrfrac{n\theta}-\varepsilon
$$
である。
したがって、
$$
\sum_{n \le N}\lrfrac{n\theta}-\sum_{n \le N}\lrfrac{n\theta+a}
= \sum_{n \le k}\lrfrac{n\theta}+\sum_{k < n \le N}\lrfrac{n\theta}-\sum_{n \ge N-k}\lrfrac{n\theta+a}-\sum_{N-k < n \le N}\lrfrac{n\theta+a}
\le k+(N-k)\varepsilon.
$$
よって、
$$
\limsup_{N \to \infty}\qty(\frac{1}{N}\sum_{n \le N}\lrfrac{n\theta}-\frac{1}{N}\sum_{n \le N}\lrfrac{n\theta+a}) \le \varepsilon.
$$
また、$a-\varepsilon < \lrfrac{k\theta} \le a$として$k$を取り同様のことをすれば
$$
\liminf_{N \to \infty}\qty(\frac{1}{N}\sum_{n \le N}\lrfrac{n\theta}-\frac{1}{N}\sum_{n \le N}\lrfrac{n\theta+a}) \ge -\varepsilon.
$$
よって$\varepsilon$は任意であることと補題より一様分布定理が証明された。
\end{proof}


\printindex

% \bibliographystyle{amsplain}
% \bibliography{references}

% \begin{thebibliography}{10}

% \end{thebibliography}

\end{document}
