
\chapter{種々の関数2}

\section{複素指数関数}

指数関数はテーラー展開\eqref{e_exp_power}が知られているが、
べき級数の節で触れたようにべき級数の理論は複素数の範囲に自然と拡張できるので、
指数関数をべき級数によって定義することで定義される範囲を拡張することができる。
つまり、複素数$z$に対して
$$
\exp(z) = e^z = \sum_{n = 0}^\infty \frac{1}{n!}z^n = 1+z+\frac{1}{2}z^2+\frac{1}{3!}z^3+\cdots
$$
とする。

このときべき級数を比較すると、オイラーの公式
$$
e^{i\theta} = \cos\theta+i\sin\theta,
\quad e^{-i\theta} = \cos\theta-i\sin\theta
$$
が成立することがわかる。
そこで、これまで明確に定義されていなかった三角関数を複素指数関数を使って、
$$
\cos x = \frac{e^{i x}+e^{-i x}}{2},
\quad \sin x = \frac{e^{i x}-e^{-i x}}{2 i}
$$
によって定義することができる。
この形によっても三角関数が双曲線関数と似ていることがわかる。

この三角関数が必要な要件を満たしていることを確認していく。
まず、複素指数関数のテーラー展開から三角関数のテーラー展開が得られる。
特に、実数$x$に対して$\cos x$, $\sin x$は実数値連続関数である。

証明は本テキストの範囲外なので省略するが、指数法則
$$
e^{z+w} = e^z e^w
$$
が複素数$z, x$に対して成り立つことを認めれば、三角関数の加法定理が導かれる。
また、$\sin x$のべき級数から極限
$$
\lim_{x \to 0}\frac{\sin x}{x} = 1
$$
がわかる。

\section{シンク関数}

\section{ガンマ関数}

\section{ベータ関数}
