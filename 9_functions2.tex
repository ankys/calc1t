
\chapter{種々の関数2}

\section{複素指数関数}

指数関数はテーラー展開\eqref{e_exp_power}が知られているが、
べき級数の節で触れたようにべき級数の理論は複素数の範囲に自然と拡張できるので、
指数関数をべき級数によって定義することで定義される範囲を拡張することができる。
つまり、複素数$z$に対して
$$
\exp(z) = e^z = \sum_{n = 0}^\infty \frac{1}{n!}z^n = 1+z+\frac{1}{2}z^2+\frac{1}{3!}z^3+\cdots
$$
とする。

このときべき級数を比較すると、オイラーの公式
$$
e^{i\theta} = \cos\theta+i\sin\theta,
\quad e^{-i\theta} = \cos\theta-i\sin\theta
$$
が成立することがわかる。
そこで、これまで明確に定義されていなかった三角関数を複素指数関数を使って、
$$
\cos x = \frac{e^{i x}+e^{-i x}}{2},
\quad \sin x = \frac{e^{i x}-e^{-i x}}{2 i}
$$
によって定義することができる。
この形によっても三角関数が双曲線関数と似ていることがわかる。

この三角関数が必要な要件を満たしていることを確認していく。
まず、複素指数関数のテーラー展開から三角関数のテーラー展開が得られる。
特に、実数$x$に対して$\cos x$, $\sin x$は実数値連続関数である。

証明は本テキストの範囲外なので省略するが、指数法則
$$
e^{z+w} = e^z e^w
$$
が複素数$z, x$に対して成り立つことを認めれば、三角関数の加法定理が導かれる。
また、$\sin x$のべき級数から極限
$$
\lim_{x \to 0}\frac{\sin x}{x} = 1
$$
がわかる。

\section{シンク関数}

今まで例などで度々登場していた関数$\frac{\sin x}{x}$はシンク関数と呼ばれる。
正確には実数全体で定義された関数
$$
\sinc x =
\begin{cases}
\frac{\sin x}{x} & (x \ne 0), \\
1 & (x = 0) \\
\end{cases}
$$
を\emph{シンク関数}という。

シンク関数は極限の式から$\mathbb{R}$上の連続関数であることがわかるが、
べき級数での表示
$$
\sinc x = \sum_{k = 0}^\infty \frac{(-1)^k}{(2 k+1)!}x^{2 k} = 1-\frac{1}{6}x^2+\frac{1}{5!}x^4-\cdots
$$
が成り立つので、無限回微分可能な関数である。

シンク関数が登場した場面を思い出すと広義積分と級数の条件収束の例としてである。

ここでは広義積分$\int_0^\infty \sinc x\dd{x}$が収束するが、絶対収束しないことを示そう。
まず$\int_0^1 \sinc x$は普通の定積分であることに注意する。
$t > 1$に対して部分積分より
$$
\int_1^t \sinc x\dd{x} = \int_1^t \frac{(-\cos x)'}{x}\dd{x} = \lreval{\frac{(-\cos x)'}{x}}_1^t-\int_1^t \frac{\cos x}{x^2}\dd{x}
$$
となる。
ここで、
$$
\abs{\frac{\cos x}{x^2}} \le \frac{1}{x^2},
\quad \int_1^\infty \frac{1}{x^2}\dd{x} < \infty
$$
なので、広義積分$\int_1^\infty \frac{\cos x}{x^2}\dd{x}$は収束し、広義積分$\int_0^\infty \sinc x\dd{x}$も収束が言える。
一方で自然数$N = 0, 1, 2, 3, \cdots$に対して
$$
\int_0^{2\pi N} |\sinc x|\dd{x} = \sum_{n = 1}^N \int_{2\pi(n-1)}^{2\pi n} \frac{|\sin x|}{x}\dd{x} \ge \sum_{n = 1}^N \int_{2\pi(n-1)}^{2\pi n} \frac{|\sin x|}{2\pi n}\dd{x} = \sum_{n = 1}^N \frac{2}{\pi n}
$$
であり、$N \to \infty$とすると右辺は正の無限大に発散するので、広義積分$\int_0^\infty \sinc x\dd{x}$は絶対収束しない。

ちなみに広義積分$\int_0^\infty \sinc x\dd{x}$の値は$\frac{\pi}{2}$であることが知られていて\emph{ディリクレ積分}と呼ばれるが、証明には複素解析学の知識が必要であり本テキストの範囲を逸脱するので取り扱わない。

\section{ガウス関数}

やはり広義積分の部分で登場した$x$について関数
$$
\exp(-x^2) = e^{-x^2}
$$
は\emph{ガウス関数}と呼ばれる。
あるいは実数定数$a$と正定数$c$をつけて
$$
\frac{1}{\sqrt{2\pi}c}\exp\qty(-\frac{(x-a)^2}{2 c^2}) = \frac{1}{\sqrt{2\pi}c}e^{-\frac{(x-a)^2}{2 c^2}}
$$
と一般化することもあるが、最初のもののみ考える。
ガウス関数は確率統計学で現れ、正規分布と強い結びつきのある重要な関数である。

ガウス関数はべき級数での表示
$$
\exp(-x^2) = \sum_{n = 0}^\infty \frac{1}{n!}(-x^2)^n = 1-\frac{1}{2}x^2+\frac{1}{4!}x^4-\frac{1}{6!}x^6+\cdots
$$
があり、収束半径は$\infty$である。

一方で原始関数はこれまでに登場した関数で表すことができないことが知られている(ガウスの誤差関数と呼ばれる)。
しかしながら広義積分
$$
\int_{-\infty}^{+\infty}\exp(-x^2)\dd{x}
$$
の値は計算できる。
これを\emph{ガウス積分}という。
ガウス積分の標準的かつ簡便な計算方法は重積分の極座標変換を用いるものだが、本テキストの範囲外なので、ここでは煩雑ではあるが範囲内の知識で説明できる方法を紹介する。



\section{ガンマ関数}

\section{ベータ関数}
